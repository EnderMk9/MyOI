% Load the kaobook class
\documentclass[
	fontsize=10.8pt, % Base font size
	twoside=true, % Use different layouts for even and odd pages (in particular, if twoside=true, the margin column will be always on the outside)
	open=any, % If twoside=true, uncomment this to force new chapters to start on any page, not only on right (odd) pages
	secnumdepth=0, % How deep to number headings. Defaults to 1 (sections)
]{kaobook}

% Choose the language
\usepackage[english]{babel} % Load characters and hyphenation
\usepackage[english=british]{csquotes}	% English quotes

% Load packages for testing
\usepackage{blindtext}
%\usepackage{showframe} % Uncomment to show boxes around the text area, margin, header and footer
%\usepackage{showlabels} % Uncomment to output the content of \label commands to the document where they are used

% Load the bibliography package
\usepackage{kaobiblio}
\addbibresource{minimal.bib} % Bibliography file

% Load mathematical packages for theorems and related environments
\usepackage{kaotheorems}

% Load the package for hyperreferences
\usepackage{kaorefs}

\graphicspath{{images/}{./}} % Paths where images are looked for

\makeindex[columns=3, title=Alphabetical Index, intoc] % Make LaTeX produce the files required to compile the index

% ENDER

\usepackage{color}

\counterwithout{subsubsection}{subsection}
\let\theequation\thesubsection
\let\thesubsubsection\thesubsection
\makeatletter
\let\mytagform@=\tagform@
\def\tagform@#1{\maketag@@@{\color{Black}(#1)}}
\makeatother

\usepackage{hyperref}
\newcommand{\inlineeqnum}{~~\color{Black}\mbox{(\theequation)}\refstepcounter{subsection}}

\usepackage{graphicx}
\usepackage{mathbbol}
\usepackage{tikz}

\input RoyalIn.fd
\newcommand*\initfamily{\usefont{U}{RoyalIn}{xl}{n}}
\input Sanremo.fd
\newcommand*\initfamilya{\usefont{U}{Sanremo}{xl}{n}}

\usepackage{beuron}
\usepackage{cancel}
%\usepackage{mathtools}
%\usepackage{calrsfs}

\DeclareMathAlphabet{\pazocal}{OMS}{zplm}{m}{n}

% Reset sidenote counter at chapters
\counterwithin*{sidenote}{chapter}

\usepackage[bottom]{footmisc}
\raggedbottom
\begin{document}

%----------------------------------------------------------------------------------------
%	BOOK INFORMATION
%----------------------------------------------------------------------------------------

%\titlehead{Document Template}
\title[Mecánica Clásica]{{\fontsize{75pt}{0pt}\selectfont \initfamily{MECANICA
CLÁSICA}}}
\author[Abel Rosado]{\fontsize{30pt}{0pt}\selectfont \initfamilya{ABEL ROSADO} \fontsize{12pt}{0pt}\\
\url{https://github.com/EnderMk9/MyOI}}
\date{\today}
%\publishers{An Awesome Publisher}

%----------------------------------------------------------------------------------------

\frontmatter % Denotes the start of the pre-document content, uses roman numerals

%----------------------------------------------------------------------------------------
%	COPYRIGHT PAGE
%----------------------------------------------------------------------------------------

\makeatletter
\uppertitleback{\@titlehead} % Header

\lowertitleback{
	\textbf{Copyright} \\
	\ccby\ This work is licensed under a Creative Commons Attribution 4.0 International License.
	
	\medskip
	
	\textbf{Credits} \\
	This document was typeset with the help of \href{https://www.latex-project.org/}{\LaTeX} using the \href{https://github.com/fmarotta/kaobook/}{kaobook} class.
	
}
\makeatother

%----------------------------------------------------------------------------------------
%	DEDICATION
%----------------------------------------------------------------------------------------

\dedication{
	Come, let us hasten to a higher plane

	Where dyads tread the fairy fields of Venn,

	Their indices bedecked from one to n

	Commingled in an endless Markov chain!\\

	\vspace{20pt}

	In Riemann, Hilbert or in Banach space
	
	Let superscripts and subscripts go their ways
	
	Our asymptotes no longer out of phase,

	We shall encounter, counting, face to face.\\

	\vspace{20pt}

	For what did Cauchy know, or Christoffel,

	Or Fourier, or any Boole or Euler,

	Wielding their compasses, their pens and rulers,

	Of thy supernal sinusoidal spell?\\

	\vspace{20pt}

	Ellipse of bliss, converge, O lips divine!

	The product of our scalars is defined!

	Cyberiad draws nigh, and the skew mind

	Cuts capers like a happy haversine.\\

	\vspace{20pt}

	I see the eigenvalue in thine eye,

	I hear the tender tensor in thy sigh.

	Bernoulli would have been content to die,

	Had he but known such $a^2 \cos2\varphi$!
	\flushright -- Stanislaw Lem, The Cyberiad
}

%----------------------------------------------------------------------------------------
%	OUTPUT TITLE PAGE AND PREVIOUS
%----------------------------------------------------------------------------------------

% Note that \maketitle outputs the pages before here
\maketitle

%----------------------------------------------------------------------------------------
%	PREFACE
%----------------------------------------------------------------------------------------

% \chapter*{Preface}

% TO-DO

%----------------------------------------------------------------------------------------
%	TABLE OF CONTENTS & LIST OF FIGURES/TABLES
%----------------------------------------------------------------------------------------

\begingroup % Local scope for the following commands

% Define the style for the TOC, LOF, and LOT
%\setstretch{1} % Uncomment to modify line spacing in the ToC
%\hypersetup{linkcolor=blue} % Uncomment to set the colour of links in the ToC
\setlength{\textheight}{230\vscale} % Manually adjust the height of the ToC pages

% Turn on compatibility mode for the etoc package
\etocstandarddisplaystyle % "toc display" as if etoc was not loaded
\etocstandardlines % "toc lines as if etoc was not loaded

\tableofcontents % Output the table of contents

%\listoffigures % Output the list of figures

% Comment both of the following lines to have the LOF and the LOT on different pages
\let\cleardoublepage\bigskip
\let\clearpage\bigskip

%\listoftables % Output the list of tables

\endgroup

%----------------------------------------------------------------------------------------
%	MAIN BODY
%----------------------------------------------------------------------------------------

\mainmatter % Denotes the start of the main document content, resets page numbering and uses arabic numbers
\setchapterstyle{kao} % Choose the default chapter heading style

\pagelayout{wide} % No margins
\addpart[Mecánica Analítica]{\fontsize{55pt}{0pt} \setstretch{2} \textbeuron{Mecanica Analitica}}
\pagelayout{margin} % Restore margins

\chapter{Cálculo Variacional}

%-------------------------------------------------------------------------------------------
\refstepcounter{subsection}
Tenemos una función $f:\mathbcal{U}\in \mathbb{R} \mapsto f(x)\in \mathbb{R}$, donde tanto el dominio $\mathbcal{U}$ como la imagen pertencen a $\mathbb{R}$.
En contraposición, un funcional es una función $F: \mathbcal{f} \in \mathcal{F}\{x,\mathbb{R}\} \mapsto F[f]\in \mathbb{R}$, donde $\mathcal{F}\{x,\mathbb{R}\}$ es el conjunto de todas las funciones reales de una variable \sidenote[]{Aunque se puede definir un funcional como una función de $\mathcal{F}\{x,\mathbb{R}\}^n$ para $n$ funciones reales.}, tal que la imagen es un número real.

La forma genérica de los funcionales que nos interesan es la siguiente, donde ';' indica que $x$ es la variable independiente, y $f$ y $f'$ dependen explícitamente de x, y por consiguiente depende entre sí, aunque no de forma explícita en la mayoría de circunstancias:
\begin{equation}
    F[f]=\int_{x_A}^{x_b}{g(f(x),f'(x);x)dx} \label{1.0.1}
\end{equation} \refstepcounter{subsection}
Nos interesan solo las funciones $f$ tales que $f(x_A)=y_A; \ f(x_B)=y_B \label{1.0.2} \inlineeqnum$, de tal forma que la función este fija en los extremos de la integral, esta propiedad va a resultar muy importante más adelante.
    
El principal objetivo que tenemos en mente es encontrar una $f$ que extremize $F$, es decir, que $F(f)$ sea un máximo o mínimo del funcional.
%-------------------------------------------------------------------------------------------
\section{Método de pequeñas variaciones} \refstepcounter{subsection}
\begin{marginfigure}[0cm]
	\includegraphics{1}
	\labfig{margin1}
\end{marginfigure}
Definimos $\delta y(x)\equiv\bar{y}(x)-y(x) \label{1.1.1} \inlineeqnum$, donde $\bar{y}$ es el camino variado e $y$ es el camino de referencia. Supondremos que el camino de referencia es el camino que extremiza el funcional, entonces una pequeña variación $\delta y$ no debería alterar el funcional.

Podemos parametrizar $\delta y(x) \equiv a \eta(x) \label{1.1.2} \inlineeqnum$, donde $a$ es un parámetro independiente de $x$ y $\eta(x)=\delta y(x)/a \label{1.1.2} \inlineeqnum$ es una función arbitraria que da forma el camino variado y que debe cumplir que $\eta(x_A)=\eta(x_B)=0 \label{1.1.4} \inlineeqnum$ para verificar las condiciones que hemos impuesto en (1.0.2), ya que todo camino, sea el de referencia o el variado, debe cumplirlas.

Definimos entonces una nueva función $Y(x,a)\equiv y(x)+a\eta(x) \label{1.1.5} \inlineeqnum$ tal que $Y(x,0)=y(x)$ y $Y(x,a)=\bar{y}(x)$. Si derivamos esta función con repecto a $a$, y con respecto a $x$ tenemos
\begin{equation}\label{1.1.6}
\frac{\partial Y}{\partial a}=\eta(x) ; \ \ \frac{\partial Y}{\partial x}=y'(x)+a\eta'(x)\equiv Y'(x,a); \ \  \frac{\partial Y'}{\partial a} = \eta ' (x)
\end{equation} \refstepcounter{subsection}
Podemos definir ahora $\delta y'(x) \equiv \bar{y}'(x)-y'=Y'(x,a)-Y'(x,0)$, que por la expresión anterior nos resulta $\delta y'(x)=a \eta'(x) \label{1.1.7} \inlineeqnum$.
Combinando ahora (1.1.7) y (1.1.2) podemos llegar a la conclusión de que la derivada y $\delta$ conmutan
\begin{equation}\label{1.1.8}
    \delta y'(x)=a \frac{d}{dx} \eta{x}=\frac{d}{dx}\left(a\eta(x)\right)=\frac{d}{dx} \delta y \implies \delta \left(\frac{dy}{dx}\right)=\frac{d}{dx} \delta y 
\end{equation} \refstepcounter{subsection}
%-------------------------------------------------------------------------------------------
\subsection{Variación de una función}
Si partimos de una función $g(y,y';x)$, queremos que no dependa de un solo camino sino de una familia de ellos,  definimos $\mathbb{g}(x,a)=g(Y,Y';x)$. Definimos la variación total de la función como $\Delta \mathbb{g} \equiv \mathbb{g}(Y(x,a),Y'(x,a);x)-\mathbb{g}(Y(x,0),Y'(x,0);x) \label{1.1.9} \inlineeqnum$. Como últimamente $\mathbb{g}$ depende solo de $x$ y de $a$, podemos expandir $\mathbb{g}$ por serie de Taylor de $a$
\begin{equation} \label{1.1.10}
    \mathbb{g}(x,a) = \mathbb{g}(x,0)+\left.\frac{\partial \mathbb{g}}{\partial a}\right|_{a=0} a + O(a^2)
\end{equation} \refstepcounter{subsection}
Reorganizando los términos y volviendo a añadir la dependiencia en $Y$ e $Y'$ llegamos a 
\begin{equation} \label{1.1.11}
    \overbrace{\mathbb{g}(x,a) - \mathbb{g}(x,0) }^{\Delta \mathbb{g}} = \overbrace{\left.\frac{\partial \mathbb{g}(Y(x,a),Y'(x,a);x)}{\partial a}\right|_{a=0} a}^{\delta \mathbb{g}} + O(a^2)
\end{equation} \refstepcounter{subsection}
Donde $\delta \mathbb{g}$ es la variación primera de la función, que podemos reescribir desarrollando la derivada usando la regla de la cadena, y usamos (1.1.2) y (1.1.7)
\begin{equation} \label{1.1.12}
    \delta \mathbb{g}= \left.\left[\left.\frac{\partial \mathbb{g}}{\partial Y}\right|_{Y} \frac{\partial Y}{\partial a} + \left.\frac{\partial \mathbb{g}}{\partial Y'}\right|_{Y} \frac{\partial Y'}{\partial a}\right]\right|_{a=0} a = \left.\frac{\partial \mathbb{g}}{\partial Y}\right|_{y} a \eta + \left.\frac{\partial \mathbb{g}}{\partial Y'}\right|_{y} a \eta' = \left.\frac{\partial \mathbb{g}}{\partial Y}\right|_{y} \delta y + \left.\frac{\partial \mathbb{g}}{\partial Y'}\right|_{y} \delta y'
\end{equation} \refstepcounter{subsection}
Es \textbf{muy} importante no dejar de lado las composiciones y evaluaciones resultantes de hacer Taylor y la regla de la cadena, ya que la expresión anterior nos indica que aunque $g$ dependa de cualquier camino, cuando hacemos $\delta g$, las parciales de $g$ con respecto a sus entradas $Y$ e $Y'$ hay que \textbf{evaluarlas en el camino de referencia} $y=Y(x,0)$. De esta forma podemos reesribir (1.1.12) en términos de $g$
\begin{equation} \label{1.1.12}
    \delta \mathbb{g} = \delta g = \frac{\partial g}{\partial y} \delta y + \frac{\partial g}{\partial y'} \delta y'
\end{equation} \refstepcounter{subsection}
Observamos que nos queda una expresión similar a la regla de la cadena del diferencial exacto de una función.
%-------------------------------------------------------------------------------------------
\subsection{Variación de un funcional}
De nuevo, si partimos de un funcional $F[y]$ que depende de un único camino, definimos $\mathbb{F}([y],a) = F[Y(x,a)]$ y su variación total $\Delta \mathbb{F} = \mathbb{F}([y],a)-\mathbb{F}([y],0) \label{1.1.13} \inlineeqnum$, que desarrollando la integral llegamos inmediatamente a
\begin{equation} \label{1.1.14}
    \Delta \mathbb{F} = \int_{x_A}^{x_B}{\Delta\mathbb{g}dx}=\int_{x_A}^{x_B}{\delta \mathbb{g}dx} + O(a^2)=\underbrace{\int_{x_A}^{x_B}\delta g dx}_{\delta \mathbb{F}=\delta F} + O(a^2)
\end{equation}
\section{Extremizar un funcional} \refstepcounter{subsection}
%-------------------------------------------------------------------------------------------
Diremos que el extremo de $F$ ocurrirá cuando $\delta F = 0$, puesto que a primer orden el funcional no cambiará de valor al variar $y$.
\newpage
De (1.1.13) sustuimos en (1.1.14), sacamos factor común el parámetro $a$ e integramos por partes el segundo término, tal que $ u = \partial_{y'}g$ y $ dv = \eta' dx$
\begin{equation} \label{1.2.1}
    \int_{x_A}^{x_B}{\left[\frac{\partial g}{\partial y} \eta + \frac{\partial g}{\partial y'} \eta'\right] adx} = a \left[\int_{x_A}^{x_B}{\frac{\partial g}{\partial y} \eta dx} + \left|\frac{\partial g}{\partial y'} \eta\right|_{x_A}^{x_B} -\int_{x_A}^{x_B}{\frac{d}{dx}\left(\frac{\partial g}{\partial y'}\right) \eta dx}\right]
\end{equation} \refstepcounter{subsection} 
Por (1.1.4) el segundo término es 0, juntando las integrales y usando (1.1.2)
\begin{equation} \label{1.2.2}
    \int_{x_A}^{x_B}{\left[\frac{\partial g}{\partial y} -\frac{d}{dx}\left(\frac{\partial g}{\partial y'}\right) \right] \delta y dx}=0
\end{equation} \refstepcounter{subsection} 
Ahora, $\delta y$ es completamente arbitrario, pues depende de un parámetro independiente $a$ y de una función $\eta$ que es también arbitraria, esto es lema fundamental del Cálculo Variacional, y garantiza que si la integral debe valer 0, el primer factor debe valer siempre 0, y concluimos

\vspace{-20pt}
\Large\begin{equation} \label{1.2.3}
    \boxed{\frac{\partial g}{\partial y} -\frac{d}{dx}\left(\frac{\partial g}{\partial y'}\right) =0} \iff \delta F =0
\end{equation} \refstepcounter{subsection}\normalsize
Esta es la ecuación de \textit{Euler-Lagrange}, una ecuación diferencial en derivadas parciales de segundo orden cuya solución $y$ extremiza el funcional definido por $g$.
%-------------------------------------------------------------------------------------------
\subsubsection{Geodésica del plano}
Un ejemplo para aplicar (1.2.3) es minimizar la distancia $d=\int{ds}$ en el plano ecuclídeo. Si $y=y(x)$, entonces $ds=\sqrt{dx^2+dy^2}=\sqrt{1+y'^2}dx=g dx$, tal que
\[\frac{\partial g}{\partial y}=0 \implies \frac{d}{dx}\left(\frac{\partial g}{\partial y'}\right)=0 \implies \frac{\partial g}{\partial y'} = \frac{y'}{\sqrt{1+y'^2}}= K \rightarrow y' = \frac{K}{\sqrt{1-K^2}}=\alpha\]
Lo cual implica que $y=\alpha x + y_0$, la ecuación de una recta.
%-------------------------------------------------------------------------------------------

\subsection{Identidad de Beltrami}
Podemos reescribir (1.2.3) de otra forma que nos va resultar últil para resolver algunos problemas y va a resultar muy importante en episodios posteriores.
\[\frac{dg}{dx} = \frac{\partial g}{\partial y} y' + \frac{\partial g}{\partial y'}y'' + \frac{\partial g}{\partial x}\rightarrow \frac{\partial g}{\partial y} y' = \frac{dg}{dx} - \frac{\partial g}{\partial y'}y'' - \frac{\partial g}{\partial x}\]
Podemos observar que el término en el primer miembro de la segunda expresión aparece en (1.2.3) sin multiplicar por $y'$.
\[\frac{dg}{dx} - \frac{\partial g}{\partial x} - \left[\frac{\partial g}{\partial y'}y'' + y' \frac{d}{dx}\left(\frac{\partial g}{\partial y'}\right)\right]=0 \rightarrow \frac{dg}{dx} - \frac{\partial g}{\partial x} - \frac{d}{dx}\left(\frac{\partial g}{\partial y'}y'\right)=0\]
Observando que lo de dentro del paréntesis de la primera expresión es la derivada de un producto, usamos la linearidad de la derivada para obtener
\begin{equation} \label{1.3.4}
    \frac{d}{dx}\left(g -\frac{\partial g}{\partial y'}y'\right)=\frac{\partial g}{\partial x}
\end{equation}
%-------------------------------------------------------------------------------------------

\section{Generalización a varias variables} \refstepcounter{subsection} 
Denotamos $\{f_\alpha(x)\}$ a un conjunto de $N$ funciones distintas, que verifican una expresión similar a (1.0.2), $f_\alpha(x_A)=f_{\alpha A}; \ f_\alpha(x_B)=f_{\alpha B} \label{1.3.1} \inlineeqnum$.
%\sidenote{Si existe una función que las relacione, se trata de una ligadura, veáse el apartado siguiente}
Definimos entonces el siguiente funcional que depende de $\{f_\alpha\}$
\[F[\{f_\alpha\}]=\int_{x_A}^{x_B}{g(\{f_\alpha,f'_\alpha\};x)dx}\]
Ahora siguiendo un desarrollo idéntico a (1.1.12), desarrollando la regla de la cadena para cada una de las variables de $g$ resulta en un sumatorio y los argumentos siguientes para llegar a $\delta g$ son idénticos puesto que son lineales, de tal forma llegamos a la siguiente expresión
\begin{equation} \label{1.3.2}
    \delta g = \sum{\frac{\partial g}{\partial f_\alpha} \delta f_\alpha + \frac{\partial g}{\partial f'_\alpha} \delta f'_\alpha}
\end{equation} \refstepcounter{subsection} 
La expresión (1.1.15) no dependía de las variables de $g$, por lo que es directamente aplicable, sustituyendo (1.3.2) y haciendo la regla de la cadena igual que en (1.2.1) llegamos a una expresión similar a (1.2.2), usando que la integral conmuta con el sumatorio
\begin{equation} \label{1.3.3}
    \delta F = \sum \int_{x_A}^{x_B}{\left[\frac{\partial g}{\partial f_\alpha} -\frac{d}{dx}\left(\frac{\partial g}{\partial f'_\alpha}\right) \right] \delta f_\alpha dx}=0
\end{equation} \refstepcounter{subsection} 
Para poder concluir que cada sumando es 0, y que entonces por ser $\delta f_\alpha$ arbitraria cada término en corchetes es 0, es necesario que los $\delta f_\alpha$ sean independientes entre sí, que es equivalente a que no exista una dependencia explícita entre los $f_\alpha(x)$, que podria estar por ejemplo expresada por una ecuación relacionando varias de ellas. Si se cumple que son independientes, entonces

\vspace{-20pt}
\Large\begin{equation} \label{1.3.3}
    \boxed{\frac{\partial g}{\partial f_\alpha} -\frac{d}{dx}\left(\frac{\partial g}{\partial f'_\alpha}\right) =0} \iff \delta F =0
\end{equation} \refstepcounter{subsection}\normalsize
Ahora tenemos un sistema de ecuaciones de \textit{Euler-Lagrange} cuyas soluciones $f_\alpha(x)$ extremizan el funcional.
%-------------------------------------------------------------------------------------------
\subsection{Ligaduras}
En el caso de que existan $m$ ecuaciones de ligadura de la forma $G_i(\{f_\alpha\};t)=0$, tenemos dos opciones, la primera es resolver el sistema de ecuaciones que forman expresando $m$ funciones como dependientes de las otras $N-m$ funciones restantes, y aplicar (1.3.4) a las $N-m$ funciones independientes.

En el caso de que esto no sea posible resolver el sistema, debemos recurrir a multiplicadores de \textit{Lagrange}.
%-------------------------------------------------------------------------------------------
\subsubsection{Multiplicadores de Lagrange}
Partimos de que tenemos $m$ ecuaciones $G_i(\{f_\alpha\};t)=0$ que no sabemos resolver, $\Delta G_i =0$, es decir, $G_i$ se aplica de la misma forma tanto a los caminos de referencia como a los variados, además $\Delta G_i = \delta G_i + O(a^2) = 0$, como $a$ es arbitrario, entonces $\delta G_i =0$. Aplicando la regla de la cadena de (1.1.13)
\begin{equation} \label{1.3.5}
    \delta G_i(\{f_\alpha\};x) = \sum_\alpha^m{\frac{\partial G_i}{\partial f_\alpha} \delta f_\alpha}=\sum_\alpha^m{a_{i\alpha} \delta f_\alpha}=0; \ \ \ a_{i\alpha} = \frac{\partial G_i}{\partial f_\alpha}
\end{equation} \refstepcounter{subsection}
Así tenemos la ecuación que nos relaciona las distintas $\delta f_\alpha$, el término de la derivada lo podemos expresar como las componentes de un matriz. Podemos separar la expresión anterior tal que
\begin{equation} \label{1.3.6}
    \delta G_i= \sum_{\gamma=1}^{N-m}{a_{i\gamma} \delta f_\gamma} + \sum_{\beta=N-m+1}^{N}{a_{i\beta} \delta f_\beta} = 0
\end{equation} \refstepcounter{subsection}
La matriz del segundo término es cuadrada $(m\times m)$, y es una matriz jacobiana cuyo determinante va a ser no nulo si las ecuaciones de ligadura son independientes entre sí, de lo contrario algunas sobran. Esto implica que esa matriz tiene inversa, expresando (1.3.6) como operaciones matriciales ($N-m < \beta \leq N$)
\begin{equation} \label{1.3.7}
    0= A\mathbf{x} + J\mathbf{y} \implies \mathbf{y}=-J^{-1}A\mathbf{x}; \ \ \delta f_\beta = -\sum_{a=1}^m{\sum_{\gamma =1}^{N-m}{J^{-1}_{\beta a}a_{a\gamma}\delta f_\gamma}}
\end{equation} \refstepcounter{subsection}
De esta forma, hemos encontrado la dependencia explícita de $\delta f_\beta$ en función de los $\delta f_\gamma$, estos últimos siendo independientes entre sí. Ahora tomamos (1.3.3) y renombramos el factor en corchetes por $\Gamma_\alpha$ y separamos como en (1.3.6)
\begin{equation} \label{1.3.8}
    0 = \delta F = \int_{x_A}^{x_B}{\sum_\alpha^N{(\Gamma_\alpha \delta f_\alpha)}dx=\int_{x_A}^{x_B}{\sum_{\gamma =1}^{N-m}{(\Gamma_\gamma \delta f_\gamma)}dx+\int_{x_A}^{x_B}{\sum_{\beta=N-m+1}^{N}{(\Gamma_\beta \delta f_\beta)}dx}}}
\end{equation} \refstepcounter{subsection}
Sustituyendo $\delta f_\beta$ de (1.3.7)
\begin{equation} \label{1.3.9}
    0 =\int_{x_A}^{x_B}{\sum_{\gamma =1}^{N-m}{(\Gamma_\gamma \delta f_\gamma)}dx-\int_{x_A}^{x_B}{\sum_{\beta=N-m+1}^{N}{\left(\Gamma_\beta \sum_{a=1}^m{\sum_{\gamma =1}^{N-m}{J^{-1}_{\beta a}a_{a\gamma}\delta f_\gamma}}\right)}dx}}
\end{equation} \refstepcounter{subsection}
Como los sumatorios conmutan podemos llegar a
\begin{equation} \label{1.3.10}
    0  =\int_{x_A}^{x_B}{\sum_{\gamma =1}^{N-m}{(\Gamma_\gamma \delta f_\gamma)}dx-\int_{x_A}^{x_B}\sum_{\gamma =1}^{N-m}{{\sum_{a=1}^m{\sum_{\beta=N-m+1}^{N}{\Gamma_\beta J^{-1}_{\beta a}a_{a\gamma}\delta f_\gamma}}}dx}}
\end{equation} \refstepcounter{subsection} 
Y ahora podemos unificar los sumatorios de $\gamma$ y sacar factor común $\delta f_\gamma$
\begin{equation} \label{1.3.11}
    0  =\int_{x_A}^{x_B}{\sum_{\gamma =1}^{N-m}{\delta f_\gamma \left(\Gamma_\gamma -{\sum_{a=1}^m{\sum_{\beta=N-m+1}^{N}{\Gamma_\beta J^{-1}_{\beta a}a_{a\gamma}}}}\right)} dx}
\end{equation} \refstepcounter{subsection} 
Definimos entonces $\lambda_a=\sum_{\beta=N-m+1}^{N}{\Gamma_\beta J^{-1}_{\beta a}}$ como los multiplicadores de \textit{Lagrange} y reemplazando $a_{a\gamma}$ por su definición de (1.3.5)
\begin{equation} \label{1.3.12}
    0  =\int_{x_A}^{x_B}{\sum_{\gamma =1}^{N-m}{\delta f_\gamma \left(\Gamma_\gamma -{\sum_{a=1}^m{\lambda_a \frac{\partial G_a}{\partial f_\gamma}}}\right)} dx}
\end{equation} \refstepcounter{subsection}
Ahora como $\delta f_\gamma$ son independientes entre sí, podemos aplicar el mismo argumento que en los otros casos y concluir que lo del paréntesis debe ser igual a 0 para todos los $\gamma$, tal que ($1 \leq \gamma \leq N-m$)
\begin{equation} \label{1.3.13}
    \Gamma_\gamma -{\sum_{a=1}^m{\lambda_a \frac{\partial G_a}{\partial f_\gamma}}}=0
\end{equation} \refstepcounter{subsection}
Podemos ahora comprobar que si $N-m < \gamma \leq N$
\[\Gamma_\gamma -{\sum_{a=1}^m{\lambda_a \frac{\partial G_a}{\partial f_\gamma}}}\right)=\Gamma_\gamma -{\sum_{a=1}^m{\lambda_a J_{a\gamma}}}\right) = \Gamma_\gamma -{\sum_{a=1}^m{\sum_{\beta=N-m+1}^{N}{\Gamma_\beta J^{-1}_{\beta a}} J_{a\gamma}}}=\]
\begin{equation} \label{1.3.14}
    =\Gamma_\gamma -\sum_{\beta=N-m+1}^{N}{\Gamma_\beta \delta_{\beta \gamma}} \right) = \Gamma_\gamma - \Gamma_\gamma = 0
\end{equation} \refstepcounter{subsection}
También se verifica (1.3.12), por lo que entonces
\Large\begin{equation} \label{1.3.15}
    \boxed{\frac{\partial g}{\partial f_\alpha} -\frac{d}{dx}\left(\frac{\partial g}{\partial f'_\alpha}\right) ={\sum_{i=1}^m{\lambda_i \frac{\partial G_i}{\partial f_\alpha}}} \ \ \ \ \ G_i(\{f_\alpha\}) = 0}
\end{equation} \refstepcounter{subsection}\normalsize
Tenemos por lo tanto un sistema de $N+m$ ecuaciones, que incluye las ecuaciones de \textit{Euler-Lagrange} modificadas y las ecuaciones de ligadura, y las incognitas son las $f_\alpha$ y las $\lambda_i$.

\chapter{Mecánica Lagrangiana}
\labch{intro}
%-------------------------------------------------------------------------------------------
Ahora la variable independiente sobre la que vamos a trabajar va a ser el tiempo, $t$, y las variables dependientes son las coordenadas cartesianas $\{x_{\alpha i}\}$, donde $\alpha$ indica la partícula y $i$ indica la componente de la posición.
Definimos además las derivadas totales temporales como $\{\dot{x}_{\alpha i}\}$.
\refstepcounter{section}
\section{Principio de Hamilton}\refstepcounter{subsection}

Definimos una función llamada \textbf{Lagrangiano}\sidenote{La definición de \textbf{Lagrangiano} dependerá de la configuración del sistema físico, pero como norma géneral en mecánica clásica \eqref{2.1.1} es la expresión más común de la función que verifica \eqref{2.1.3}. Esto se demuestra más adelante.}
\begin{equation} \label{2.1.1}
    \pazocal{L}(\{x_{\alpha i},\dot{x}_{\alpha i}\};t)=T-U
\end{equation} \refstepcounter{subsection}
Dónde $T$ es la energía cinética del sistema y $U$ es la energía potencial (conservativa o no), de tal forma que definimos el siguiente funcional llamado \textbf{acción}
\begin{equation} \label{2.1.2}
    S \equiv \int_{t_A}^{t_B}{\pazocal{L}(\{x_{\alpha i},\dot{x}_{\alpha i}\};t)dt}
\end{equation} \refstepcounter{subsection}
\textbf{Principio de Hamilton o de mínima acción.} La evolución temporal de un sistema físico es aquella que extremiza la acción, es decir que $\delta S = 0$ para la evolución real del sistema, lo cual es equivalente a
\begin{equation} \label{2.1.3}
    \frac{\partial \pazocal{L}}{\partial x_{\alpha i}} -\frac{d}{dt}\left(\frac{\partial \pazocal{L}}{\partial \dot{x}_{\alpha i}}\right) =0
\end{equation} \refstepcounter{subsection}
Para la mecánica clásica, este principio es equivalente a las leyes de \textit{Newton}, cuando $\pazocal{L}$ toma la forma de \eqref{2.1.1} con ligeras modificaciones que discutiremos en las próximas secciones.
%-------------------------------------------------------------------------------------------
\subsubsection{Muelle elástico}
Un sencillo ejemplo para aplicar este principio es el de un muelle elástico en una dirección, donde $T=m\dot{x}^2/2$ y $U = kx^2/2$ (el término $mgh$ es constante y puede ser ignorado), si $\pazocal{L} = T-U$, entonces
\[\frac{\partial \pazocal{L}}{\partial x}=-kx \ \ \ \ \frac{\partial \pazocal{L}}{\partial \dot{x}}=m\dot{x} = p \ \ \ \ \frac{d p}{dt} = m\ddot{x} \rightarrow m\ddot{x} = -kx \iff F=-kx=ma\]
%-------------------------------------------------------------------------------------------
\refstepcounter{section}
\section{Coordendas generalizadas} \refstepcounter{subsection}
Podemos realizar un cambio de variables para poder expresar $x_{\alpha i}$ en función de otras variables $q_j$, las cuales pueden resultarnos más sencillas para resolver un problema, tal que $x_{\alpha i}=x_{\alpha i}(\{q_j\};t)$. Esta transformación será invertible cuando
\[J_l^k=\frac{\partial x_k}{\partial q_l}\]
el determinante de esa matriz, el jacobiano, sea no nulo, tal que existe la transformación $q_j = q_j(\{x_{\alpha i}\};t)$.

Usando la regla de la cadena podemos ver la dependencia de las velociades entre sí, $\dot{x}_{\alpha i}=\dot{x}_{\alpha i}(\{q_j,\dot{q}_j\};t)$ y que $\dot{q}_j = \dot{q}_j(\{x_{\alpha i},\dot{x}_{\alpha i}\};t)$.

De esta forma, podemos expresar $\pazocal{L}$ en función de las coordenadas y velocidades generalizadas, tal que $\pazocal{L}=\pazocal{L}(\{q_j,\dot{q}_j\};t)$ de tal forma que \eqref{2.1.3} queda como 

\vspace{-10pt}
\Large\begin{equation} \label{2.2.1}
    \boxed{\frac{\partial \pazocal{L}}{\partial q_j} -\frac{d}{dt}\left(\frac{\partial \pazocal{L}}{\partial \dot{q}_j}\right) =0} \tag{2.2.1/E-L}
\end{equation} \refstepcounter{subsection}\normalsize

Definimos además el \textbf{momento generalizado}, que para cartesianas es el momento lineal y para polares es el momento angular. También definimos la \textbf{fuerza generalizada}, que es la proyección del vector cartesiano en el sistema de vectores asociado a las coordenadas generalizadas.
\begin{equation} \label{2.2.2}
    p_j = \frac{\partial \pazocal{L}}{\partial \dot{q}_j} \ \ \ \ Q_j = -\frac{\partial U}{\partial q_j}=\sum_\alpha^N{\mathbf{F}_\alpha\cdot \frac{\partial \mathbf{r}_{\alpha}}{\partial q_j}}
\end{equation} \refstepcounter{subsection}
%-------------------------------------------------------------------------------------------
\vspace{-30pt}
\refstepcounter{section}
\section{Ligaduras}\refstepcounter{subsection}
Al igual que en la sección \textit{Ligaduras} de la sección \hyperref[sec:1.3]{1.3}, tendremos $M$ ecuaciones de ligadura, los tipos de las cuales se datallarán a continuación, pero antes definimos lo que vamos a denominar \textbf{grados de libertad}, que indica el número mínimo de parámetros que es necesario para especifcar la configuración del sistema en un tiempo dado, tal que $s = N\cdot d- M \label{2.3.1} \inlineeqnum$, donde $s$ son los \textbf{grados de libertad}, $N$ el número de partículas del sistema, y $d$ la dimensión del espacio.
%-------------------------------------------------------------------------------------------
\subsubsection{Tipos de ligaduras}
Cuando las ecuaciones de ligadura no dependen de las veclocidades, $G_i(\{q_j\})=0$, se denominan ligaduras \textbf{holónomas} y son con las que vamos a trabajar. Si las ecuaciones de ligadura dependen de la velocidad,$ G_i(\{q_j,\dot{q}_j\})=0$, se denominan \textbf{no holónomas} y salvo que sean integrables no trabjaremos con ellas. Son \textbf{integrables} cuando son de la forma siguiente donde $h=h(\{q_i\};t)$ tal que
\begin{equation} \label{2.3.2}
    \sum_j^{N\cdot d}{A_j(\{q_i\};t)\dot{q}_j} + B(\{q_i\};t)=0; \ \ \ \ A_j=\frac{\partial h}{\partial q_j}; \ \ \ \ B_j=\frac{\partial h}{\partial t}
\end{equation} \refstepcounter{subsection}
Entonces podemos ver que nos queda la regla de la cadena e integramos
\begin{equation} \label{2.3.3}
    \sum_j^{N\cdot d}{\frac{\partial h}{\partial q_j}\dot{q}_j} + \frac{\partial h}{\partial t}=\frac{d h}{dt} =0 \iff h(\{q_i\};t) - C = 0 \mbox{ (Holónoma)}
\end{equation} \refstepcounter{subsection}
Luego a parte si la ligadura depende explícitamente del tiempo se llama \textbf{forzada} o \textbf{reónoma}, si no depende explcítamente del tiempo, se denominan \textbf{naturales} o \textbf{esclerónomas}.
%-------------------------------------------------------------------------------------------

\newpage
\subsection{Sistema holonómico}
Decimos que un sistema es \textbf{holonómico} cuando podemos resolver (o bien en cartesianas o en generalizadas) las ecuaciones de ligadura (holónomas) y expresar $m$ coordenadas como explícitamente dependientes de $s$ coordenadas independientes, reduciendo el sistema a $s$ variables que podemos resolver usando \eqref{2.2.1}.
%-------------------------------------------------------------------------------------------
\subsection{Multiplicadores de Lagrange}
En el caso en el que el sistema no sea holonómico, y no podamos resolver las ecuaciones de ligadura, al igual que en la sección  \hyperref[sec:1.3]{1.3} tenemos que recurrir a multiplicadores de multiplicadores de \textit{Lagrange}, podemos obtener una expresión equivalente a \eqref{1.3.15} modificando el lagrangiano de la siguiente forma
\begin{equation} \label{2.3.4}
    \pazocal{L}^* = \pazocal{L}+\sum^m{\lambda_i G_i}
\end{equation} \refstepcounter{subsection}
Que aplicando las ecuaciones de \textit{Euler-Lagrange} tanto para $q_j$ como para $\lambda_i$ resulta

\vspace{-10pt}
\Large\begin{equation} \label{2.3.5}
    \frac{\partial \pazocal{L}}{\partial q_j} -\frac{d}{dt}\left(\frac{\partial \pazocal{L}}{\partial \dot{q}_j}\right) +\underbrace{\sum^m\lambda_i(t)\frac{\partial G_i}{\partial q_j}}_{Q^L_j}=0
\end{equation} \refstepcounter{subsection}\normalsize
Donde $Q^L_j$ es la componente $j$ de la fuerza generalizada de ligadura total, que cumplen que $dW^L = \mathbf{F}^L\cdot d\mathbf{r}=0$ son fuerzas que o siempre perpendiculares a la ligadura, o que provocan que $d\mathbf{r}=0 \iff \mathbf{v}=0$ en el punto de contacto.

\refstepcounter{section}
\section{Principio de D'Alambert}\refstepcounter{subsection}
\subsubsection{Estático}
Vamos a suponer que tenemos un sistema en equilibrio, es decir $\mathbf{F}_i = 0$ para cada partícula del sistema, o en general para cualquier punto donde se aplica una fuerza, de tal forma que $\mathbf{F}_i \cdot \delta \mathbf{r}_i = 0$.

$\delta \mathbf{r}_i$ es lo que se denomina desplazamiento virtual, el sistema no se mueve, sigue en el equilibrio, estos desplazamientos se definen respetando las ligaduras del sistema, por ejemplo, tenemos en un péndulo que por la gravedad y otra fuerza externa se encuentra en equilibrio, podemos expresar la posición en términos de las coordenadas generalizadas, que respetan las ligaduras, tal que $\mathbf{r} = l (\sin \theta \mathbf{e}_x + \cos \theta \mathbf{e}_y)$, y entonces $\delta \mathbf{r} = l (\cos \theta \mathbf{e}_x-\sin\theta \mathbf{e}_y) \delta \theta$.

Para sistemas tendremos entones $\sum_i \mathbf{F}_i \cdot \delta \mathbf{r}_i = 0$, pero esto no nos dice nada, sin embargo, podemos descomponer la fuerza total que se ejerce en un punto o partícula en fuerzas de ligadura o normal y fuerzas externas o aplicadas, tal que $\mathbf{F}_i = \mathbf{F}_i^{(a)}+\mathbf{f}_i$, aunque no es siempre facil distinguirlo a simple vista. De esta forma nos queda
\begin{equation} \label{2.4.1}
    \sum_i \mathbf{F}_i^{(a)} \cdot \delta \mathbf{r}_i + \sum_i \mathbf{f}_i \cdot \delta \mathbf{r}_i = 0
\end{equation} \refstepcounter{subsection}
Ahora, salvo casos muy extraños de ligaduras, el trabajo virtual de una fuerza normal o de ligadura se anula, como se ha explicado en el apartado anterior, de esta forma llegamos al principio de D'Alambert estático
\begin{equation} \label{2.4.2}
    \sum_i \mathbf{F}_i^{(a)} \cdot \delta \mathbf{r}_i = 0
\end{equation} \refstepcounter{subsection}
También es llamado principio del trabajo virtual, y al aplicarlo, las componentes $\delta \mathbf{r}_i$ son no nulas, en general implica que $\delta q_i \neq 0$, donde $q_i$ son las coordenadas generalizadas, por ejemplo, $\delta \theta$ en el ejemplo anterior.
\subsubsection{Dinámico}
Ahora consideraremos un sistema que no esta en equilibrio, tenemos que $\mathbf{F}_i = \dot{\mathbf{p}}_i = m_i \ddot{\mathbf{r}}_i$, de tal forma que haciendo exactamente las mismas manipulaciones que antes, y descomponiendo la fuerza total, llegamos a la versión dinámica del principio de D'Alambert
\begin{equation} \label{2.4.3}
    \sum_i (\dot{\mathbf{p}}_i-\mathbf{F}_i^{(a)}) \cdot \delta \mathbf{r}_i = 0
\end{equation} \refstepcounter{subsection}
\vspace{-25pt}
\subsubsection{Lagrangiano}
Si hacemos el cambio a coordenadas generalizadas usando la regla de la cadena, tenemos
\begin{equation} \label{2.4.4}
    \delta \mathbf{r}_i = \sum_j \frac{\partial \mathbf{r}_i}{\partial q_j} \delta q_j
\end{equation} \refstepcounter{subsection}
Podemos ahora definir de nuevo la fuerza generalizada
\begin{equation} \label{2.4.5}
    \sum_i \mathbf{F}_i \cdot \delta \mathbf{r}_i = \sum_{ij} \mathbf{F}_i \cdot \frac{\partial \mathbf{r}_i}{\partial q_j} \delta q_j = \sum_j Q_j \delta q_j \implies Q_j = \sum_i \mathbf{F}_i \cdot \frac{\partial \mathbf{r}_i}{\partial q_j}
\end{equation} \refstepcounter{subsection}
Es importante notar que $Q_j$ no tiene unidades de fuerza en general, pero $Q_j \delta q_j$ siempre tiene unidades de trabajo. Si consideramos solo fuerzas conservativas (aunque pueden depender del tiempo) tenemos
\begin{equation} \label{2.4.6}
    \mathbf{F}_i = -\nabla_i U \ \ \ \ \ \ Q_j = -\sum_i \nabla_i U \cdot \frac{\partial \mathbf{r}_i}{\partial q_j} = - \sum_{ik} \frac{\partial U}{\partial r_{ik}} \frac{\partial r_{ik}}{\partial q_j} = - \frac{\partial U}{\partial q_j}
\end{equation} \refstepcounter{subsection}
Ahora, hacemos lo mismo pero con el otro término de \eqref{2.4.3}, y descomponemos una regla del producto
\begin{equation} \label{2.4.7}
    \sum_i \dot{\mathbf{p}}_i \cdot \delta \mathbf{r}_i = \sum_{ij} m \ddot{\mathbf{r}}_i \cdot \frac{\partial \mathbf{r}_i}{\partial q_j} \delta q_j \ \ \ \ \ \ \ \sum_{i} m \ddot{\mathbf{r}}_i \cdot \frac{\partial \mathbf{r}_i}{\partial q_j} = \sum_i \left[ \frac{d}{dt}\left(m_i \dot{\mathbf{r}}_i \cdot \frac{\partial \mathbf{r}_i}{\partial q_j}\right)- m_i \dot{\mathbf{r}}_i\frac{d}{dt}\left(\frac{\partial \mathbf{r}_i}{\partial q_j}\right)\right]
\end{equation} \refstepcounter{subsection}
\begin{equation} \label{2.4.8}
    \frac{d}{dt}\left(\frac{\partial \mathbf{r}_i}{\partial q_j}\right) = \sum_k \frac{\partial^2 \mathbf{r}_i}{\partial q_j \partial q_k}\dot{q}_k + \frac{\partial^2 \mathbf{r}_i}{\partial q_j \partial t} = \frac{\partial}{\partial q_j}\left( \sum_k \frac{\partial \mathbf{r}_i}{ \partial q_k}\dot{q}_k + \frac{\partial\mathbf{r}_i}{\partial t} \right)= \frac{\partial \dot{\mathbf{r}}_i}{\partial q_j} = \frac{\partial \mathbf{v}_i}{\partial q_j}
\end{equation} \refstepcounter{subsection}
Además podemos ver que
\begin{equation} \label{2.4.9}
    \frac{\partial \mathbf{v}_i}{\partial \dot{q}_j} = \frac{\partial}{\partial \dot{q}_j}\left( \sum_k \frac{\partial \mathbf{r}_i}{ \partial q_k}\dot{q}_k + \frac{\partial\mathbf{r}_i}{\partial t} \right) = \sum_k \frac{\partial \mathbf{r}_i}{ \partial q_k} \delta_{kj} =  \frac{\partial \mathbf{r}_i}{ \partial q_j}
\end{equation} \refstepcounter{subsection}
Ahora, introducimos \eqref{2.4.8} en \eqref{2.4.7} tal que
\[\sum_i \dot{\mathbf{p}}_i \cdot \delta \mathbf{r}_i = \sum_{ij} \left[ \frac{d}{dt}\left(m_i \mathbf{v}_i \cdot \frac{\partial \mathbf{v}_i}{\partial \dot{q}_j}\right)- m_i \mathbf{v}_i\frac{\partial \mathbf{v}_i}{\partial q_j}\right] \delta q_j = \]
\begin{equation} \label{2.4.10}
     = \sum_{ij} \left[ \frac{d}{dt}\left(\frac{\partial}{\partial \dot{q}_j} \left(\frac{1}{2}m_i \mathbf{v}_i \cdot \mathbf{v}_i\right)\right)- \frac{\partial}{\partial q_j} \left(\frac{1}{2}m_i \mathbf{v}_i \cdot \mathbf{v}_i\right)\right] \delta q_j = \sum_{j} \left[ \frac{d}{dt}\left(\frac{\partial T}{\partial \dot{q}_j}\right)- \frac{\partial T}{\partial q_j}\right] \delta q_j
\end{equation} \refstepcounter{subsection}
Ahora metiendo \eqref{2.4.10} y \eqref{2.4.6} en \eqref{2.4.3}
\begin{equation} \label{2.4.11}
    \sum_{j} \left[ \frac{d}{dt}\left(\frac{\partial T}{\partial \dot{q}_j}\right)- \frac{\partial T}{\partial q_j} +\frac{\partial U}{\partial q_j}\right] \delta q_j = 0
\end{equation} \refstepcounter{subsection}
Ahora, si suponemos que tenemos un sistema holonómico y que podemos resolver las ligaduras y que $q_j$ son holónomas, y por lo tanto independientes, y a du vez $\delta q_j$ también, tenemos el siguiente conjunto de ecuaciones
\begin{equation} \label{2.4.12}
    \frac{d}{dt}\left(\frac{\partial (T-U)}{\partial \dot{q}_j}\right)-\frac{\partial (T-U)}{\partial q_j} = 0 \ \ \ \ \ \ T-U= \pazocal{L}
\end{equation} \refstepcounter{subsection}
Como estamos considerando fuerzas conservativas, sus potenciales no dependen de las velocidades generalizadas, y por lo tanto la podemos introducir en el primer término de la ecuación sin pérdida de generalidad, y voalá, hemos demostrado \eqref{2.1.3}, el principio de Hamilton para la mecánica de Newton bajo unos ciertos supuestos.

Podemos considerar fuerzas no conservativas, como por ejemplo, un rozamiento dinámico, separandolas, tal que
\begin{equation} \label{2.4.13}
    Q_j = - \frac{\partial U}{\partial q_j} + \sum \mathbf{F}^{(nc)} \cdot \frac{\partial \mathbf{r}_i}{\partial q_j} = - \frac{\partial U}{\partial q_j} + Q_j^{(nc)} \implies \frac{d}{dt}\left(\frac{\partial \pazocal{L}}{\partial \dot{q}_j}\right)-\frac{\partial \pazocal{L}}{\partial q_j} =  Q_j^{(nc)}
\end{equation} \refstepcounter{subsection}
\vspace{-20pt}
\subsubsection{Potencial generalizado}
Además, podemos crear un tipo de potencial generalizado que dependa de $\dot{q}_j$, para ello, primero tomamos la forma genérica de \eqref{2.4.11} sin asumir nada sobre la fuerza
\begin{equation} \label{2.4.14}
    \frac{d}{dt}\left(\frac{\partial T}{\partial \dot{q}_j}\right)-\frac{\partial T}{\partial q_j} = Q_j
\end{equation} \refstepcounter{subsection}
Y por otro lado descomponemos la parte cinética y potencial de \eqref{2.4.12} e igualamos
\begin{equation} \label{2.4.15}
    \frac{d}{dt}\left(\frac{\partial U}{\partial \dot{q}_j}\right)-\frac{\partial U}{\partial q_j} = Q_j = \sum_i \mathbf{F}_i \cdot \frac{\partial \mathbf{r}_i}{\partial q_j}
\end{equation} \refstepcounter{subsection}
Entonces cualquier fuerza que pueda ser derivada de un potencial $U(\{q_i\},\{\dot{q}_i\})$ (no puede depender del tiempo en este caso) siguiendo \eqref{2.4.15} verifica también el principio de Hamilton.

El ejemplo más notable de este tipo de potencial es el que crea la fuerza electromagnética, vamos a estudiarlo, lo primero es saber que es $Q_j$ en este caso, usaremos coordenadas cartesianas
\begin{equation} \label{2.4.16}
    Q_j = \sum_i \mathbf{F}_i \cdot \frac{\partial \mathbf{r}_i}{\partial x_j} = \sum_i \mathbf{F}_i \cdot \mathbf{e}_i \delta_{ij} = F_j
\end{equation} \refstepcounter{subsection}
La fuerza de Lorentz es, y podemos definir los campos $\mathbf{E}$ y $\mathbf{B}$ como 
\begin{equation} \label{2.4.17}
    \mathbf{F}(\mathbf{r},\dot{\mathbf{r}}) = q(\mathbf{E}(\mathbf{r})+\dot{\mathbf{r}} \times \mathbf{B}(\mathbf{r})) \ \ \ \ \ \ \ \mathbf{E}(\mathbf{r}) = -\nabla_\mathbf{r} \phi -\frac{\partial \mathbf{A}}{\partial t}  \ \ \ \ \ \ \ \mathbf{B}(\mathbf{r}) = \nabla_\mathbf{r} \times \mathbf{A}
\end{equation} \refstepcounter{subsection}
\vspace{-10pt}
\[
    B_x = \partial_y A_z - \partial_z A_y \ \ \ \ \ B_y = \partial_z A_x-\partial_x A_z \ \ \ \ \ B_z = \partial_x A_y - \partial_y A_x
\]
\[
    (\dot{\mathbf{r}} \times \mathbf{B})_x= \dot{y}B_z - \dot{z}B_y \ \ \ \ \ (\dot{\mathbf{r}} \times \mathbf{B})_y = \dot{z}B_x - \dot{x}B_z \ \ \ \ \ (\dot{\mathbf{r}} \times \mathbf{B})_z = \dot{x}B_y - \dot{y}B_x
\]
Es sencillo comprobar que el siguiente potencial verifica \eqref{2.4.15}
\begin{equation} \label{2.4.18}
    U = q(\phi -\dot{\mathbf{r}}\cdot \mathbf{A})
\end{equation} \refstepcounter{subsection}
Hay muchas manipulaciones que se pueden hacer para ver esto más sencillo, por ejemplo, usando relaciones vectoriales.
%-------------------------------------------------------------------------------------------
\refstepcounter{section}
\section{Teorema de la Energía Cinética}\refstepcounter{subsection}
%-------------------------------------------------------------------------------------------
\subsection{Función k-homogénea}
Una función k-homogéna cumple la siguiente expresión, donde $\lambda$ es un parámetro arbitrario cualquiera
\begin{equation} \label{2.5.1}
    f(\{\lambda x_i\})=\lambda^k f(\{x_i\})
\end{equation} \refstepcounter{subsection}
\vspace{-35pt}
\subsubsection{Teorema de Euler}
Podemos derivar cada lado de \eqref{2.5.1} con respecto al parámetro
\[\frac{\partial}{\partial \lambda} f(\{\lambda x_i\})= \frac{\partial}{\partial \lambda} \lambda^k f(\{x_i\})\]
En el primer miembro hacemos la regla de la cadena y en el segundo es la derivada de una potencia
\[\sum_j^N \frac{\partial f(\{\lambda x_i\})}{\partial (\lambda x_j)} \frac{d (\lambda x_j)}{d \lambda}= \sum_j^N \frac{\partial f(\{\lambda x_i\})}{\partial (\lambda x_j)} x_j=  k \lambda^{k-1} f(\{x_i\})\]
Como $\lambda$ es un parámetro arbitrario, podemos tomar $\lambda=1$ y tenemos
\begin{equation} \label{2.5.2}
    \sum_j^N \frac{\partial f(\{x_i\})}{\partial x_j} x_j=  k f(\{x_i\})
\end{equation} \refstepcounter{subsection}
%------------------------------------------------------------------------------------------
\vspace{-30pt}
\subsection{Forma cuadrática}
Una forma cuadrática es una función 2-homogénea de la siguiente forma
\begin{equation} \label{2.5.3}
    f(\{x_i\})=\sum_{i,j}^N{{a_{jk} x_j x_k}}=\mathbf{x} A \mathbf{x}^T
\end{equation} \refstepcounter{subsection}
Donde $a_{jk}$ no tienen por que ser constantes, pueden ser funciones de otras variables, pero no de $x_i$.
%------------------------------------------------------------------------------------------
\subsection{Teorema de la E.C.}
En coordenadas cartesianas la expresión de la energía cinética es una forma cuadrática que solo depende de las velocidades que tiene la siguiente forma
\vspace{-12pt}
\begin{equation} \label{2.5.4}
    T=T(\{x_{\alpha i}\})=\frac{1}{2} \sum_{\alpha, i}^{N,d} m_\alpha \dot{x}_{\alpha i}^2
\end{equation} \refstepcounter{subsection}

\vspace{-25pt}
Si $x_{\alpha i}(\{q_j\};t)$, entonces
\vspace{-12pt}
\begin{equation} \label{2.5.5}
    \dot{x}_{\alpha i}=\sum_j^s \frac{\partial x_{\alpha i}}{\partial q_j}\dot{q}_j + \frac{\partial x_{\alpha i}}{\partial t}=\dot{x}_{\alpha i}(\{q_j,\dot{q}_j\};t)
\end{equation} \refstepcounter{subsection}

\vspace{-17pt}
Elevando \eqref{2.5.5} al cuadrado tenemos
\[\dot{x}_{\alpha i}^2= \left(\sum_j^s \frac{\partial x_{\alpha i}}{\partial q_j}\dot{q}_j \right) \left(\sum_k^s \frac{\partial x_{\alpha i}}{\partial q_k}\dot{q}_k \right) + 2\frac{\partial x_{\alpha i}}{\partial t} \sum_j^s \frac{\partial x_{\alpha i}}{\partial q_j}\dot{q}_j  + \left(\frac{\partial x_{\alpha i}}{\partial t}\right)^2 = \]
\begin{equation} \label{2.5.6}
    = \sum_{j,k}^s{\frac{\partial x_{\alpha i}}{\partial q_j}\frac{\partial x_{\alpha i}}{\partial q_k}\dot{q}_j\dot{q}_k} + 2\frac{\partial x_{\alpha i}}{\partial t} \sum_j^s \frac{\partial x_{\alpha i}}{\partial q_j}\dot{q}_j + \left(\frac{\partial x_{\alpha i}}{\partial t}\right)^2
\end{equation} \refstepcounter{subsection}
Sustituyendo \eqref{2.5.6} en \eqref{2.5.4}
\[ T = \frac{1}{2} \sum_{\alpha, i}^{N,d} m_\alpha \left[\sum_{j,k}^s{\frac{\partial x_{\alpha i}}{\partial q_j}\frac{\partial x_{\alpha i}}{\partial q_k}\dot{q}_j\dot{q}_k} + 2\frac{\partial x_{\alpha i}}{\partial t} \sum_j^s \frac{\partial x_{\alpha i}}{\partial q_j}\dot{q}_j + \left(\frac{\partial x_{\alpha i}}{\partial t}\right)^2\right]\]
Usando que los sumatorios conmutan y son lineales llegamos a
\begin{equation}  
    T = \sum_{j,k}^s{\left(\sum_{\alpha,i}^{N,d} \frac{1}{2} m_\alpha \frac{\partial x_{\alpha i}}{\partial q_j}\frac{\partial x_{\alpha i}}{\partial q_k}\right)\dot{q}_j\dot{q}_k} + \sum_j^s{\left( \sum_{\alpha,i}^{N,d}m_\alpha \frac{\partial x_{\alpha i}}{\partial t}\frac{\partial x_{\alpha i}}{\partial q_j}\right)}\dot{q}_j + \sum_{\alpha,i}^{N,d}{\frac{1}{2} m_\alpha\left(\frac{\partial x_{\alpha i}}{\partial t}\right)^2} \label{2.5.7}
\end{equation} \refstepcounter{subsection}
De una forma más reducida obtenemos
\begin{equation} \label{2.5.8}
    T = T(\{q_j,\dot{q}_j\};t) = \sum_{j,k}^sA_{ij}\dot{q}_j\dot{q}_k + \sum^s_j B_j \dot{q}_j + C
\end{equation} \refstepcounter{subsection}
Así, fijándonos en \eqref{2.5.7}, si el cambio de coordenadas no depende explícitamente del tiempo, $B_j$ y $C$ se anulan, y entonces $T$ es una forma cuadrática en los $\dot{q}_j$.

\textbf{Teorema de la Energía cinética.} Si las coordenadas no dependen explícitamente del tiempo, entonces $T$ es una forma cuadrática en los $\dot{q}$.

Si ahora partimos de este supuesto y hacemos la parcial de $T$ con respecto a un $\dot{q}_l$ dado, obtenemos
\[\frac{\partial T}{\partial \dot{q}_l} = \cancelto{0}{\frac{\partial}{\partial \dot{q}_l} \sum_{j,k\neq l}^s{A_{jk}^s\dot{q}_j\dot{q}_k}}+\frac{\partial}{\partial \dot{q}_l} \sum_{j=l, k\neq l}^s{A_{lk}\dot{q}_l\dot{q}_k} + \frac{\partial}{\partial \dot{q}_l} \sum_{j\neq l, k = l}^s A_{jl}\dot{q}_j\dot{q}_l + \frac{\partial}{\partial \dot{q}_l} \left(A_{ll} \dot{q}_l^2\right) = \]
\begin{equation} \label{2.5.9}
    =\sum_{j=l, k\neq l}^s{A_{lk}\dot{q}_k} + \sum_{j\neq l, k = l}^s{A_{jl}\dot{q}_j} + 2A_{ll} \dot{q}_l =2 \sum_i^s A_{li} \dot{q}_i=2 \sum_i^s A_{il} \dot{q}_i
\end{equation} \refstepcounter{subsection}
Si ahora hacemos lo siguiente usando \eqref{2.5.9}, vemos que se verifica \eqref{2.5.2}
\begin{equation} \label{2.5.10}
    \sum_j^s \frac{\partial T}{\partial \dot{q}_j} \dot{q}_j = 2 \sum_{j,k}^s{A_{kj}\dot{q}_j\dot{q}_k=2 T}
\end{equation} \refstepcounter{subsection}

%\newpage\null\thispagestyle{empty}\newpage

\chapter{Simetrias y cantidades conservadas}
\labch{intro}

\section{Ejemplos de invariancias} 
\subsection{Invariancia temporal y energía generalizada}\refstepcounter{subsection}
Si tenemos un desplazamiento arbitratio en el tiempo, $t\mapsto t + \delta t$, y se verifica que $\pazocal{L}(\{q_j,\dot{q}_j\};t)=\pazocal{L}(\{q_j,\dot{q}_j\};t+\delta t)$, esto implica que la parcial de $\pazocal{L}$ con respecto a $t$ es 0. Si ahora desarrollamos la derivadada total de de $\pazocal{L}$ con respecto a $t$, tenemos
\vspace{-13pt}
\[\frac{d \pazocal{L}}{dt}=\sum^s\left(\frac{\partial \pazocal{L}}{\partial q_j}\dot{q}_j+\frac{\partial \pazocal{L}}{\partial \dot{q}_j}\ddot{q}_j\right)+\cancelto{0}{\frac{\partial \pazocal{L}}{\partial t}}\]
El primer término del primer sumando dentro del sumario lo podemos expresar en función de (2.2.1)  (\textit{E-L}), tal que
\[\sum^s\left[\frac{d}{dt}\left(\frac{\partial \pazocal{L}}{\partial \dot{q}_j}\right)\dot{q}_j+\frac{\partial \pazocal{L}}{\partial \dot{q}_j}\ddot{q}_j\right]-\frac{d \pazocal{L}}{dt} = -\frac{\partial \pazocal{L}}{\partial t} =0\]
Ahora lo de dentro del paréntesis es la derivada de un producto, y usando la linearidad de la derivada
\begin{equation} \label{3.1.1}
    \frac{d}{dt}\left(\sum^s \frac{\partial \pazocal{L}}{\partial \dot{q}_j}\dot{q}_j-\pazocal{L}\right) = -\frac{\partial \pazocal{L}}{\partial t} =0
\end{equation} \refstepcounter{subsection}
Definimos entonces la función energía generalizada $H$, que se conservará cuando $\pazocal{L}$ no dependa explícitamente del tiempo.
\begin{equation} \label{3.1.2}
    H \equiv \sum^s \frac{\partial \pazocal{L}}{\partial \dot{q}_j}\dot{q}_j-\pazocal{L} \ \ \ \ \ \frac{d H}{dt}=-\frac{\partial \pazocal{L}}{\partial t}
\end{equation} \refstepcounter{subsection} 
Podemos además observar que si se verifican los supuestos del teorema de la energía cinética (el cambio de coordenadas no depende del tiempo) podemos aplicar (2.4.10), y la energía potencial es conservativa, llegamos a $H=E$, es decir, la energía generalizada es igual a la energía clásica.
\vspace{-10pt}
\begin{equation} \label{3.1.3}
    H = \cancelto{2T}{\sum^s\frac{\partial \pazocal{T}}{\partial \dot{q}_j}\dot{q}_j}-\cancelto{0}{\sum^s\frac{\partial \pazocal{U}}{\partial \dot{q}_j}\dot{q}_j}-(T-U)= 2T -T + U = T+U =E 
\end{equation} \refstepcounter{subsection}

\vspace{-25pt}
\subsection{Invariancia espacial}\refstepcounter{subsection}
Si tenemos un desplazamiento arbitrario en una de las coordenadas generalizadas, $q_k\mapsto q_k + \delta q_k$, y se verifica que $\pazocal{L}(q_k,\{q_j,\dot{q}_j\};t)=\pazocal{L}(q_k+\delta q_k,\{q_j,\dot{q}_j\};t)$, esto implica que la parcial de $\pazocal{L}$ con respecto a $q_k$ es 0. Cuando esto ocurre se dice que $q_k$ es una \textbf{variable ignorable}, y de (2.2.1)  (\textit{E-L}) deducimos que su momento generalizado asociado se conserva.
\begin{equation} \label{3.2.1}
    \frac{d}{dt}\left(\frac{\partial \pazocal{L}}{\partial \dot{q}_k}\right) = \dot{p}_k = 0 \implies p_k = C
\end{equation} \refstepcounter{subsection}
\section{Teorema de Noether} \refstepcounter{subsection}
Consideremos unas transformaciones genéricas $h_j$ de las coordendas $q_j$, además de la transformación $\tau$ de t, parametrizadas por un parámetro $\epsilon$ de cualquier orden
\begin{equation} \label{3.3.1}
    \begin{matrix}
        q_j \mapsto q'_j=h_j(\{q_i\},t,\epsilon) \ \ \ \ h_j(\{q_i\},t,0)=q_j \\
        t \mapsto t'=\tau(\{q_i\},t,\epsilon) \ \ \ \ \tau(\{q_i\},t,0)=t \\
    \end{matrix}
\end{equation} \refstepcounter{subsection}
Si estas transformaciones verifican para una cierta función $B(\{q_j\};t)$
\begin{equation} \label{3.3.2}
    \int_{\tau_A}^{\tau_B} \pazocal{L}(h_j,h'_j;\tau)d \tau=\int_{t_A}^{t_B}\left(\pazocal{L}(\(\{q_j,\dot{q}_j\};t) + \epsilon\frac{dB}{dt}\right)dt \ \ \ \ \ h'_j = \frac{dh_j}{d\tau}
\end{equation} \refstepcounter{subsection}
Se dice que esas transformaciones son una simetría del sistema. Vamos a expandir las transformaciones en Taylor en función de $\epsilon$
\begin{equation} \label{3.3.2}
    \begin{split}
        q_j' = h_j(\{q_i\},t,\epsilon) &= q_j + \epsilon\left.\frac{\partial h_j}{\partial \epsilon}\right|_{\epsilon=0} \hspace{-10pt}+ O(\epsilon^2) = q_j + \epsilon\psi_j + O(\epsilon^2) \\ 
        t' = \tau(\{q_i\},t,\epsilon) &= t + \epsilon\left.\frac{\partial \tau}{\partial \epsilon}\right|_{\epsilon=0} \hspace{-10pt}+ O(\epsilon^2) = t + \epsilon\phi + O(\epsilon^2)
    \end{split} 
\end{equation} \refstepcounter{subsection}
Ahora vamos a calcular $h'_j$, para ello usaremos la regla de la cadena, el teorema de la función inversa, y expandiremos en serie
\begin{equation} \label{3.3.2}
    \begin{split}
        h'_j& =\frac{d h_j}{d\tau} = \frac{\partial h_j}{\partial t} \frac{\partial t}{\partial \tau} = \frac{\partial h_j}{\partial t} \frac{\partial \tau}{\partial t}^{-1} = \left(\dot{q}_j + \epsilon \dot{\psi}_j + O(\epsilon^2)\right)  \left(1+\epsilon \dot{\phi} + O(\epsilon^2)\right)^{-1} =  \\ 
        & =\left(\dot{q}_j + \epsilon \dot{\psi}_j\right) \left(1-\epsilon\dot{\phi} + O(\epsilon^2)\right) = \dot{q}_j + \epsilon \left[\dot{\psi}_j-\dot{q}_j \dot{\phi}\right] + O(\epsilon^2)
    \end{split} 
\end{equation} \refstepcounter{subsection}
Expandimos en Taylor $\pazocal{L}(h_j,h'_j;\tau)$ en términos de $\epsilon$ usando (3.2.3) y (3.2.4), $\pazocal{L}$ a secas es el lagrangiano origianal sin variar
\begin{equation} \label{3.3.2}
    \begin{split}
        \tilde{\pazocal{L}}=\pazocal{L}(h_j,h'_j;\tau) = \pazocal{L} + \epsilon \left.\frac{d \tilde{\pazocal{L}}}{d\epsilon}\right|_{\epsilon=0} \hspace{-10pt} +O(\epsilon^2) &=  \pazocal{L} + \epsilon\frac{\partial \pazocal{L}}{\partial t} \left.\frac{\partial \tau}{\partial \epsilon}\right|_{\epsilon=0} \hspace{-10pt} + \epsilon\sum_j \left(\frac{\partial \pazocal{L}}{\partial q_j} \left.\frac{\partial h_j}{\partial \epsilon}\right|_{\epsilon=0} \hspace{-10pt} + \frac{\partial \pazocal{L}}{\partial \dot{q}_j} \left.\frac{\partial h'_j}{\partial \epsilon}\right|_{\epsilon=0} \right) +O(\epsilon^2) = \\ 
        &= \pazocal{L} + \epsilon\frac{\partial \pazocal{L}}{\partial t} \phi + \epsilon\sum_j \left[\frac{\partial \pazocal{L}}{\partial q_j} \psi_j + \frac{\partial \pazocal{L}}{\partial \dot{q}_j} \left(\dot{\psi}_j-\dot{q}_j \dot{\phi}\right)\right] +O(\epsilon^2)
    \end{split}
\end{equation} \refstepcounter{subsection}
Tenemos además el siguiente cambio de variable, y nótese que los límites de la integral vuelven a $t_A$ y $t_B$ al hacerlo
\begin{equation} \label{3.3.2}
    d\tau = \frac{\partial\tau}{\partial t} dt = (1+\epsilon \dot{\phi} + O(\epsilon^2))dt
\end{equation} \refstepcounter{subsection}
Ahora sustituimos en la parte izquierda de (3.2.2)
\begin{equation} \label{3.3.2}
    \begin{split}
        \int_{\tau_A}^{\tau_B} \tilde{\pazocal{L}}d \tau &= \int_{t_A}^{t_B}  dt\left(1+\epsilon \dot{\phi} + O(\epsilon^2)\right) \left(\pazocal{L} + \epsilon\frac{\partial \pazocal{L}}{\partial t} \phi + \epsilon\sum_j \left[\frac{\partial \pazocal{L}}{\partial q_j} \psi_j + \frac{\partial \pazocal{L}}{\partial \dot{q}_j} \left(\dot{\psi}_j-\dot{q}_j \dot{\phi}\right)\right] +O(\epsilon^2)\right)= \\ 
        &= \int_{t_A}^{t_B} dt\left[\pazocal{L} +\epsilon \dot{\phi}\pazocal{L} +\epsilon \frac{\partial \pazocal{L}}{\partial t} +\epsilon\sum_j \left(\frac{\partial \pazocal{L}}{\partial q_j} \psi_j + \frac{\partial \pazocal{L}}{\partial \dot{q}_j} \left(\dot{\psi}_j-\dot{q}_j \dot{\phi}\right)\right)\right]  + O(\epsilon^2)
    \end{split}
\end{equation} \refstepcounter{subsection}
Igualamos al término derecho de (3.2.2), que nos permite cancelar el primer término de la integral, pasamo el termino de $K$, y sacando $\epsilon$ y dividiendo nos queda
\begin{equation} \label{3.3.2}
    \int_{t_A}^{t_B} dt\left[\dot{\phi}\pazocal{L} +\frac{\partial \pazocal{L}}{\partial t}\phi +\sum_j \left(\frac{\partial \pazocal{L}}{\partial q_j} \psi_j + \frac{\partial \pazocal{L}}{\partial \dot{q}_j} \left(\dot{\psi}_j-\dot{q}_j \dot{\phi}\right)\right)- \frac{dB}{dt}\right] = O(\epsilon)
\end{equation} \refstepcounter{subsection}
Ahora, podemos hacer el límite cuando $\epsilon \rightarrow 0$ a ambos miembros, obteniendo que la integral debe anularse, pues esta no depende de $\epsilon$. Vamos a integrar por partes los términos que van multiplicando a $\dot{\phi}$ y $\dot{\psi}_j$
\begin{equation} \label{3.3.2}
    \begin{split}
        \int_{t_A}^{t_B} dt \dot{\phi}\left(\pazocal{L}-\sum_j \dot{q}_j\frac{\partial \pazocal{L}}{\partial \dot{q}_j}\right) & = \left[\phi \left(\pazocal{L}-\sum_j \dot{q}_j\frac{\partial \pazocal{L}}{\partial \dot{q}_j}\right) \right]_{t_A}^{t_B} - \int_{t_A}^{t_B} dt \phi \frac{d}{dt}\left(\pazocal{L}-\sum_j \dot{q}_j\frac{\partial \pazocal{L}}{\partial \dot{q}_j}\right) \\ 
        \int_{t_A}^{t_B} dt \sum_j \frac{\partial \pazocal{L}}{\partial \dot{q}_j} \dot{\psi}_j & = \left[\sum_j \frac{\partial \pazocal{L}}{\partial \dot{q}_j} \psi_j\right]_{t_A}^{t_B} - \int_{t_A}^{t_B} dt \sum_j \frac{d}{dt}\left(\frac{\partial \pazocal{L}}{\partial \dot{q}_j}\right) \psi_j
    \end{split}  
\end{equation} \refstepcounter{subsection}
Sustituyendo de nuevo en (3.2.8) tenemos
\begin{equation} \label{3.3.2}
    \begin{split}
        0 =  \left[ \phi \left(\pazocal{L}-\sum_j \dot{q}_j\frac{\partial \pazocal{L}}{\partial \dot{q}_j}\right) +\sum_j \frac{\partial \pazocal{L}}{\partial \dot{q}_j} \psi_j - B\right]_{t_A}^{t_B} & +  \int_{t_A}^{t_B} dt \sum_j \psi_k \left(\frac{\partial \pazocal{L}}{\partial q_j}-\frac{d}{dt}\left(\frac{\partial \pazocal{L}}{\partial \dot{q}_j}\right)\right) \\ 
        &+\int_{t_A}^{t_B} dt \phi\left[\frac{\partial \pazocal{L}}{\partial t} -\frac{d}{dt}\left(\pazocal{L}-\sum_j \dot{q}_j\frac{\partial \pazocal{L}}{\partial \dot{q}_j}\right)\right]
    \end{split} 
\end{equation} \refstepcounter{subsection}
Vemos entonces que el segundo sumando se anula cuando $q_j$ verifican (2.2.1) (E-L), y el tercero es la expresión (3.1.1), que se anula también cuando $q_j$ verifican (2.2.1) (E-L). En el primer término sustituimos por (3.2.3) y (3.1.2) y, como los límites de integración son arbitrarios, lo contenido en los corchetes debe ser igual $\forall t$, por lo tanto es constante
\begin{equation} \label{3.3.2}
    I = \sum_j \frac{\partial \pazocal{L}}{\partial \dot{q}_j} \psi_j - H\phi -B \ \ \ \ \ \ \psi_j = \left.\frac{\partial h_j}{\partial \epsilon}\right|_{\epsilon=0} \ \ \ \ \ \ \phi = \left.\frac{\partial \tau}{\partial \epsilon}\right|_{\epsilon=0}  \ \ \ \ \ \ \ \frac{dI}{dt} = 0 
\end{equation} \refstepcounter{subsection}
\vspace{-25pt}
\subsection{Resumen}
Sean las siguientes transformaciones (nótese que solo necesitamos conocer $\phi$ y $\psi_j$ para aplicar el teorema)
\vspace{-15pt}
\[
    \begin{split}
        q_j \mapsto \tilde{q}_j(\{q_i\},t,\epsilon) \ \ \ \ \tilde{q}_j(\{q_i\},t,0)=q_j  \ \ \ \ \ \  \psi_j & = \left.\frac{\partial \tilde{q}_j}{\partial \epsilon}\right|_{\epsilon=0} \ \ \ \ \ s_j = q_j + \psi_j \epsilon\\ 
        t \mapsto \tilde{t}(\{q_i\},t,\epsilon) \ \ \ \ \tilde{t}(\{q_i\},t,0)=t \ \ \ \ \ \ \ \ \ \  \phi & = \left.\frac{\partial \tilde{t}}{\partial \epsilon}\right|_{\epsilon=0} \ \ \ \ \ \ \  \tau = t + \phi \epsilon
    \end{split}     
\]

\vspace{-20pt}
Si verifican para una cierta función $B(\{q_j\};t)$
\vspace{-5pt}
\[
    \int_{\tau_A}^{\tau_B} \pazocal{L}(s_j,s'_j;\tau)d \tau=\int_{t_A}^{t_B}\left(\pazocal{L}(\(\{q_j,\dot{q}_j\};t) + \epsilon\frac{dB}{dt}\right)dt + O(\epsilon^2)\ \ \ \ \ s'_j = \frac{d s_j}{d \tau}
    \]
Entonces se verifica la siguiente expresión para los $q_j$ solución de (2.2.1) (E-L)
\[
    I = \sum_j \frac{\partial \pazocal{L}}{\partial \dot{q}_j} \psi_j - H\phi -B \ \ \ \ \ \ \ \ \ \frac{dI}{dt} = 0 
    \]
\subsection{Ejemplo}
Tenemos la acción
\[S = \int_{t_A}^{t_B} -mc^2 \sqrt{1-\frac{\dot{x}^2}{c^2}}dt = \int_{t_A}^{t_B} \frac{-mc^2}{\gamma_0} dt\]
tenemos las transformaciones $\tilde{x} = \gamma(x-\beta c t)$ y $\tilde{t} = \gamma(t-\beta x/c})$, con $\gamma = \sqrt{1-\beta^2}^{-1}$, podemos verificar que dejan invariante la acción, para ello primero calculamos
\[\tilde{\dot{x}}=\frac{d\tilde{x}}{d\tilde{t}} = \frac{\gamma (dx - \beta c dt)}{\gamma(dt-\beta dx/c)} = \frac{\dot{x}-\beta c}{1-\beta \dot{x}/c} = c \frac{\beta_0-\beta}{1-\beta \beta_0} \ \ \ \ \ \beta_0 = \frac{\dot{x}}{c}\]
\[\tilde{S} = \int_{t_A}^{t_B} -mc^2 \sqrt{1-\frac{\tilde{\dot{x}}^2}{c^2}}d\tilde{t} = \int_{t_A}^{t_B} -mc^2 \gamma (1-\beta \beta_0)\sqrt{1-\left(\frac{\beta_0-\beta}{1-\beta \beta_0}\right)^2}  dt = \]
\[= \int_{t_A}^{t_B} -mc^2 \gamma \sqrt{1 + \beta^2 \beta_0^2 -\beta^2-\beta_0^2}} dt= \int_{t_A}^{t_B} -mc^2 \gamma \sqrt{1-\beta^2}\sqrt{1-\beta_0^2}}dt = \int_{t_A}^{t_B} -mc^2 \sqrt{1-\beta_0^2}}dt = S\]
Por lo tanto, podemos aplicar el Teorema de Noether, calculamos 
\[\psi = \left.\frac{\partial \tilde{x}}{\partial \beta}\right|_{\beta = 0} = \left[\frac{\beta}{(1-\beta^2)^{-3/2}}(x-\beta c t) - \frac{ct}{\sqrt{1-\beta^2}} \right]_{\beta = 0} = -ct\]
\[\phi = \left.\frac{\partial \tilde{t}}{\partial \beta}\right|_{\beta = 0} = \left[\frac{\beta}{(1-\beta^2)^{-3/2}}(t-\beta x/c) - \frac{x/c}{\sqrt{1-\beta^2}} \right]_{\beta = 0} = -\frac{x}{c}\]
\[H = \frac{\partial \pazocal{L}}{\partial \dot{x}} \dot{x} - \pazocal{L} = \gamma_0 m \dot{x}^2+\frac{mc^2}{\gamma_0} = \gamma_0 m \left(\dot{x}^2+c^2\left(1-\frac{\dot{x}^2}{c^2}\right)\right) = +\gamma_0 m c^2 \ \ \ \ \ \ \frac{\partial \pazocal{L}}{\partial \dot{x}} = \gamma_0 m \dot{x}\]
Por lo tanto, la cantidad conservada será
\[I =\gamma_0 m c x- \gamma_0 m \dot{x} ct = H \frac{x}{c} - pct\]
Se pueden encontrar las ecuaciones del movimiento aplicando (E-L), que salen $x = x_0 + vt$, entonces $I = \gamma_0 m c x_0$, que efectivamente se conserva, puesto que $x_0$ y $v$ son constantes.

\newpage\null\thispagestyle{empty}\newpage

\chapter{Mecánica Hamiltoniana} 
\refstepcounter{section}
\section{Transformada de Legendre} \refstepcounter{subsection}
Si consideramos una función de una variable $y=f(x)$ tal que $f''(x)\neq 0$, entonces a cada punto le corresponde una sola recta tangente asociada, asociada con su pendiente $f'(x)$ y su ordenada en el origen $g(x) = f(x)-x f'(x)$, tal que $y = f'(x)x + g(x)$, a esta familia de rectas definida por el par $(f'(x),g)$ se le llama \textbf{envolvente} y contiene toda la información original de la función.
\begin{marginfigure}[0cm]
	\includegraphics{envelope.png}
	\labfig{margin2}
\end{marginfigure}
Así tenemos dos nuevas coordenadas $[p,g(p)]$, relacionadas con $[x,f(x)]$ mediante
\vspace{-5pt}
\begin{equation} \label{4.1.1}
    \begin{matrix}
        p(x)=f'(x) && \color{blue}g(p)\color{black}=f(x(p))-x(p)\color{blue}p\color{black} &&[x,f(x)] \mapsto [p,g(p)] \\
        x(p)=(f')^{-1}(p) &&  \color{blue}f(x) \color{black}=p(x)\color{blue}x\color{black}+g(p(x))  && [p,g(p)] \mapsto [x,f(x)]\
    \end{matrix}
\end{equation} \refstepcounter{subsection}
Donde la primera expresión es la \textit{Transformada de Legendre}, y será invertible (la segunda expresión) siempre que $f'(x)$ sea invertible (cierto si $f''(x)\neq 0$).
\subsection{Varias variables}
Si ahora tenemos $f(\{x_i,y_i\})$ donde $\{y_i\}$ son las variables sobre las que queremos hacer la transformada, la transformada es entonces
\begin{equation} \label{4.1.2}
    \begin{matrix}
        p_i(\{x_i,y_i\})=\frac{\partial f}{\partial y_i} && g(\{x_i,\color{blue}p_i\color{black}\}) =f(\{x_i,y_i(\{x_i,p_i\})\}) -\sum_j \color{blue}p_j\color{black} y_j(\{x_i,p_i\})  &&[y_i,f] \mapsto [p_i,g] \\
        y_i(\{x_i,p_i\})=\left[\frac{\partial f}{\partial y_i}\right]^{-1} && f(\{x_i,\color{blue}y_i\color{black}\}) =\sum_j \color{blue}y_j\color{black} p_j (\{x_i,y_i\})  +g(\{x_i,p_i(\{x_i,y_i\})\}) &&[p_i,g] \mapsto [y_i,f]\\
    \end{matrix}
\end{equation} \refstepcounter{subsection}
La transformación será inversible si el jacobiano de $y_i \mapsto p_i$ es no nulo.
\subsubsection{Transformada de Legendre del Lagrangiano}
Ahora si tenemos $\pazocal{L}(\{q_j,\dot{q}_j\};t)$, $\{\dot{q}_j\}$ serán nuestras antiguas variables y las nuevas variables serán $\partial_{\dot{q}_j}\pazocal{L}=p_j$, los momentos generalizados o conjugados. Entonces aplicando (4.1.2) llegamos a (3.1.2)
\begin{equation} \label{4.1.3}
        p_i(\{q_i,\dot{q}_i\};t)=\frac{\partial \pazocal{L}}{\partial q_i} \ \ \ \ \ \  g(\{q_i,p_i\};t) =\pazocal{L}(\{q_i,p_i\};t) -\sum_j^s \dot{q}_j (\{q_i,p_i\}) p_j = -\pazocal{H}
\end{equation} \refstepcounter{subsection}
De esta forma, $\pazocal{H}$ es equivalente a la \textit{Transformada de Legendre} de $\pazocal{L}$ con respecto a los $\dot{q}_j$, y esta es inversible, la demostración de que el jacobiano $[\partial_{\dot{q}_j}p_i]$ es no nulo bajo ciertas circumstancias se deja como un ejercicio al lector.

Se observa que la función $\pazocal{H}$, llamada \textit{Hamiltoniano} es funcionalmente igual a la función energía generalizada $H$ definida en el capítulo anterior, pero en función de las nuevas variables $q_j$ y $p_j$.

De esta forma, no hemos perdido ninguna información del sistema al pasar de $\pazocal{L}$ a $\pazocal{H}$, y a continuación reformularemos las ecuaciones del movimiento en función de esta cantidad de una forma equivalente a la fomulación lagrangiana.
\refstepcounter{section}
\section{Ecuaciones de Hamilton} \refstepcounter{subsection}
Si hacemos la diferencial exacta de $\pazocal{H}$ usando la regla de la cadena tenemos
\begin{equation} \label{4.2.1}
    d\pazocal{H} = \sum^s\left(\frac{\partial \pazocal{H}}{\partial q_j}dq_j+\frac{\partial \pazocal{H}}{\partial p_j}dp_j\right)+\frac{\partial \pazocal{H}}{\partial t} dt
\end{equation} \refstepcounter{subsection}
Si por otro lado hacemos el diferencial de $\pazocal{H}$ desde (3.1.2) o (4.1.3)
\begin{equation} \label{4.2.2}
    d\pazocal{H} = \sum^s\left(p_j d\dot{q}_j+\dot{q}_j dp_j\right)-d\pazocal{L}
\end{equation} \refstepcounter{subsection}
si $d\pazocal{L}$ es por regla de la cadena, y usando (2.2.1) y (2.2.2)
\begin{equation} \label{4.2.3}
    d\pazocal{L}= \sum^s\left(\frac{\partial \pazocal{L}}{\partial q_j}dq_j+\frac{\partial \pazocal{L}}{\partial \dot{q}_j}d\dot{q}_j\right) + \frac{\partial \pazocal{L}}{\partial t}dt = \sum^s\left(\dot{p}_j dq_j+p_j d\dot{q}_j\right) + \frac{\partial \pazocal{L}}{\partial t}dt
\end{equation} \refstepcounter{subsection}
Sustituyendo (4.2.3) en (4.2.2)
\begin{equation} \label{4.2.4}
    d\pazocal{H} = \sum^s p_j d\dot{q}_j+\dot{q}_j dp_j-\sum^s \dot{p}_j dq_j+p_j d\dot{q}_j - \frac{\partial \pazocal{L}}{\partial t}dt = \sum^s \dot{q}_j dp_j-\dot{p}_j dq_j - \frac{\partial \pazocal{L}}{\partial t}dt
\end{equation} \refstepcounter{subsection}
Como $dq_j$, $dp_j$ y $dt$ son funciones independientes y arbitrarias, podemos igualar término a término (4.2.4) y (4.2.1), de tal forma que obtenemos tres ecuaciones
\Large \begin{equation} \label{4.2.5}
    \boxed{\dot{q}_j = \frac{\partial \pazocal{H}}{\partial p_j} \ \ \ \ \ \ \dot{p}_j= -\frac{\partial \pazocal{H}}{\partial q_j}}
\end{equation} \refstepcounter{subsection} \normalsize
Estas dos primeras ecuaciones son las \textit{Ecuaciones de Hamilton} del movimiento o \textit{Ecuaciones canónicas}. Por otro lado tenemos la tercera ecuación, que junto a (3.1.2)
\begin{equation} \label{4.2.6}
    \frac{\partial \pazocal{H}}{\partial t} = - \frac{\partial \pazocal{L}}{\partial t} = \frac{d \pazocal{H}}{dt}
\end{equation} \refstepcounter{subsection}
De esta forma, si $\pazocal{H}$ no depende explícitamente del tiempo, este se conserva.

Para aplicar estas ecuaciones en un sistema holonómico tenemos que hayar primero $\pazocal{L}$, tras esto hallar los momentos generalizados y despues invertir la relación, tal que
\begin{equation} \label{4.2.7}
    p_j = \frac{\partial \pazocal{L}}{\partial \dot{q}_j}=p_j(\{q_k,\dot{q}_k\};t) \rightarrow \dot{q}_j = \dot{q}_j(\{q_k,p_k\};t)
\end{equation} \refstepcounter{subsection}
Entonces usamos la ecuación (4.1.3) con mucho cuidado de reemplazar todas las $\dot{q}_j$ por (4.2.7), y ya tendremos $\pazocal{H}$ en una forma que nos permita resolverlo usando (4.2.5).
\subsection{Variando la acción}
Además, usando (4.1.3) podemos escribir $\pazocal{L}$ como
\begin{equation} \label{4.2.7}
    \pazocal{L}(\{q_i,p_i\};t) = \sum_j^s p_j \dot{q}_j (\{q_i,p_i\};t) -\pazocal{H}(\{q_i,p_i\};t)
\end{equation} \refstepcounter{subsection}
Si hacemos la acción de ese lagrangiano tendremos, y al extremizarla se puede comprobar que se obtienen (4.2.5)
\begin{equation} \label{4.2.7}
    S = \int_{t_A}^{t_B} \left(\sum_j^s p_j \dot{q}_j -\pazocal{H}\right) dt
\end{equation} \refstepcounter{subsection}
Para ello, aplicamos los metodos explicados en el Capítulo 1, teniendo en cuenta que $\pazocal{L}$ no depende de $\dot{p}_i$ 
\[
    \delta S = 0 = \int_{t_A}^{t_B} \delta \pazocal{L} dt = \int_{t_A}^{t_B} \sum_i\left(\frac{\partial \pazocal{L}}{\partial q_i} \delta q_i + \frac{\partial \pazocal{L}}{\partial \dot{q}_i} \delta \dot{q}_i + \frac{\partial \pazocal{L}}{\partial p_i} \delta p_i\right) dt
\]
Integrando el segundo término por partes como en (1.2.1), usando que $\delta \dot{q}_i =  \dot{(\delta q_i)}$
\[
    \delta S = 0 = \int_{t_A}^{t_B} \sum_{i}\left[\left(\frac{\partial \pazocal{L}}{\partial q_i} - \frac{d}{dt}\left(\frac{\partial \pazocal{L}}{\partial \dot{q}_i}\right)\right) \delta q_i + \frac{\partial \pazocal{L}}{\partial p_i}\delta p_i\right] dt
\]
Haciendo las derivadas de $\pazocal{L}$ obtenemos
\begin{equation} \label{4.2.7}
    \delta S = 0 = \int_{t_A}^{t_B} \sum_{i}\left[\left(-\frac{\partial \pazocal{H}}{\partial q_i} - \dot{p}_i\right) \delta q_i + \left(\dot{q}_i-\frac{\partial \pazocal{H}}{\partial p_i}\right)\delta p_i\right] dt
\end{equation} \refstepcounter{subsection}
Y ahora, establecemos que los términos entre paréntesis deben ser 0, puesto que las variaciones son arbitrarias e independientes, obteniendo (4.2.5).
\vspace{-20pt}
\subsubsection{Comparación Ecs. Lagrange-Hamilton}
La formulación Lagrangiana es mejor para tratar con ligaduras, pero la hamiltoniana nos permite reducir el orden de la ecuación diferencial resultante cuando no hay dependencia explícita en una o varias de las $q_j$, puesto que en (3.1.5) $  \pazocal{L}$ sigue dependiendo de $\dot{q}_j$, solo conseguimos reducir en 1 el orden de un ecuación de E-L, mientras que en la formulación hamiltoniana, si una variable es cíclica, es decir $\partial_{q_j}\pazocal{H}=0$, entonces ya hemos resuelto $p_j=\alpha$ por (4.2.5.B) y también por definición $\pazocal{H}$ no depende de $q_j$, de esta forma nos hemos eliminado dos dependencias y reducir el orden en 2 unidades, podemos integrar $q_j$ usando (4.2.5.A) que como no depende de $q_j$ es una EDO separable que podremos resolver integrando al final una vez tengamos el resto de variables resueltas. 
\subsubsection{Ejemplo}
Un ejemplo sencillo es el péndulo simple donde 
\[\pazocal{L}=\frac{1}{2}ml^2\dot{\theta}^2+mgl\cos{\theta} \ \ \ \ \ \ p_\theta = \frac{\partial \pazocal{L}}{\partial \dot{\theta}}=ml^2\dot{\theta} \ \ \ \ \ \ \dot{\theta}=\frac{p_\theta}{ml^2}=\dot{\theta}(p_\theta)\]
Sustituyendo tenemos 
\[\pazocal{H}=p_\theta \dot{\theta} -\pazocal{L}=\frac{p_\theta^2}{ml^2} - \frac{p_\theta^2}{2ml^2}-mgl\cos\theta = \frac{p_\theta^2}{2ml^2}-mgl\cos\theta = T+U\]

Ahora aplicamos (4.2.5.A), tal que $\dot{\theta} = p_\theta/ml^2$, de donde sacamos que $\dot{p}_\theta=ml^2 \ddot{\theta}$ y de (4.2.5.B) sacamos $\dot{p}_\theta=-mgl\sin\theta$, igualando y depejando tenemos $\ddot{\theta} + g/l \sin\theta = 0$, la ecuación del movimiento.
\refstepcounter{section}
\section{Transformaciones canónicas} \refstepcounter{subsection}
Nos va a interesar encontrar un cambio de variables llamado transformaciones canónicas (restringidas, porque no dependen de $t$) de la forma $Q_i=Q_i(\{q_i,p_i\})$ y $P_i=P_i(\{q_i,p_i\})$ invertible que preserve (4.2.5) (Ecs. H.), es decir, que
\begin{equation} \label{4.2.7}
    \pazocal{K} (\{Q_i,P_i\};t) = \pazocal{H} (\{q_j(\{Q_i,P_i\}),p_j(\{Q_i,P_i\})\};t) \implies \dot{Q}_j = \frac{\partial \pazocal{K}}{\partial P_j} \ \ \ \ \ \ \dot{P}_j= -\frac{\partial \pazocal{K}}{\partial Q_j}
\end{equation} \refstepcounter{subsection}
Puesto que si conseguimos encontrar una transformación de este tipo en el que todas las coordenadas sean cíclicas, la resolución de las ecuaciones del movimiento será trivial.
\subsection{Matrices simplécticas (I)}
Vamos a usar la siguente notación para ver que tipos de transformaciones son canónicas, donde los corchetes indican que se recorren todos los índices, es decir que son vectores de $2s$ componentes
\begin{equation} \label{4.2.7}
    \mathbf{z}(\mathbf{Z})=\left[\begin{matrix} \{q_i\}\\ \{p_i\}\end{matrix}\right] \ \ \ \ \ \ \mathbf{Z}(\mathbf{z})=\left[\begin{matrix} \{Q_i\}\\ \{P_i\}\end{matrix}\right]
\end{equation} \refstepcounter{subsection}
\begin{equation} \label{4.2.7}
    \frac{\partial \pazocal{H}}{\partial \mathbf{z}}=\left[\begin{matrix} \left\{\frac{\partial \pazocal{H}}{\partial q_i}\right\} \\ \left\{\frac{\partial \pazocal{H}}{\partial p_i}\right\}\end{matrix}\right] \ \ \ \ \ \ \frac{\partial \pazocal{K}}{\partial \mathbf{Z}}=\left[\begin{matrix} \left\{\frac{\partial \pazocal{K}}{\partial Q_i}\right\} \\ \left\{\frac{\partial \pazocal{K}}{\partial P_i}\right\}\end{matrix}\right]
\end{equation} \refstepcounter{subsection}
\begin{equation} \label{4.2.7}
    \mathbb{J} = \left[\begin{array}{c|c} \mathbb{0}_s & \mathbb{I}_s \\ \hline -\mathbb{I}_s & \mathbb{0}_s \end{array}\right]
\end{equation} \refstepcounter{subsection}
Puede comprobarse entonces que podemos escribir (4.2.5) y (4.3.1) (Ecs. H.) como
\begin{equation} \label{4.2.7}
    \dot{\mathbf{z}} = \mathbb{J} \frac{\partial \pazocal{H}}{\partial \mathbf{z}} \ \ \ \ \ \ \ \dot{\mathbf{Z}} = \mathbb{J} \frac{\partial \pazocal{K}}{\partial \mathbf{Z}}
\end{equation} \refstepcounter{subsection}
Usando la regla de la cadena encontramos la siguiente expresión, donde $\mathbb{M}$ es la jacobiana de la transformación
\begin{equation} \label{4.2.7}
    \dot{\mathbf{Z}} = \mathbb{M}\dot{\mathbf{z}}  \ \ \ \ \ \ \ M_{ij} = \frac{\partial Z_i}{\partial z_j} \ \ \ \ \ \mathbb{M} = \left[\begin{array}{cc} \left\{\frac{\partial Q_i}{\partial q_j}\right\} & \left\{\frac{\partial Q_i}{\partial p_j}\right\} \\  \left\{\frac{\partial P_i}{\partial q_j}\right\} & \left\{\frac{\partial P_i}{\partial p_j}\right\} \end{array}\right]
\end{equation} \refstepcounter{subsection}
Ahora usando la regla de la cadena con (4.3.1) y aplicando el Teorema de la función inversa obtenemos la siguiente relación 
\begin{equation} \label{4.2.7}
    \frac{\partial \pazocal{K}}{\partial Z_j} = \sum_i^{2s} \frac{\partial \pazocal{H}}{\partial z_i} \frac{\partial z_i}{\partial Z_j} = \sum_i^{2s} ({M^{-1}})_{ij} \frac{\partial \pazocal{H}}{\partial z_i} \implies \frac{\partial \pazocal{K}}{\partial \mathbf{Z}} = (\mathbb{M}^{-1})^T \frac{\partial \pazocal{H}}{\partial \mathbf{z}} \rightarrow  \frac{\partial \pazocal{H}}{\partial \mathbf{z}}  = \mathbb{M}^T \frac{\partial \pazocal{K}}{\partial \mathbf{Z}}
\end{equation} \refstepcounter{subsection}
Ahora introducimos (4.3.5) en (4.3.6) y después introducimos (4.3.7)
\begin{equation} \label{4.2.7}
    \mathbb{J} \frac{\partial \pazocal{K}}{\partial \mathbf{Z}} = \mathbb{M}\mathbb{J} \frac{\partial \pazocal{H}}{\partial \mathbf{z}} \rightarrow \mathbb{J} \frac{\partial \pazocal{K}}{\partial \mathbf{Z}} = \mathbb{M}\mathbb{J} \mathbb{M}^T \frac{\partial \pazocal{K}}{\partial \mathbf{Z}} \implies \mathbb{M}\mathbb{J} \mathbb{M}^T = \mathbb{J} 
\end{equation} \refstepcounter{subsection}
Si la jacobiana $\mathbb{M}$ es una matriz simpléctica, es decir, cumple (4.3.8), entonces la transformación es canónica (para cualquier problema, no depende del \textit{Hamiltoniano}).
\newpage
Si se hace el producto matricial de $\mathbb{J} \mathbb{M}^T = \mathbb{M}^{-1}\mathbb{J}$ equivalente a (4.3.8) se pueden obtener unas condiciones más explícitas para concluir si una trasformación es canónica
\begin{equation} \label{4.2.7}
    \begin{split}
        \frac{\partial Q_i(\{q_k,p_k\})}{\partial q_j} = \frac{\partial p_j(\{Q_k,P_k\})}{\partial P_i} \ \ \ \ \ \ \ \ \frac{\partial Q_i(\{q_k,p_k\})}{\partial p_j} = -\frac{\partial q_j(\{Q_k,P_k\})}{\partial P_i} \\ 
        \frac{\partial P_i(\{q_k,p_k\})}{\partial q_j} = -\frac{\partial p_j(\{Q_k,P_k\})}{\partial Q_i} \ \ \ \ \ \ \ \ \frac{\partial P_i(\{q_k,p_k\})}{\partial p_j} = \frac{\partial q_j(\{Q_k,P_k\})}{\partial Q_i}
    \end{split}
\end{equation} \refstepcounter{subsection}
Existe un teorema\sidenote{Sothanaphan, Nat. "Determinants of block matrices with noncommuting blocks" \hspace{1pt}\href{https://arxiv.org/abs/1805.06027}{arXiv:1805.06027l}} que dice que si las matrices $\mathbb{C}$ y $\mathbb{D}$ de una matriz de bloques del mismo tamaño ($\mathbb{A}$, $\mathbb{B}$, $\mathbb{C}$ y $\mathbb{D}$) conmutan, entonces $\det{(\mathbb{J})} = \det{(\mathbb{A}\mathbb{D}-\mathbb{B}\mathbb{C})}$. En el caso de $\mathbb{J}$, $\mathbb{D}= \mathbb{0}$, por lo tanto conmuta con cualquier matriz y el determinante es $\det{(\mathbb{J})} = \det{(\mathbb{0}\mathbb{0}-\mathbb{I}(-\mathbb{I}))} = 1$. Otras propiedades de $\mathbb{J}$ son que $\mathbb{J}^2 = -\mathbb{I}_{2s}$ o que $\mathbb{J}^{-1} = \mathbb{J}^T = -\mathbb{J}$.

Gracias a esto sabemos que $\det{(\mathbb{M}\mathbb{J} \mathbb{M}^T)} = \det{\mathbb{M}} \det{\mathbb{J}} \det{\mathbb{M}^T}=\det{\mathbb{M}}^2 = 1$
\subsection{Funciones generadoras}
El caso anterior es un caso particular para transformaciones canónicas restringidas, ahora vamos a estudiar un caso más general, en el que dependen del tiempo. Como se demostró antes, si tenemos la acción (4.2.9), entonces si su variación es $0$ se verifican las ecuaciones de hamilton, entonces podemos escribir lo siguiente
\begin{equation} \label{4.2.7}
    \delta S = \delta\int_{t_A}^{t_B} \left(\sum_j^s p_j \dot{q}_j -\pazocal{H}(\{q_i,p_i\};t)\right) dt = 0 = \delta\int_{t_A}^{t_B} \left(\sum_j^s P_j \dot{Q}_j -\pazocal{K}(\{Q_i,P_i\};t)\right) dt = \delta \tilde{S}
\end{equation} \refstepcounter{subsection}
Puesto que queremos que las nuevas coordenadas $Q_i=Q_i(\{q_i,p_i\};t)$ y $P_i=P_i(\{q_i,p_i\};t)$ verifiquen (4.2.5) (Ecs. H.) para una cierta función $\pazocal{K}$ que en este caso no necesariamente verifica (4.3.1). (4.3.10) es equivalente a la siguiente expresión, usando lo demostrado en (1.2.5) y que la constante conmuta con la variación
\begin{equation} \label{4.2.7}
    \lambda\left(\sum_j^s p_j \dot{q}_j -\pazocal{H}\right) = \sum_j^s P_j \dot{Q}_j -\pazocal{K} +  \frac{dF(\{q_i,p_i\},\{Q_i,P_i\};t)}{dt}
\end{equation} \refstepcounter{subsection}
La constante $\lambda$ es un factor de escala, si por ejemplo $Q_j = \mu q_j$, $P_j = \nu p_j$ y $\pazocal{K} = \mu \nu \pazocal{H}$, entonces tenemos que la ecuación de arriba nos queda $\mu\nu\left(p_j \dot{q}_j -\pazocal{H}\right) = P_j \dot{Q}_j -\pazocal{K}$ (usando el criterio de suma de indices de Einstein), por lo tanto $\lambda$ no es nada mas que el producto de dos constantes de escala, y a partir de una transformación canónica, siempre podemos encontrar otra simplemente reescalando tal que $\lambda = 1$, así, para simplificar, nos centraremos en la siguiente expresión 
\begin{equation} \label{4.2.7}
    \sum_j^s p_j \dot{q}_j -\pazocal{H} = \sum_j^s P_j \dot{Q}_j -\pazocal{K} + \frac{dF(\{q_i,p_i\},\{Q_i,P_i\};t)}{dt}
\end{equation} \refstepcounter{subsection}
La función $F$, cuya variación de su acción se anula porque las variaciones de todas las coordenadas se anulan en los extremos, se denomina \textit{Función generadora}. Se llama así porque nos va a permitir generar transformaciones canónicas a partir de transformaciones canónicas parciales.
\newpage
Ahora, la dependencia de esa función es redundante, puesto que tenemos dos relaciones $Q_i=Q_i(\{q_i,p_i\};t)$ y $P_i=P_i(\{q_i,p_i\};t)$, o equivalentes obtenidas despejando esas, entonces podemos estudiar casos concretos de dependencias de $F$ en función de qué relaciones tengamos.

Vamos a suponer primero que $F$ tiene la dependencia $F = F_1(\{q_i,Q_i\};t)$, entonces haciendo la regla de la cadena en (4.3.12) tenemos
\begin{equation} \label{4.2.7}
    \sum_j^s p_j \dot{q}_j -\pazocal{H} = \sum_j^s P_j \dot{Q}_j -\pazocal{K} + \frac{\partial F_1}{\partial t} + \sum_j^s \left(\frac{\partial F_1}{\partial q_j}\dot{q}_j + \frac{\partial F_1}{\partial Q_j}\dot{Q}_j\right)
\end{equation} \refstepcounter{subsection}
Para que se cumpla esta ecuación idénticamente deben cumplirse las siguientes relaciones 
\begin{equation} \label{4.2.7}
    \frac{\partial F_1}{\partial q_i} = p_i \ \ \ \ \ \ \ \ \frac{\partial F_1}{\partial Q_i} = -P_i \ \ \ \ \ \ \ \ \frac{\partial F_1}{\partial t} = \pazocal{K}-\pazocal{H}
\end{equation} \refstepcounter{subsection}
Entonces si por ejemplo conocemos $P_i=P_i(\{q_j,Q_j\};t)$ pero no $Q_i$, podemos integrar primero la segunda expresión de (4.3.14), teniendo $F_1 = -\sum_j \int P_j dQ_j + h(\{q_j\};t)$, con h arbitraria, entonces introducimos en la primera expresión y tendremos $p_i \{q_j,Q_j\}= -\sum_j\int \partial_{q_i} P_j dQ_j + \partial_{q_i} h(\{q_j\};t)$, y entonces ahora podemos invertir la relación para tener $Q_i=Q_i(\{q_j,p_j\};t)$, ahora volvemos a sustituir esta última expresión en $P_i$, así, obtenemos una familia de transformaciones canónicas, y también obtenemos la familia $\pazocal{K} = \pazocal{H}-\sum_j\int \partial_t P_j dQ_j + \partial_t h(\{q_j\};t)$ con respecto a las cuales se verifican las ecuaciones canónicas. También podríamos haber partido de de $p_i=p_i(\{q_j,Q_j\};t)$ para obtener otra familia de transfomaciones canónicas.

Si tenemos otras relaciones, como por ejemplo $Q_i=Q_i(\{q_j,P_j\};t)$, podemos definir otra función $F = F_2(\{q_j,P_j\};t) - \sum_j Q_j P_j$, siendo $F_2$ equivalente a una Transformada de Legendre de $F_1$ en su segundo conjunto de variables, pero $F \neq F_1(\{q_i,Q_i(\{q_j,P_j\})\};t)$, sino que en general es una función de todas variables como se indicó en (4.3.12). Si sustituimos esta nueva dependencia y aplicamos la regla de la cadena teniendo en cuenta esta dependencia obtenemos una ecuación similar a (4.3.13) en la que se cancelan algunos términos y obtenemos las siguientes relaciones que se deben cumplir para que esta sea cierta
\begin{equation} \label{4.2.7}
    \frac{\partial F_2}{\partial q_i} = p_i \ \ \ \ \ \ \ \ \frac{\partial F_2}{\partial P_i} = Q_i \ \ \ \ \ \ \ \ \frac{\partial F_2}{\partial t} = \pazocal{K}-\pazocal{H}
\end{equation} \refstepcounter{subsection}
Podemos aplicar este procedimiento de hacer 'como' una Transformada de Legendre dos veces con las dos expresiones anteriores para obtener otras dos funciones distintas con sus relaciones correspondientes. Estas 4 funciones generadoras se resumen en la siguiente tabla
\begin{equation} \label{4.2.7}
        \arraycolsep=3pt\def\arraystretch{1.5}
        \begin{array}{|c|ccc|} \hline \mbox{Función generadora} & & \mbox{Derivadas} & \\ \hline 
            F = F_1(\{q_i,Q_i\};t) & \frac{\partial F_1}{\partial q_i} = p_i &  \frac{\partial F_1}{\partial Q_i} = -P_i & \frac{\partial F_1}{\partial t} = \pazocal{K}-\pazocal{H} \\ \hline 
            F = F_2(\{q_i,P_i\};t) -\sum_j Q_j P_j& \frac{\partial F_2}{\partial q_i} = p_i &  \frac{\partial F_2}{\partial P_i} = Q_i  & \frac{\partial F_2}{\partial t} = \pazocal{K}-\pazocal{H} \\ \hline 
            F = F_3(\{p_i,Q_i\};t) +\sum_j q_j p_j& \frac{\partial F_3}{\partial p_i} = -q_i &  \frac{\partial F_3}{\partial Q_i} = -P_i  & \frac{\partial F_3}{\partial t} = \pazocal{K}-\pazocal{H}\\ \hline 
            F = F_4(\{p_i,P_i\};t) +\sum_j (q_j p_j - Q_j P_j)& \frac{\partial F_4}{\partial p_i} = -q_i & \frac{\partial F_4}{\partial P_i} = Q_i  & \frac{\partial F_4}{\partial t} = \pazocal{K}-\pazocal{H}\\ \hline 
        \end{array}
\end{equation} \refstepcounter{subsection}
Entonces podremos aplicar el procedimiento que he explicado antes de forma análoga para $F_i$ cuando tengamos una relación de dependencias de variables $A_i=A_i(\{b_j,C_j\};t)$ (también es posible que solo dependa de unas de ellas, por lo general las primeras) igual a la de $F_i$ especificada en (4.3.16)

Es posible (incluso necesario en algunos casos) mezclar funciones generadoras cuando tengamos varios grados de libertad, es decir, usar distintas funciones generadoras para distintas variables.
\subsection{Matrices simplécticas (II)}
Como se observa en (4.3.15), si la función generadora no depende explícitamente del tiempo, entonces $\pazocal{H} = \pazocal{K}$ y se recupera la condición (4.3.1) usada para demostrar (4.3.8) cuando la transformación canónica no dependía explítamente del tiempo, por lo tanto ambos acercamientos son equivalentes.

Ahora, queremos saber que ocurre con la condición (4.3.8) cuando la transformación depende explícitamente del tiempo, para ello vamos a desarrollar en pequeñas variaciones mezclando ambos acercamientos.

Primero, escribimos una transformación infinitesimal de la siguiente forma forma, 
\begin{equation} \label{4.2.7}
    \mathbf{Z} = \mathbf{Z}(\mathbf{z},t) = \vec{\zeta}(\mathbf{Z}(\mathbf{z},t_0),t) \rightarrow \vec{\zeta}(\vec{\mu},\epsilon) = \vec{\mu} + \epsilon \vec{\eta}(\vec{\mu}) +O(\epsilon^2) \ \ \ \ \ \ \epsilon = t-t_0 \ \ \ \ \ \ \vec{\mu} = \mathbf{Z}(\mathbf{z},t_0) \ \ \ \ \ \  \vec{\eta}(\mathbf{z}(\vec{\mu})) = \left.\frac{\partial \mathbf{Z}}{\partial t}\right|_{t=t_0}
\end{equation} \refstepcounter{subsection}
Ahora podemos escribir la siguiente función generadora, dónde $\tilde{q}_i$ y $\tilde{p}_i$ son las componentes de $\vec{\mu}$, el primer sumatorio es la función generadora de la identidad

\vspace{-15pt}
\begin{equation} \label{4.2.7}
    F_2(\{\tilde{q}_i,P_i\};t) = \sum_j^s \tilde{q}_j P_j + \epsilon G(\{\tilde{q}_i,P_i\}) + O(\epsilon^2)
\end{equation} \refstepcounter{subsection}

\vspace{-20pt}
Usando (4.3.16) nos da la relación con la transformación anterior

\vspace{-18pt}
\begin{equation} \label{4.2.7}
    \begin{split}
        \tilde{p}_i = P_i + \epsilon \frac{\partial G}{\partial \tilde{q}_i} + O(\epsilon^2) \ \ \ \ \ P_i-\tilde{p}_i &= \epsilon \frac{\partial G}{\partial \tilde{q}_i} + O(\epsilon^2)= \epsilon \eta_i +O(\epsilon^2)\ \ \ i = 1,\dots,s\\ 
        Q_i = \tilde{q}_i + \epsilon \frac{\partial G}{\partial P_i} + O(\epsilon^2)\ \ \ \ \ Q_i-\tilde{q}_i &= -\epsilon \frac{\partial G}{\partial P_i} + O(\epsilon^2)= \epsilon \eta_i +O(\epsilon^2)\ \ \ i = s+1,\dots,2s \\
        \pazocal{K} &= \pazocal{H} + G
    \end{split}
\end{equation} \refstepcounter{subsection}

\vspace{-30pt}
Ahora en la siguiente expresión podemos sustituir $P_i$ y expandir

\vspace{-15pt}
\begin{equation} \label{4.2.7}
        \epsilon \frac{\partial G}{\partial P_i} = \epsilon \frac{\partial G (\{\tilde{q}_i,\tilde{p}_i + \epsilon \partial{\tilde{q}_i} G\})}{\partial (\tilde{p}_i + \epsilon \partial{\tilde{q}_i} G)} = \epsilon \frac{\partial G (\{\tilde{q}_i,\tilde{p}_i + 0 \cdot  \partial_{\tilde{q}_i} G\})}{\partial (\tilde{p}_i + 0 \cdot \partial_{\tilde{q}_i} G)} +\epsilon^2 \left.\frac{\partial^2 G (\{\tilde{q}_i,P_i\})}{\partial^2 P_i }\right|_{\epsilon=0} \hspace{-10pt}= \epsilon \frac{\partial G}{\partial \tilde{p}_i} + O(\epsilon^2)
\end{equation} \refstepcounter{subsection}
Entonces ahora podemos escribir sustituyendo en (4.3.19), considerando $G = G(\{\tilde{q}_i,\tilde{p}_i\})$
\begin{equation} \label{4.2.7}
    \begin{split}
        \eta_i &= \ \ \ \frac{\partial G}{\partial \tilde{q}_i} +O(\epsilon) \ \ \ \ i = 1,\dots,s\\ 
        \eta_i &= -\frac{\partial G}{\partial \tilde{p}_i} +O(\epsilon) \ \ \ \ i = s+1,\dots,2s
    \end{split} \implies \vec{\eta} = \mathbb{J} \frac{\partial G}{\partial \vec{\mu}} +O(\epsilon)
\end{equation} \refstepcounter{subsection}
Si ahora calculamos el jacobiano ($\mathbb{M}$) de $\vec{\zeta}$ (4.3.17) usando (4.3.21), llegamos a la siguiente expresión, donde $\mathbb{H}$ es la Hessiana de $G$
\begin{equation} \label{4.2.7}
    \mathbb{M} = \frac{\partial \vec{\zeta}}{\partial \vec{\mu}} = \mathbb{I} + \epsilon \mathbb{J} \mathbb{H} + O(\epsilon^2) \ \ \ \ \ \ \mathbb{H}_{ij} = \frac{\partial^2 G}{\partial \mu_i \partial \mu_j}
\end{equation} \refstepcounter{subsection}
Ahora podemos que verifica la condición (4.3.8) a primer orden, aplicando el Teorema de Schwarz ($\mathbb{H}^T = \mathbb{H}$) y que $\mathbb{J}^T = -\mathbb{J}$
\begin{equation} \label{4.2.7}
    \begin{split}
        \mathbb{M}^T = \mathbb{I} + \epsilon \mathbb{H}^T \mathbb{J}^T  + O(\epsilon^2) &= \mathbb{I} - \epsilon \mathbb{H} \mathbb{J}  + O(\epsilon^2)\\ 
        \mathbb{J} = \left(\mathbb{I} + \epsilon  \mathbb{J}\mathbb{H} + O(\epsilon^2)\right) \mathbb{J} \left(\mathbb{I} - \epsilon \mathbb{H} \mathbb{J}  + O(\epsilon^2)\right) &= \mathbb{J} + \epsilon  \mathbb{J}\mathbb{H}\mathbb{J} -  \epsilon  \mathbb{J}\mathbb{H}\mathbb{J} + O(\epsilon^2) = \mathbb{J} + O(\epsilon^2)
    \end{split}
\end{equation} \refstepcounter{subsection}
Por lo tanto podemos concluir que la transformación $\mathbf{Z}(\mathbf{z};t_0) \rightarrow \mathbf{Z}(\mathbf{z};t_0 + \epsilon)$ es canónica ($\mathbf{z} \rightarrow \mathbf{Z}(\mathbf{z};t_0)$ tiene que ser canónica en primer lugar) y que verifica (4.3.8) a primer orden.

De esta forma podemos escribir la transformación original como 
\begin{equation} \label{4.2.7}
    \mathbf{Z}(\mathbf{z},t) = \lim_{\epsilon \rightarrow 0 \ \ n \rightarrow \infty} \vec{\zeta}_n(\vec{\zeta}_{n-1}(\dots(\vec{\zeta}_0))) \ \ \ \ \ \vec{\zeta}_0 = \mathbf{Z}(\mathbf{z},t_0) \ \ \ \  n = \frac{t_n-t_0}{t_1-t_0} = \frac{t-t_0}{\epsilon} \ \ \ \ \ \ \mathbb{J} = \mathbb{M}_\mathbf{z}\mathbb{J}\mathbb{M}_\mathbf{z}^T
\end{equation} \refstepcounter{subsection}
Esta expresión es equivalente a aplicar el método de Euler para integrar una EDO de aplicando la ecuación $\vec{\zeta}$ de forma iterativa, además, es sencillo demostrar que la jacobiana de la composición es el producto de las jacobianas, pudiendo sustuir en (4.3.24) y usando (4.3.23) iterativamente comprobar que la expresión es cierta. Así, hemos demostrado que cualquier transformación, incluso si depende del tiempo, es canónica si y sólo si su jacobiano $\mathbb{M}$ verifica (4.3.8) ($\mathbb{J} = \mathbb{M}_\mathbf{z}\mathbb{J}\mathbb{M}_\mathbf{z}^T$) (aunque si depende del tiempo no necesariamente $\pazocal{H}=\pazocal{K}$).
Se puede demostrar además que las transfomaciones canónicas, y concreto las matrices $\mathbb{M}$ que verifican (4.3.8) ($\mathbb{J} = \mathbb{M}_\mathbf{z}\mathbb{J}\mathbb{M}_\mathbf{z}^T$) forman un grupo bajo la operación composición.


\refstepcounter{section}
\section{Paréntesis de Poisson (I)} \refstepcounter{subsection}
Sea $f=f(\{q_j,p_j\};t)$ una función de las coordenadas canónicas, podemos hacer su derivada total con respecto al tiempo, tal que
\begin{equation} \label{4.4.1}
    \frac{d f}{dt} = \sum^s \left(\frac{\partial f}{\partial q_j}\dot{q}_j+\frac{\partial f}{\partial p_j}\dot{p}_j\right) + \frac{\partial f}{\partial t}
\end{equation} \refstepcounter{subsection}
Usando (3.2.5) (Ecs. H.) llegamos a 
\begin{equation} \label{4.4.2}
\frac{d f}{dt} = \sum^s \left(\frac{\partial f}{\partial q_j}\frac{\partial \pazocal{H}}{\partial p_j}-\frac{\partial f}{\partial p_j}\frac{\partial \pazocal{H}}{\partial q_j}\right) + \frac{\partial f}{\partial t} = [f,\pazocal{H}]_\mathbf{z} + \frac{\partial f}{\partial t}
\end{equation} \refstepcounter{subsection}
Dónde $[f,\pazocal{H}]$ es el \textit{paréntesis de Poisson} de $f$ y $\pazocal{H}$, en general lo definimos para dos funciones y unas ciertas variables canónicas $\mathbf{z}$ como
\begin{equation} \label{4.4.3}
    [f,g]_\mathbf{z}=  \sum^s \left(\frac{\partial f}{\partial q_j}\frac{\partial g}{\partial p_j}-\frac{\partial f}{\partial p_j}\frac{\partial g}{\partial q_j}\right) = \left[\frac{\partial f}{\partial \mathbf{z}}\right]^T \hspace{-7pt} \mathbb{J} \left[\frac{\partial g}{\partial \mathbf{z}}\right]
\end{equation} \refstepcounter{subsection}
Sus propiedades algebraicas son muy similares a aquellas del producto vectorial puesto que su expresión es muy similar, puesto que ambos son ejemplos de álgebras de Lie, sus propiedades son y se pueden verificar reemplazando en (4.4.3).
\begin{itemize}
    \item Es alternada $[f,g]=-[g,f]$ y $[f,f]=0$.
    \item Si $[f,g]=0 \iff [f,g]=[g,f]=0$ las funciones conmutan.
    \item Es bilineal, $[f,\alpha g + \beta h] = \alpha [f,g] + \beta [f,h]$ en ambas entradas.
    \item Existe una regla del producto $[f,gh]=g[f,h]+[f,g]h$.
    \item Se verifica la \textit{Identidad de Jacobi}, $\left[f,[g,h]\right]+\left[h,[f,g]\right]+\left[g,[h,f]\right]=0$.
\end{itemize}
%Otra regla del producto que se verifica, usando la conmutividad de las derivadas parciales, es $\frac{\partial}{\partial t}[f,g]=[\frac{\partial f}{\partial t},g]+[f,\frac{\partial g}{\partial t}]$.

Si la función $f$ no depende explícitamente del tiempo, entonces si $f$ conmuta con $\pazocal{H}$, eso implica por (4.4.2) y las propiedades anteriores, que $f$ se conserva.

Además, si tenemos dos cantidades conservadas $f$ y $g$, entonces tenemos que, usando la \textit{Identidad de Jacobi}, se conserva su paréntesis
\begin{equation} \label{4.4.5}
    \frac{d}{dt}[f,g]_\mathbf{z}=0
\end{equation} \refstepcounter{subsection}
Si hacemos $[q_k,\pazocal{H}]$ y $[p_k,\pazocal{H}]$ aplicando (4.4.3) y (3.2.5) (Ecs. H.), obtenemos las ecuaciones del movimiento expresadas en términos de \textit{paréntesis de Poisson}.
\begin{equation} \label{4.4.6}
    \boxed{[q_k,\pazocal{H}]_\mathbf{z} = \dot{q}_k \ \ \ \ \ [p_k,\pazocal{H}]_\mathbf{z}=\dot{p}_k} \iff [\mathbf{z},\pazocal{H}]_\mathbf{z} = \dot{\mathbf{z}}
\end{equation} \refstepcounter{subsection}
Tenemos también los paréntesis de \textit{paréntesis de Poisson} fundamentales
\begin{equation} \label{4.4.7}
    [q_k,q_l]_\mathbf{z}=[p_k,p_l]_\mathbf{z}=0 \ \ \ \ \ [q_k,p_l]_\mathbf{z}=\delta_{kl} \iff [\mathbf{z},\mathbf{z}]_\mathbf{z} = \mathbb{J}
\end{equation} \refstepcounter{subsection}

\subsection{Invariancia bajo T. Canónicas}
Como el Corchete de Poisson esta relacionado con las ecuaciones de Hamilton y con la evolución temporal de funciones del sistema, esperamos que este se mantenga invariante bajo una transformación canónica, demostrémoslo.

Vamos a suponer ahora unas transformaciones $\mathbf{Z}=\mathbf{Z}(\mathbf{z};t)$ canónicas, tal que $\mathbb{J} = \mathbb{M}_\mathbf{z}\mathbb{J}\mathbb{M}_\mathbf{z}^T$ para la jacobiana de la transformación $\mathbb{M}$. Usando la expresión (4.3.7) para una función arbitraria y su equivalente en las nuevas coordenadas, en vez tomar especificamente el \textit{Hamiltoniano}, tenemos
\begin{equation} \label{4.4.7}
    [f,g]_\mathbf{z} = \left[\frac{\partial f}{\partial \mathbf{z}}\right]^T \hspace{-7pt} \mathbb{J} \left[\frac{\partial g}{\partial \mathbf{z}}\right] = \left[\mathbb{M}^T\frac{\partial f}{\partial \mathbf{Z}}\right]^T \hspace{-7pt} \mathbb{J} \left[\mathbb{M}^T\frac{\partial g}{\partial \mathbf{Z}}\right] = \left[\frac{\partial f}{\partial \mathbf{Z}}\right]^T \hspace{-7pt}\mathbb{M} \mathbb{J}\mathbb{M}^T \left[\frac{\partial g}{\partial \mathbf{Z}}\right] = \left[\frac{\partial f}{\partial \mathbf{Z}}\right]^T \hspace{-7pt} \mathbb{J} \left[\frac{\partial g}{\partial \mathbf{Z}}\right] = [f,g]_\mathbf{Z}
\end{equation} \refstepcounter{subsection}
De esta forma, el paréntesis queda invariante si y sólo si la transfomación es canónica. De hecho podemos reformular la condición de que una transformación sea canónica a que mantenga el paréntesis invariante, puesto que son equivalentes.

Además, ahora no vamos a especificar las coordenadas con respecto a las cuales hacemos el paréntesis y todas las fórmulas de la sección anterior son válidas para cualquier conjunto de coordendas canónicas.
\subsection{Corchete de Lagrange}
Podemos definir otra operación similar, llamada Corchete de Lagrange, que se puede demostrar de manera muy similar a (4.4.7) que se conserva bajo transformaciones canónicas
\begin{equation} \label{4.4.3}
    \{f,g\}=  \sum^s \left(\frac{\partial q_j}{\partial f}\frac{\partial p_j}{\partial g}-\frac{\partial p_j}{\partial f}\frac{\partial q_j}{\partial g}\right) = \left[\frac{\partial \mathbf{z}}{\partial f}\right]^T \hspace{-7pt} \mathbb{J} \left[\frac{\partial \mathbf{z}}{\partial g}\right]
\end{equation} \refstepcounter{subsection}
A priori, parece como hacer la 'inversa' de los corchetes de Poisson, y en cierto sentido lo es, si $\mathbf{u}$ son $2s$ funciones independientes, puede demostrarse la siguiente expresión
\begin{equation} \label{4.4.3}
    [\mathbf{u},\mathbf{u}]\{\mathbf{u},\mathbf{u}\}=  -1
\end{equation} \refstepcounter{subsection}
\refstepcounter{section}
\section{Espacio de fase} \refstepcounter{subsection}
El hecho de que solo haya una sola solución para las ecuaciones del movimiento, es decir, que solo hay una posible trayectoria dadas unas condiciones dadas, significa que el sistema con el que estamos tratando es \textit{determinista}.

Si es el espacio de configuración es $\{q_j\}$ para un $t$ dado, entonces definimos el \textbf{Espacio de fase} como $\{q_j,p_j\}$, donde $\{p_j\}$ es el espacio de momentos o impulsos.

Este espacio es de dimensión $2s$ y nos da toda la información dinámica del sistema pues nos permite predecir su evolución, puesto que con unas condiciones iniciales de posición y momento (o velocidad) definidas por unas coordendas del espacio de fase, podemos usar (4.2.5) (Ecs. H.) para hallar la evolución del sistema.
\subsection{Diagrama de fases}
Es la trayectoria que sigue un sistema en el espacio de fase, normalmente representada en un conjunto de $s$ planos bidimensionales como una curva en cada uno de ellos, cuyos ejes representan $q_j$ y $p_j$, donde por cada punto en un $t$ dado solo puede pasar una sola trayectoria, de lo contrario el sistema no sería determinista, ya que de unas mismas condiciones iniciales podría evolucionar de varias formas.

Además si $\pazocal{H}$ se conserva, entonces por cada punto del espacio de fase solo puede pasar una trayectoria independientemente del tiempo.

\subsubsection{Ejemplo}
\begin{marginfigure}[0cm]
	\includegraphics{phase.png}
	\labfig{margin3}
\end{marginfigure}
Como ejemplo vamos a ver un péndulo, tomando las expresiones del ejemplo de (4.2), donde las coordenadas del espacio de fases son $(\theta,p_\theta)$, si $\theta \ll 1$, tenemos $\ddot{\theta}+\frac{g}{l}\theta=0$, cuya solución, donde $\omega^2=g/l$, es
\[
    \begin{split}
        \theta = A \sin{(\omega t + \theta_0)}+ B \cos{(\omega t + \theta_0)} \ \ \ \ \dot{\theta} &= A\omega \cos{(\omega t + \theta_0)}- B \omega \sin{(\omega t + \theta_0)} \ \ \ \ p_\theta = ml^2 \dot{\theta}\\ 
        \theta^2 &+ \frac{p_\theta^2}{\omega^2 m^2l^4}=A^2+B^2 \mbox{ (elipse)}
    \end{split}    
\]

\vspace{-35pt}

\subsection{Teorema de Liouville} \refstepcounter{subsection}
\subsubsection{Volumen en el espacio de fase}
Definimos el volumen en el espacio de fases como
\begin{equation} \label{4.3.1}
    V = \int \dots \int \prod_j^s d q_j d p_j
\end{equation} \refstepcounter{subsection}
Donde $d q_j d p_j$ son elementos de área en uno de los $s$ planos. Es sencillo demostrar que una transformación deja invariante el volumen puesto que al hacer un cambio de variable hay que introducir el valor absoluto del determinante de la jacobiana, que como demostramos, es $1$, por lo tanto $dV$ se conserva y $V$ también.

Si ahora tenemos una serie de condiciones iniciales distribuidas dentro de una región volumétrica del espacio de fases, siendo $\pazocal{N}$ el número de condiciones iniales dentro de $V$, entonces si consideramos como la frontera de $V$ se transforma con el tiempo para dar $V'$, entonces $\pazocal{N}$ se conserva, puesto que para que una trayectoria entre o salga del volumen sería necesario que cortase una trayectoria de la frontera, lo cual no puede ocurrir en un sistema determinista.
\subsubsection{Teorema de Liouville} \refstepcounter{subsection}
El volumen $V(S)$ dentro de una superficie $S(t)$ del espacio de fase se conserva.
\begin{equation} \label{4.3.2}
    \frac{dV}{dt}=0 \implies \frac{d\rho}{dt} = 0 \ \ \ \ \rho = \frac{\pazocal{N}}{V}
\end{equation} \refstepcounter{subsection}

\textbf{Demostración}

Sea $\mathbf{\nabla} = (\mathbf{\nabla}_\textbf{q},\mathbf{\nabla}_\textbf{p})$, entonces la divergencia de $\dot{\mathbf{z}}$, tal que
\begin{equation} \label{4.3.4}
    \mathbf{\nabla} \cdot \dot{\mathbf{z}} = \sum^s \frac{\partial \dot{q}_j}{\partial q_j} + \frac{\partial \dot{p}_j}{\partial p_j}
\end{equation} \refstepcounter{subsection}
entonces por el Teorema de la Divergencia
\begin{equation} \label{4.3.5}
    \int_V{\mathbf{\nabla} \cdot \dot{\mathbf{z}} dV} = \int_S \dot{\mathbf{z}} \cdot d\mathbf{S}
\end{equation} \refstepcounter{subsection}
La variación de $V$ en términos del tiempo es la siguiente, ya que $\dot{\mathbf{z}}dt$ indica como se mueven las partículas de dentro de $V$, y multiplicando por $d\mathbf{S}$ nos indica como varía el volumen infinitesimalmente en un punto de la superficie, integrando en la superficie para ver la variación total de $V$ tenemos
\begin{equation} \label{4.3.6}
    dV=\int_S \dot{\mathbf{z}} \cdot d\mathbf{S} dt \implies \frac{dV}{dt} = \int_S \dot{\mathbf{z}} \cdot d\mathbf{S}
\end{equation} \refstepcounter{subsection}
Combinando (4.4.4), (4.4.5) y sustituyendo (4.4.3) llegamos a 
\begin{equation} \label{4.3.7}
    \frac{dV}{dt} = \int_V{\mathbf{\nabla} \cdot \dot{\mathbf{z}} dV} = \int_V{\left(\sum^s \frac{\partial \dot{q}_j}{\partial q_j} + \frac{\partial \dot{p}_j}{\partial p_j}\right)dV}
\end{equation} \refstepcounter{subsection}
Ahora usando (4.2.5) (Ecs. H.) y que las parciales conmutan.
\begin{equation} \label{4.3.8}
    \frac{dV}{dt} = \int_V{\left(\sum^s \frac{\partial \pazocal{H}}{\partial q_j p_j} - \frac{\partial \pazocal{H}}{\partial p_j q_j}\right)dV} = 0
\end{equation} \refstepcounter{subsection}


\refstepcounter{section}
\section{Paréntesis de Poisson (II)} \refstepcounter{subsection}

\pagelayout{wide} % No margins
\addpart[Fuerzas centrales y sistemas no inerciales]{\fontsize{55pt}{0pt} \setstretch{2} \textbeuron{Fuerzas centrales sistemas no inerciales}}
\pagelayout{margin} % Restore margins

\chapter{Fuerzas centrales} 
\refstepcounter{subsection}
Llamamos fuerza central a toda fuerza $\mathbf{F}(\mathbf{r})=F(\mathbf{r}) \hat{\mathbf{e}}_r \label{5.0.1} \inlineeqnum$, es decir, que ocurre en dirección radial a un punto determinado, si además esta fuerza central es conservativa, es equivalente a $\mathbf{F}(r)=F(r) \hat{\mathbf{e}}_r \label{5.0.2} \inlineeqnum$, es decir que es esférica simétricamente y solo depende de la distancia al origen, ya que
\[\mathbf{F}=F(r) \hat{\mathbf{e}}_r=-\nabla U(r,\theta,\varphi)=\frac{\partial U}{\partial r}\hat{\mathbf{e}}_r + \frac{1}{r}\frac{\partial U}{\partial \theta}\hat{\mathbf{e}}_\theta+\frac{1}{r\sin\theta}\frac{\partial U}{\partial \varphi}\hat{\mathbf{e}}_\varphi \implies \frac{\partial U}{\partial \theta}=\frac{\partial U}{\partial \varphi}=0\]
puesto que $1/r$ y $1/r\sin\theta$ no pueden ser 0, esto implica que $U=U(r)$ y $F=F(r)$, además llegamos a la siguiente expresión de $F$
\begin{equation} \label{5.0.3}
    F(r)=-\frac{\partial U}{\partial r}
\end{equation} \refstepcounter{subsection}
La recíproca, que $\mathbf{F}(r)=F(r) \hat{\mathbf{e}}_r$ es conservativa se puede obtener calculando su rotacional y verificando que es igual a 0.
\section{Problema de los dos cuerpos} \refstepcounter{subsection}
\begin{marginfigure}[-2cm]
    \def\svgwidth{125 pt}
    \tiny
	\input{images/centralf.pdf_tex}
	\labfig{margin2}
\end{marginfigure}
Si tenemos dos masas $m_1$ y $m_2$ con posiciones $\mathbf{r}_1$ y $\mathbf{r}_2$, de tal forma que sufren cada una una fuerza central conservativa creada por la otra masa, siguiendo la tercera ley de newton, entonces $U=U(r)$, donde $r=|\mathbf{r}_1-\mathbf{r}_2|=|\mathbf{r}| \label{5.1.1} \inlineeqnum$, tal que $\mathbf{r}=\mathbf{r}_1-\mathbf{r}_2 \label{5.1.2} \inlineeqnum$.

Podemos definir también el centro de masas del sistema de ambas masas, que se encuentra necesariamente en un punto intermedio entre ambas masas, y más cercano a la masa mayor
\begin{equation} \label{5.1.3}
    \mathbf{R} = \frac{1}{M}\sum^n{m_i \mathbf{r_i}} = \frac{m_1 \mathbf{r}_1 + m_2 \mathbf{r}_2}{m_1+m_2} \ \ \ \ \ M=\sum^n m_i
\end{equation} \refstepcounter{subsection}
De esta forma podemos hacer el cambio de las coordenadas $(\mathbf{r}_1,\mathbf{r}_2) \mapsto (\mathbf{r},\mathbf{R})$, que podemos invertir despejando $\mathbf{r}_1$ y de (5.1.2) y (5.1.3) e igualando para despejar $\mathbf{r}_2$, después sacamos $\mathbf{r}_1$ de una de las anteriores, tal que
\begin{equation} \label{5.1.4}
    \mathbf{r}_1 = \mathbf{R} + \frac{m_2}{M}\mathbf{r} \ \ \ \ \ \ \mathbf{r}_2 = \mathbf{R} - \frac{m_1}{M}\mathbf{r}
\end{equation} \refstepcounter{subsection}
Ahora podemos escribir $\pazocal{L}$ del sistema, para la energía cinética, veremos que los términos cruzados se cancelan
\begin{equation} \label{5.1.5}
    T=\frac{1}{2}m_1(\dot{\mathbf{r}}_1)^2+\frac{1}{2}m_2(\dot{\mathbf{r}}_2)^2=\frac{1}{2}M(\dot{\mathbf{R}})^2 + \frac{1}{2}\mu(\dot{\mathbf{r}})^2 \ \ \ \ \ 
    \boxed{\mu = \frac{m_1 m_2}{m_1+m_2}} \ \ \ \ \ \ \ U = U(r)
\end{equation} \refstepcounter{subsection}
\begin{equation} \label{5.1.6}
    \pazocal{L} = \pazocal{L}_{CM} + \pazocal{L}_{\mbox{\small rel}}= \left(\frac{1}{2}M(\dot{\mathbf{R}})^2\right) + \left(\frac{1}{2}\mu(\dot{\mathbf{r}})^2-U(r)\right)
\end{equation} \refstepcounter{subsection}
Es de notar que cuando la diferencia en las masas es muy grande, la masa reducida, $\mu$ tiende a la masa más pequeña.

De la ecuación (5.1.6) podemos concluir usando (E-L) que el momento asociado a $\mathbf{R}$ se conserva, puesto que que $\pazocal{L}$ no depende explícitamente de $\mathbf{R}$, entonces podemos llegar a tres ecuaciones resumidas en $M\ddot{\mathbf{R}}=0 \label{5.1.7} \inlineeqnum$, que indican que la velocidad del CM es constante.

Para el movimiento relativo en $\mathbf{r}$, aplicando (E-L), podemos llegar a tres ecuaciones que resuminos en $\mu\ddot{\mathbf{r}}=-\nabla U \label{5.1.8} \inlineeqnum$.

Entonces por (5.1.7), el sistema de referencia relativo al CM es un sistema inercial, de tal forma que estableciendo $\mathbf{R}=0$, podemos obtener las expresiones de $\mathbf{r}_1$ y $\mathbf{r}_2$ en el sistema del CM.
\begin{equation} \label{5.1.9}
    \mathbf{r}_1=\frac{m_2}{M}\mathbf{r} \ \ \ \ \ \mathbf{r}_2=-\frac{m_1}{M}\mathbf{r}
\end{equation} \refstepcounter{subsection}
Observando el dibujo de la página anterior, esta claro que en el sistema del CM, las posiciones de ambas masas deben estar en el mismo eje, es decir, sus vectores de posición son paralelos, puesto que $\mathbf{R}$ se encuentra siempre entre la recta que une a ambas masas.

Hay que tener cuidado por que $\mathbf{r}$ no es un vector posición, sino como definimos en (5.1.2), es la diferencia entre los dos vectores de posición.
\section{Conservación del momento angular}  \refstepcounter{subsection}
Definimos el momento angular total con respecto a O como $\mathbf{J} = \mathbf{J}_1 + \mathbf{J}_2 \label{5.1.10} \inlineeqnum$, donde $\mathbf{J}_i = \mathbf{r}_i \times \mathbf{p}_i = m_i \mathbf{r}_i \times \dot{\mathbf{r}}_i \label{5.1.11} \inlineeqnum$. La derivada del momento angular será entonces
\begin{equation} \label{5.1.12}
    \dot{\mathbf{J}_i}= m \left(\dot{\mathbf{r}}_i \times \dot{\mathbf{r}}_i + \mathbf{r}_i \times \ddot{\mathbf{r}}_i \right) = m \mathbf{r}_i \times \mathbf{r}_i = \mathbf{r}_i \times \mathbf{F}_i
\end{equation} \refstepcounter{subsection}
Entonces, usando la 3ª LN, (5.1.2) y (5.0.1), el momento angular total se conserva.
\begin{equation} \label{5.1.13}
    \dot{\mathbf{J}}=\mathbf{r}_1 \times \mathbf{F}_{12}+\mathbf{r}_2 \times \mathbf{F}_{21}=(\mathbf{r}_1-\mathbf{r}_2)\times \mathbf{F}=F \mathbf{r} \times \hat{\mathbf{u}}_r = 0
\end{equation} \refstepcounter{subsection}
El momento angular total en el sistema del CM es entonces, usando (5.1.9)
\begin{equation} \label{5.1.14}
    \mathbf{J} = \frac{m_1 m_2^2}{M^2} (\mathbf{r}\times \dot{\mathbf{r}})+ \frac{m_2 m_1^2}{M^2} (\mathbf{r}\times \dot{\mathbf{r}}) = \mu (\mathbf{r}\times \dot{\mathbf{r}})
\end{equation} \refstepcounter{subsection}
Como este se conserva puesto que sigue siendo inercial, esto implica que el movimiento de ambas masas debe ocurrir en un plano*, el perpendicular a $\mathbf{J}$.

Entonces podemos expresar la configuración del sistema con coordenadas polares, puesto que tenemos dos grados de libertad. Expresando el lagrangiano del sistema en coordenadas polares usando (5.1.6) y $\dot{\mathbf{r}}=d{(r \hat{\mathbf{u}}_r)}/dt=\dot{r}\hat{\mathbf{u}}_r + r \dot{\varphi}\hat{\mathbf{u}}_\varphi$ tenemos
\begin{equation} \label{5.1.15}
    \pazocal{L} = \frac{1}{2}\mu(\dot{\mathbf{r}})^2-U(r) = \frac{1}{2}\mu\left(\dot{r}^2+r^2\dot{\varphi}^2\right)-U(r)
\end{equation} \refstepcounter{subsection}
Vemos que entonces $\varphi$ es ignorable pues no aparece explícitamente y entonces su momento se conserva
\begin{equation} \label{5.1.16}
    p_\varphi = J = \frac{\partial \pazocal{L}}{\partial \dot{\varphi}} = \mu r^2 \dot{\varphi} \ \ \ \  \dot{p_\varphi} = 0
\end{equation} \refstepcounter{subsection}
Lo cual es exáctamente el módulo de $\mathbf{J}=\mu (r\hat{\mathbf{u}_r}\times (\dot{r}\hat{\mathbf{u}_r} + r \dot{\varphi} \hat{\mathbf{u}_\varphi}))= \mu r^2 \dot{\varphi}\hat{\mathbf{u}_z}$
\vspace{5pt}
\subsection{Esféricas *}
Podemos también demostrar que el movimiento ocurre en un plano escribiento el lagrangiano usando coordenadas esféricas, similar a (5.2.6), donde $\dot{\mathbf{r}}=d{(r \hat{\mathbf{u}}_r)}/dt=\dot{r}\hat{\mathbf{u}}_r + r \dot{\theta }\hat{\mathbf{u}}_\theta + r\sin \theta \dot{\varphi} \hat{\mathbf{u}}_\varphi$, tal que 
\begin{equation} \label{5.1.17}
    \pazocal{L} = \frac{1}{2}\mu(\dot{\mathbf{r}})^2-U(r) = \frac{1}{2}\mu\left(\dot{r}^2+r^2\dot{\theta}^2 + r^2 \sin^2 \theta \dot{\varphi}^2\right)-U(r)
\end{equation} \refstepcounter{subsection}
De esta forma vemos que $\varphi$ es la ignorable, de tal forma que su momento asociado se conservará
\begin{equation} \label{5.1.18}
    p_\varphi = \frac{\partial \pazocal{L}}{\partial \dot{\varphi}} = \mu r^2 \sin^2 \theta \dot{\varphi} \ \ \ \  \dot{p_\varphi} = 0
\end{equation} \refstepcounter{subsection}
Si ahora consideramos $\mathbf{J}$ en estas coordenadas usando (5.2.5)
\begin{equation} \label{5.1.19}
    \mathbf{J} = \mu (\mathbf{r} \times (\dot{r}\hat{\mathbf{u}}_r + r \dot{\theta}\hat{\mathbf{u}}_\theta + r\sin \theta \dot{\varphi} \hat{\mathbf{u}}_\varphi))=-\mu r^2 \sin \theta \dot{\varphi} \hat{\mathbf{u}}_\theta + \mu r^2 \dot{\theta}\hat{\mathbf{u}}_\varphi
\end{equation} \refstepcounter{subsection}
Como $\mathbf{J}$ se conserva, sus componentes se conservan, y entonces usando (5.2.10) en (5.2.9), verificamos que el movimiento ocurre en un plano, donde $\theta$ es constante.
\begin{equation} \label{5.1.20}
    p_\varphi = \mu r^2 \sin^2 \theta \dot{\varphi} = J_\theta \sin\theta \implies \frac{d}{dt} \sin\theta = \dot{\theta} \cos \theta = 0 \implies \dot{\theta} = 0
\end{equation} \refstepcounter{subsection}
\vspace{-30pt}
\subsubsection{Velocidad areolar}
Si consideramos el área que barre $\mathbf{r}$ en un pequeño incremento del tiempo como si fuera un triángulo, tenemos que, donde el primer término es la base del triángulo y el segundo la altura del triángulo, vemos que esta cantidad se conserva.
\begin{equation} \label{5.1.21}
    dA = \frac{1}{2} r\cdot r\dot{\varphi} dt \ \ \ \ \ \dot{A} = \frac{J}{2\mu} \ \ \ \ \ddot{A} = 0
\end{equation} \refstepcounter{subsection}
Esta ecuación se conoce como la segunda ley de \textit{Kepler}.
\section{Energía} \refstepcounter{subsection}
La energía (conservada e igual a $\pazocal{H}$), es, por (5.2.6) y (5.2.7)
\begin{equation} \label{5.1.22}
    E = T+U = \frac{1}{2}\mu\left(\dot{r}^2+r^2\dot{\varphi}^2\right)+U(r)=\frac{1}{2}\mu\dot{r}^2 + \frac{J^2}{2\mu r^2}+U(r)
\end{equation} \refstepcounter{subsection}
La ecuación (5.3.1) es una EDO de primer orden separable que nos permite hallar $r(t)$, invirtiendo la siguiente expresión
\begin{equation} \label{5.1.23}
    \int_{r_0}^r{\left[\frac{2}{\mu}\left(E-U(r)\right)-\frac{J^2}{\mu^2 r^2}\right]^{-\frac{1}{2}}dr}=t-t_0
\end{equation} \refstepcounter{subsection}

Usando (5.2.7) podemos encontrar $\varphi(t)$ una vez tenemos $r(t)$, de forma similar al formalismo Hamiltoniano.
\section{Ecuación del movimiento} \refstepcounter{subsection}
Haciendo (2.2.1)(E-L) con respecto a $r$ obtenemos la ecuación del movimiento del sistema
\begin{equation} \label{5.1.24}
    \mu \ddot{r} = \mu r \dot{\varphi}^2 - \frac{\partial U}{\partial r} = \frac{J^2}{\mu r^3}+ F(r)
\end{equation} \refstepcounter{subsection}

Nos va a interesar encontrar $r(\varphi)$ para no tener una expresión paramétrica de ambos sino la ecuación de una curva, para ello haremos el cambio de variable $u=1/r$.

Usando la regla de la cadena, el teorema de la función inversa y (5.1.16)
\begin{equation} \label{5.1.25}
    \frac{d u}{d\varphi}  = -\frac{1}{r^2} \frac{dr}{d\varphi} = -\frac{1}{r^2} \frac{dr}{dt} \frac{dt}{d\varphi}= -\frac{1}{r^2} \dot{r} \frac{1}{\frac{d\varphi}{dt}}=-\frac{1}{r^2} \dot{r} \frac{1}{\dot{\varphi}}= -\frac{1}{r^2} \dot{r} \frac{r^2 \mu}{J} = -\frac{\mu \dot{r}}{J}
\end{equation} \refstepcounter{subsection}
\begin{equation} \label{5.1.26}
    \frac{d^2 u}{d\varphi^2}  = \frac{d}{d\varphi}\left(-\frac{\mu \dot{r}}{J}\right)=-\frac{\mu}{J} \frac{d \dot{r}}{dt} \frac{dt}{d\varphi} = -\frac{\mu}{J} \frac{d \dot{r}}{dt} \frac{1}{\frac{d\varphi}{dt}} = -\frac{\mu}{J} \ddot{r} \frac{1}{\dot{\varphi}}= - \frac{\mu^2}{J^2} r^2 \ddot{r}
\end{equation} \refstepcounter{subsection}
Despejando $\ddot{r}$ de (5.4.3) y sustituyendo en (5.4.1) llegamos a la ecuación de la trayectoria, cuya solución es $r(\varphi)$
\begin{equation} \label{5.1.27}
    \frac{\partial^2 u}{\partial \varphi^2} + u = -\frac{\mu}{J^2 u^2}F(u) \iff \frac{\partial^2}{\partial \varphi^2}\left(\frac{1}{r}\right) + \frac{1}{r} = -\frac{\mu}{J^2}r^2F(r)
\end{equation} \refstepcounter{subsection}
\section{Potencial efectivo} \refstepcounter{subsection}
El primer término de la ecuación (5.4.1) se denomina fuerza centrífuga, a la que podemos asociar un potencial, tal que 
\begin{equation} \label{5.1.28}
    F_{\mbox{\small cf}} = \frac{J}{\mu r^3} = -\frac{\partial U_{\mbox{\small cf}}}{\partial r} \implies U_{\mbox{\small cf}} = \frac{J^2}{2\mu r^2}
\end{equation} \refstepcounter{subsection}
De esta forma las expresiónes (5.4.1) y (5.3.1) nos quedan
\begin{equation} \label{5.1.29}
    \mu \ddot{r} = -\frac{\partial}{\partial r} (U_{\mbox{\small cf}}+U) = -\frac{\partial U_{\mbox{\small ef}}}{\partial r} \ \ \ \ \ \  U_{\mbox{\small ef}} = U(r) + \frac{J^2}{2\mu r^2}
\end{equation} \refstepcounter{subsection}
\begin{equation} \label{5.1.29}
    E = \frac{1}{2}\mu \dot{r}^2 + \left(\frac{J^2}{2\mu r^2}+U(r)\right) = \frac{1}{2}\mu\dot{r}^2 + U_{\mbox{\small ef}}(r)
\end{equation} \refstepcounter{subsection}
El primer término de (5.5.3) lo llamamos el término cinético y siempre es positivo, esto implica necesariamente la siguiente relación que determinará que valores de $r$ podrá tomar el sistema.
\begin{equation} \label{5.1.30}
    E\geq U_{\mbox{\small uf}}(r) \ \ \forall t \ \ \ \ \ \ E = U_{\mbox{\small uf}}(r) \implies \dot{r}=0
\end{equation} \refstepcounter{subsection}


\clearpage
\section{Potenciales $-\gamma/r$}\refstepcounter{subsection}
Si tenemos un potencial de la forma siguiente, entonces el potencial efectivo asociado toma la siguiente expresión representada en la figura.
\begin{equation} \label{5.6.1}
    U(r) = -\frac{\gamma}{r} \ \ \ \ \gamma>0 \ \ \ \ \ \ U_{\mbox{\small ef}} =  \frac{J^2}{2\mu r^2} -\frac{\gamma}{r}
\end{equation} \refstepcounter{subsection}
\begin{figure}[h]
    \includegraphics[width=12cm]{uf.png}
\end{figure}
Aplicando (5.5.4) podemos deducir ciertas propiedades del movimiento.

Si $E>0$, tenemos que la recta corta en un solo punto a $U_{\mbox{\small ef}}$, en ese punto serán iguales y la velocidad radial se anula. Esto nos indica que si r va disminuyendo, su velocidad radial es negativa pero su modulo va aumentando hasta que llega a $r_c$, donde la diferencia entre $E$ y $U_{\mbox{ \small ef}}$ es mayor y alcaza su pico, entonces el modulo de la velocidad radial disminuye hasta que se anula en el punto $r$ donde se cortan, entonces r volverá a aumentar, siendo su velocidad positiva y creciente, hasta alcanzar su pico en $r_c$, tras lo cual la velocidad decrece hasta un valor límite cuanto r tiende a infinito.

En cambio, si $E>0$, esta corta en dos puntos a $U_{ \mbox{\small ef}}$, donde la velocidad radial se anulará, lo que significa que $r$ esta acotado entre esos dos puntos $r_{\mbox{\small min}}$ y $r_{\mbox{\small max}}$, llamados \textit{periápside} y \textit{apoápside} respectivamente, entorno a los cuales oscilará, puesto que fuera de esa región no se cumple (5.5.4).

Estos valores pueden encontrarse igualando (5.6.1) a $E$ y resolviendo para $1/r$ como una ecuación cuadrática, obteniendo
\begin{equation} \label{5.6.2}
    \frac{1}{r}=\frac{\gamma \mu}{J^2}\left(1\pm\sqrt{1+\frac{2J^2 E}{\gamma^2 \mu}}\right)
\end{equation} \refstepcounter{subsection}

Diremos que una de trayectoria es cerrada cuando exista un periodo $\tau$ tal que $r(t+\tau)=r(t)$ y $\varphi(t+\tau)=\varphi(t) +2 \pi k$ para algún $k \in \mathbb{Z}$.

Si $E=U_{\mbox{\small ef}}$, la velocidad radial se anula y $r$ es constante, describiendo una órbita circular de radio $r_c$.

\section{Órbitas de Kepler} \refstepcounter{subsection}
Tenemos de nuevo $U(r)=-\gamma/r$ y $F(r)=-\gamma/r^2$ tal que $\gamma > 0$. Usando la ecuación de la trayectoria (5.4.4), tenemos que $F(u)=-\gamma u^2$, si $u=u(\varphi)$, entonces
\begin{equation} \label{5.7.1}
    u'' + \left(u -\frac{\mu \gamma}{J^2}\right) = 0 = u'' + \omega(\varphi) \rightarrow \omega'' = u'' \implies \omega'' + \omega = 0 
\end{equation} \refstepcounter{subsection}
Haciendo ese cambio de variable hemos encontrado una EDO facil de resolver, tal que , pudiendo escoger $\delta =0$ al escoger los ejes adecuados (el origen de $\varphi$)
\begin{equation} \label{5.7.2}
    \omega = A \cos{\varphi +\delta} \implies u(\varphi) = \omega + \frac{\mu \gamma}{J^2} = A \cos{\varphi} +\frac{\mu \gamma}{J^2} = \frac{\mu \gamma}{J^2}\left(1+ \frac{A J^2}{\mu \gamma} \cos(\varphi)\right)
\end{equation} \refstepcounter{subsection}
Renombrando ciertas constantes, usando que $A\geq0$ y sustituyendo $u$ llegamos a 
\begin{equation} \label{5.7.3}
    \frac{1}{c} = \frac{\mu \gamma}{J^2}>0 \ \ \ \ \epsilon=\frac{A J^2}{\mu \gamma}\geq 0 \ \ \ \ \ \ r(\varphi) = \frac{c}{1+\epsilon \cos\varphi}
\end{equation} \refstepcounter{subsection}
Veremos que (5.7.3) es la ecuación de las secciones cónicas en coordenadas polares.
\subsection{Caso $0 \leq \epsilon < 1 $}
Si $0 \leq \epsilon < 1 $, entonces el denominador de (5.7.3) nunca se anula, lo que significa que $r$ va a estar acotado con extremos $r_{\mbox{\small min}}$ y $r_{\mbox{\small max}}$ que ocurrirán en $\cos \varphi = \{1,-1\}$
\begin{equation} \label{5.7.4}
    r_{\mbox{\small min}}= \frac{c}{1+\epsilon} \ \ \ \ \ \ r_{\mbox{\small max}} = \frac{c}{1-\epsilon}
\end{equation} \refstepcounter{subsection}
Como el denominador no se anula, $r(\varphi+2\pi k)=r(\varphi)$ para cualquier $k\in \mathbb{Z}$, es decir es periódica en $\varphi$.

Si ahora expresamos (5.7.3) en cartesianas, primero definiendo las transformaciones
\begin{equation} \label{5.1.5}
    x = r\cos\theta \ \ \ \ y = r\sin \theta \ \ \ \ r^2=x^2+y^2
\end{equation} \refstepcounter{subsection}
\vspace{-15pt}
\begin{equation} \label{5.1.6}
    c = r +\epsilon r \cos\varphi = r+\epsilon x \rightarrow r = c-\epsilon x
\end{equation} \refstepcounter{subsection}
\vspace{-20pt}
\begin{equation} \label{5.1.7}
    (c-\epsilon x)^2 = c^2 + \epsilon^2 x^2 -2\epsilon c x = x^2 + y^2 \rightarrow x^2 +2 \frac{c\epsilon}{1-\epsilon^2}x +\frac{y^2}{1-\epsilon^2}=\frac{c^2}{1-\epsilon^2}
\end{equation} \refstepcounter{subsection}
Despejando c de (5.7.3) en (5.7.6), sustituyendo en (5.7.5) y operando llegamos a (5.7.7). Si ahora definimos las siguientes constantes
\begin{equation} \label{5.1.8}
    d = \frac{c\epsilon}{1-\epsilon^2} \ \ \ \ \ b^2 = \frac{c^2}{1-\epsilon^2} \ \ \ \ \ b^2 + d^2 = \frac{c^2}{(1-\epsilon^2)^2} = a^2
\end{equation} \refstepcounter{subsection}
\vspace{-15pt}
\begin{equation} \label{5.1.8}
    b^2 = a^2(1-\epsilon^2) \ \ (b<a) \ \ \ \ \ d=a\epsilon
\end{equation} \refstepcounter{subsection}
Podemos reescribir (5.7.7) y completar el cuadrado de $x$
\begin{equation} \label{5.1.7}
    x^2 +2dx +\frac{y^2}{1-\epsilon^2}=b^2 \rightarrow (x+d)^2 +\frac{y^2}{1-\epsilon^2}=b^2 + d^2 = a^2
\end{equation} \refstepcounter{subsection}
De esta forma pasando $a^2$ dividiendo y usando (5.7.9) obtenemos la ecuación de una elipse
\begin{equation} \label{5.1.7}
    \left(\frac{x+d}{a}\right)^2 +\left(\frac{y}{b}\right)^2=1
\end{equation} \refstepcounter{subsection}
\begin{figure}[H]
    \def\svgwidth{15 cm}
    \normalsize
	\input{images/orbit.pdf_tex}
	\labfig{margin2}
    \vspace{-75pt}
    \caption{Trayectoria $r(\varphi)$}
\end{figure}
\vspace{15pt}
Como se puede apreciar en (5.7.11), el centro de la elipse esta desplazado $d$ unidades hacía la derecha de $O$, la posición de $m_2$. Las constantes $a$ y $b$ son los semiejes mayor y menor respectivamente.

\subsubsection{Primera Ley de \texit{Kepler}}
Es importante notar que no estamos en el sistema del CM, sino en el sistema de $m_2$, aunque si la relación de masas es muy grade ambas posiciones son muy cercanas, de lo contrario siempre podemos usar (5.1.9) para obtener el moviemiento entorno al CM.

$m_2$ se encuentra en uno de los focos de la elipse por estar precisamente una distancia d del centro, esta es la primera ley de \texit{Kepler}.

\subsubsection{Excentricidad}
$\epsilon$ es la excentricidad de la elipse, podemos hallar una expresión de esta en función de $a$ y $b$ usando (5.7.9)
\begin{equation} \label{5.1.12}
    \epsilon = \sqrt{1-\frac{b^2}{a^2}}
\end{equation} \refstepcounter{subsection}
Cuando $a$ y $b$ son iguales tenemos un círculo y su excentricidad es 0, lo que implica que $r_{\mbox{\small min}} = r_{\mbox{\small max}}$.

Tenemos dos nuevas expresiones de los extremos $r_{\mbox{\small min}} = a (1-\epsilon) $ y  $r_{\mbox{\small max}} = a (1+\epsilon)$ usando (5.7.4) y (5.7.8).
\subsubsection{Periodo}
Como vimos en (5.2.12), la velocidad areolar es constante, lo que implica que el área total debe ser igual a la velocidad areolar por el periodo, tal que
\begin{equation} \label{5.1.13}
    a b \pi =A = \frac{J}{2 \mu} \tau
\end{equation} \refstepcounter{subsection}
Usando (5.7.3), (5.7.8) y (5.7.9) llegamos a 
\begin{equation} \label{5.1.14}
    \tau^2 = 4 \pi^2 \frac{\mu^2 a^2 b^2}{J^2} = 4 \pi^2 \frac{\mu^2 a^2 b^2}{c \gamma \mu} =  \frac{4 \pi^2  \mu}{\gamma} a^3
\end{equation} \refstepcounter{subsection}
Que dado $m_2 >> m_1$, entonces $\mu = m_1$ y si $\gamma = G m_1 m_2 $, (5.7.14) se transforma en la tercera ley de \texit{Kepler}.
\begin{equation} \label{5.1.14}
    \tau^2 =\frac{4 \pi^2}{G m_2} a^3
\end{equation} \refstepcounter{subsection}
\subsubsection{Energía}
Como la energía se conserva, podemos relacionar la energía con las constantes que hemos estado definiendo en un punto concreto de la trayectoria y se cumplirá para todos.
Para ello tomamos el caso del apoápside, donde la velocidad radial se anula.
\begin{equation} \label{5.1.14}
    E = \frac{J^2}{2\mu r_{\mbox{\small min}}^2}-\frac{\gamma }{r_{\mbox{\small min}}}
\end{equation} \refstepcounter{subsection}
Despejando $r_{\mbox{\small min}}$ de (5.7.4) y sustituyendo $c$ de (5.7.3) llegamos a 
\begin{equation} \label{5.1.14}
    r_{\mbox{\small min}} = \frac{J^2}{\gamma \mu (1+\epsilon)}
\end{equation} \refstepcounter{subsection}
Sustituyendo en (5.17.16) y operando llegamos a 
\begin{equation} \label{5.1.14}
    E = \frac{\gamma^2 \mu}{2 J^2} (\epsilon^2-1) \ \ \ \ \ \epsilon = \sqrt{1+\frac{2EJ^2}{\gamma^2 \mu}}
\end{equation} \refstepcounter{subsection}
Esta expresión se cumple para cualquier valor de $\epsilon$, lo que nos permite realcionar los valores de $\epsilon$ a las energías y relacionar con lo visto en (5.6), donde por ejemplo (5.6.2) es equivalente a las expresiones encontradas ahora.
\subsection{Caso $ç\epsilon = 1 $}
En este caso, el denominador se anula en $\cos \varphi = -1$, que ocurre cuando $\varphi$ tiende a $\pi$. Si de nuevo expresamos la ecuación (5.3.7) en cartesianas tenemos.
\begin{equation} \label{5.1.14}
    r = \frac{c}{1+\cos\varphi} \rightarrow r+x=c \rightarrow x^2+y^2 = (c-x)^2
\end{equation} \refstepcounter{subsection}
\vspace{-15pt}
\begin{equation} \label{5.1.14}
    y^2 = c^2 -2cx \rightarrow x = \frac{c^2 - y^2}{2c}
\end{equation} \refstepcounter{subsection}
Esta es la ecuación de una parábola en $y$ que se abre hacía la izquierda, cuando $\varphi$ tiende a $\pi$.
\subsection{Caso $ç\epsilon > 1 $}
En este caso, el denominador se anulará cuando $\cos \varphi = -1/\epsilon \label{5.1.7} \inlineeqnum$. Podemos aprovechar las mismas expresiones que en (5.7.11), pero teneiendo en cuenta que en (5.7.8) y (5.7.9), $1-\epsilon < 0$ y $1-\epsilon^2 < 0$, redefiniendo las constantes para que nos queden positivas tenemos
\begin{equation} \label{5.1.14}
    \delta = -d >0 \ \ \ \ \  \beta^2 = -b^2 >0 \ \ \ \ \alpha = -a>0
\end{equation} \refstepcounter{subsection}  
Tal que  (5.7.11) nos queda la ecuación de una hipérbola cuyas asíntotas verifican (5.7.21)
\begin{equation} \label{5.1.7}
    \left(\frac{x+d}{\alpha}\right)^2 -\left(\frac{y}{\beta}\right)^2=1
\end{equation} \refstepcounter{subsection}
\subsection{Cambio de Órbitas}
Vamos a suponer que partimos del periápside de una órbita, le damos un cierto impulso tangencial con $\mu$ y $\gamma$ constante, cambiando la órbita de $r_1(\varphi)$ a $r_2(\varphi)$, teniendo que
\begin{equation} \label{5.1.24}
    r_1(\varphi_0)=r_2(\varphi_0) \rightarrow \frac{c_1}{1+\epsilon_1 \cos(\varphi_0-\delta_1)} = \frac{c_2}{1+\epsilon_2 \cos(\varphi_0-\delta_2)}
\end{equation} \refstepcounter{subsection}
Como partimos del periápside, podemos escoger $\delta_1=0 \label{5.1.25} \inlineeqnum$ y entonces $\varphi_0=0 \label{5.1.25} \inlineeqnum$. Como el impulso es tangencial, es perpendicular a $\mathbf{r}_1$.

Esto solo ocurre para los extremos debido a la geometría de la elipse, esto implica entonces entonces que cuando cambiemos a la nueva órbita, también nos hallaremos en un extremo, pues la velocidad será perpendicular a $\mathbf{r}_2$, así $\delta_2=0 \label{5.1.25} \inlineeqnum$.

El impulso va a cambiar la velocidad de $v_1$ a $v_2$, y llamamos factor de impulso a $\lambda=v_2/v_1$, si $\lambda>1$, entonces la velocidad aumenta, si $\lambda < 1$, la velocidad disminuye.

Como la velocidad es perpendicular a $\mathbf{r}$, eso implica que $J_1=\mu r_1 v_1$ y $J_2=\mu r_2 v_2$, y haciendo despejando $\mu$ e igualando llegamos a $J_2 = \lambda J_1  \label{5.1.25} \inlineeqnum$.

Por otro lado, usando (5.7.3), la expresión de $c$, y (5.7.27), despejamos $\mu \gamma$ e igualamos y obtenemos $c_2= \lambda^2 c_1\label{5.1.25} \inlineeqnum$.

Ahora de (5.7.24) podemos despejar $\epsilon_2$ y susituimos (5.7.25-26-27) y (5.7.29)
\begin{equation} \label{5.1.24}
    \epsilon_2 = \lambda^2 \epsilon_1 + \lambda^2 -1
\end{equation} \refstepcounter{subsection}

Entonces si $\lambda >1$, tenemos que $\epsilon_2>\epsilon_1$, y entonces la órbita es mayor, si  $\lambda <1$, tenemos que $\epsilon_2<\epsilon_1$, y entonces la órbita es menor.
\chapter{Sistemas de referencia no inerciales} 
\refstepcounter{subsection}
Llamemos $\pazocal{S}_0$, al sistema de referencia inercial, que lleva asociado un origen espacial $\pazocal{O}_0 = \mathbf{O}_{\pazocal{S}_0}$, un origen temporal $\pazocal{O}_{t_0}=0_{\pazocal{S}_0}$, cuyos ejes cartesianos se encuentran fijos, con coordenadas $(x_0,y_0,z_0)$. Lo representamos por $\pazocal{S}_0 = \left\{\pazocal{O}_0,\pazocal{O}_{t_0},(\mathbf{e}_{x_0},\mathbf{e}_{y_0},\mathbf{e}_{z_0})\right\}$.

Tenemos otro sistema de referencia $\pazocal{S}$ no inercial, con sus orígenes, $\pazocal{O}$ y $\pazocal{O}_{t}$ que pueden ser iguales o distintos a los de $\pazocal{S}_0$, con sus ejes cartesianos fijos en él mismo, con coordenadas $(x,y,z)$, tal que $\pazocal{S} = \left\{\pazocal{O},\pazocal{O}_{t},(\mathbf{e}_x,\mathbf{e}_y,\mathbf{e}_z)\right\}$.

Definimos formalmente que los ejes estan fijos con respecto a un sistema de referencia cuando 
\begin{equation} \label{6.1.1}
    \left(\frac{d}{dt}\mathbf{e}_{j_0}\right)_{\pazocal{S}_0} = 0 \ \ \ \ \ \left(\frac{d}{dt}\mathbf{e}_j\right)_{\pazocal{S}} = 0 \ \ \ \ \ \forall j
\end{equation} \refstepcounter{subsection}
Entonces un sistema de referencia es no inercial cuando, para algún $j$,
\begin{equation} \label{6.1.1}
    \left(\frac{d}{dt}\mathbf{e}_j\right)_{\pazocal{S}_0} \neq 0 \iff \left(\frac{d}{dt}\mathbf{e}_{j_0}\right)_{\pazocal{S}} \neq 0 
\end{equation} \refstepcounter{subsection}
es decir, alguno de los vectores de $\pazocal{S}$ no es constante respecto a $\pazocal{S}_0$. También es un sistema no inercial en general cuando $\pazocal{O}$ este acelerado con respecto a un $\pazocal{S}_0$.
\section{Aceleración rectilínea}
\refstepcounter{subsection}
Si tenemos que $\pazocal{S}$ esta se mueve de forma rectilínea respecto a $\pazocal{S}_0$ con aceleración $\mathbf{A}=\dot{\mathbf{V}}$, donde $\mathbf{V}$ es la velocidad de $\pazocal{S}$ respecto a $\pazocal{S}_0$.

Si $\mathbf{r}_0$ y $\mathbf{r}$ son vectores de posición equivalentes en cada sistema de referencia, estos se relacionan por
\begin{equation} \label{6.1.1}
    \dot{\mathbf{r}}_0=\dot{\mathbf{r}}+\mathbf{V} \rightarrow \ddot{\mathbf{r}}_0=\ddot{\mathbf{r}}+\mathbf{A}
\end{equation} \refstepcounter{subsection}
Si ahora tenemos la 2LN, llegamos a 
\begin{equation} \label{6.1.1}
    \mathbf{F} = m\ddot{\mathbf{r}}_0 = m\ddot{\mathbf{r}} + m \mathbf{A} \implies m\ddot{\mathbf{r}} = \mathbf{F}-m \mathbf{A} = \mathbf{F}+\mathbf{F}_{\mbox{\small iner.}}
\end{equation} \refstepcounter{subsection}

\pagelayout{wide} % No margins
\addpart[Sólido Rígido]{\fontsize{55pt}{0pt} \setstretch{2} \textbeuron{Solido Rigido}}
\pagelayout{margin} % Restore margins

\chapter{Sólido Rígido} 
\refstepcounter{subsection}
Un sólido rígido es un conjunto de $N$ partículas tales que las distancias entre ellas $r_{\alpha \beta}$ son constantes.

En el espacio, bastan las posiciones de otras tres masas y sus distancias a otra para ubicarla por triangulación, de tal forma que si partimos de un sistema de tres masas no colineales con distancia fija, tenemos 3 ligaduras, lo que implica 6 grados de libertad.

Si ahora añadimos otra masa cuya distancia a las tres anteriores debe ser fija, hemos añadido 3 ligaduras y tres grados de libertad, por que lo que los grados de libertad del sistema no aumentan, y así podemos hacer contínuamente, puesto que solo necesitamos 3 ligaduras cada vez que añadimos una partícula para asegurar que la distancia al resto es fija.

Por lo tanto, un sólido rígido tiene a lo sumo 6 grados de libertad, 3 de ellos relacionados con la posición del centro de masas, y otros 3 de ellos relacionados con la posición rotacional del sólido.
\section{Orientación}
\refstepcounter{subsection}
Vamos a analizar dos sistemas de referencia, ambos con origen en el centro de masas del sólido, uno de ellos, $\pazocal{S}_0 = \left\{\pazocal{O},(\mathbf{e}_x,\mathbf{e}_y,\mathbf{e}_z)\right\}$ no tiene por que ser un sistema inercial, y tenemos $\pazocal{S} = \left\{\pazocal{O},(\mathbf{e}_1,\mathbf{e}_2,\mathbf{e}_3)\right\}$ cuyo movimiento relativo con respecto a $\pazocal{S}_0$ es el de rotación con respecto al centro de masas.

Representamos el cambio de base como un producto matricial
\begin{equation} \label{6.1.1}
    \mathbf{e}'_i = A(t)_{ij} \mathbf{e}_j \ \ \ \ \ A_{ij} = \mathbf{e}'_i \cdot \mathbf{e}_j
\end{equation} \refstepcounter{subsection}
Si ahora hacemos el producto de dos vectores de la base $\pazocal{S}'$, teniendo en cuenta que $\mathbf{e}'_i \cdot \mathbf{e}'_j = \delta_{ij}$, es decir, la base es ortonormal, obtenemos
\begin{equation} \label{6.1.1}
    \mathbf{e}'_i \cdot \mathbf{e}'_j = \left(A_{ik} \mathbf{e}_k\right)\cdot \left(A_{jl} \mathbf{e}_l\right) = A_{ik} A_{jl} \mathbf{e}'_k \cdot \mathbf{e}'_l = A_{ik} A_{jl} \delta_{kl} =  A_{ik} A_{jk} = A_{ik} A^T_{kj} = \delta_{ij}
\end{equation} \refstepcounter{subsection}
Esta expresión nos da las siguientes relaciones entre las entradas de $A$
\begin{equation} \label{6.1.1}
    \sum_k (A_{ik})^2 = 1 \ \ \ i=1,2,3 \ \ \ \ \ \ \ \ A_{ik} A_{jk} = 0 \ \ \ \ i\neq j \rightarrow (1,2), (1,3), (2,3)
\end{equation} \refstepcounter{subsection}
Por lo que tenemos en total 6 relaciones y 9 componentes, lo que resulta en que $A$ solamente tiene 3 componentes independientes. Las relaciones de (8.1.3) lo que nos dicen es que las columnas de $A$ deben ser vectores unitarios y que las columnas entre sí son ortogonales. Es decir, la matriz es ortogonal, de hecho la igualdad final de (8.1.2) es la definición de matriz ortogonal, $AA^T = I$.
\newpage
Para expresar un vector $\mathbf{b}$ de una base a otra tenemos en cuenta que
\begin{equation} \label{6.1.1}
    b_i = \mathbf{b} \cdot \mathbf{e}_i \ \ \ \ \ \ \ b'_i = \mathbf{b} \cdot \mathbf{e}'_i
\end{equation} \refstepcounter{subsection}
\vspace{-20pt}
\begin{equation} \label{6.1.1}
    b'_i = \mathbf{b} \cdot A_{ij} \mathbf{e}_j = A_{ij} \mathbf{b} \cdot  \mathbf{e}_j = A_{ij} b_j
\end{equation} \refstepcounter{subsection}
Las matrices ortogonales se corresponden a rotaciones o simetrías, y tienen la propiedad de que $det(A) = \pm 1$, pero nos vamos a centrar en las transformaciones de determinante positivo, que son rotaciones.
\vspace{-10pt}
\subsection{Ángulos de Euler (I)}
\begin{tikzpicture}[remember picture, overlay]
    \node [shift={(-2.8cm, -12cm)}] at (current page.north east)
        { \normalsize
        \def\svgwidth{4.5 cm}
        \input{images/eangles.pdf_tex} };
\end{tikzpicture}
Podemos describir cualquier orientación en el espacio mediante tres rotaciones con respecto a ciertos planos, tal que $\mathbf{T} = \mathbf{T}_\psi \mathbf{T}_\theta \mathbf{T}_\phi$
\[\mathbf{T}_\phi = \left[\begin{matrix}
    \cos\phi && \sin\phi&& 0\\
    -\sin\phi && \cos\phi && 0\\
    0 && 0 && 1
\end{matrix}\right] \ \ \ \ \ \ \ 
\mathbf{T}_\theta = \left[\begin{matrix}
    \cos\theta && 0 && -\sin\theta \\
    0&& 1&& 0\\
    \sin\theta&& 0&& \cos\theta
\end{matrix}\right]\]
\[\mathbf{T}_\psi = \left[\begin{matrix}
    \cos\psi && \sin\psi&& 0\\
    \sin\psi && \cos\psi && 0\\
    0 && 0 && 1
\end{matrix}\right]\]
\vspace{-15pt}
\section{Tensor de Inercia}
\vspace{-15pt}
\subsection{Tensores}
Consideramos $\mathbb{R}^d$ con base ortonormal, un tensor $\mathbb{T}$ rango $N$ tiene $d^N$ componentes, con índices ordenados $\mathbb{T}_{ijk \dots}$ donde cada letra toma valores $(1,2,3,\dots,d)$. Este se transforma bajo transformaciones ortonormales como
\begin{equation} \label{6.1.1}
    \mathbb{T}'_{ijk \dots} = \sum_{l,m,n,\dots^N} A_{il} A_{jm} A_{kn} \dots \mathbb{T}_{lmn \dots}
\end{equation} \refstepcounter{subsection}
Si $N=0$ no tiene índices y solo una componente, por lo que se trata de un escalar y es invariante ante transformaciones.
Para $N=1$ tenemos un índice y $d$ componentes, por lo que se trata de un vector y (8.1.6) es idéntica a (8.1.5) y es un cambio de base.
Para $N=2$ tenemos dos índices y $d^2$ componentes, y entonces la podemos interpretar como una matriz que puede actuar como una transformación lineal o una forma bilineal, que se transforma como 
\begin{equation} \label{6.1.1}
    \mathbb{T}'_{ij} = \sum_{l,m} A_{il} A_{jm} \mathbb{T}_{lm} = \sum_{l,m} A_{il} \mathbb{T}_{lm} A^T_{mj} \rightarrow  \mathbb{T}' = A \mathbb{T} A^T = A \mathbb{T} A^{-1}
\end{equation} \refstepcounter{subsection}
\vspace{-25pt}
\subsection{Momento angular}
El centro masas de un sólido es
\begin{equation} \label{6.1.1}
    \mathbf{R} = \frac{1}{M} \sum_\alpha^N m_\alpha \mathbf{r}_\alpha = \frac{1}{M} \int_V \rho \mathbf{r} dV \ \ \ \ \ \ M = \sum_\alpha^N m_\alpha=\int_V \rho dV
\end{equation} \refstepcounter{subsection}
Si además $\mathbf{r}_\alpha = \mathbf{R} + \mathbf{r}'_\alpha$, entonces
\begin{equation} \label{6.1.1}
    \mathbf{R} = \frac{1}{M} \sum_\alpha^N m_\alpha \mathbf{r}_\alpha = \mathbf{R} \frac{1}{M} \sum_\alpha^N m_\alpha + \frac{1}{M} \sum_\alpha^N m_\alpha \mathbf{r}'_\alpha \implies \sum_\alpha^N m_\alpha \mathbf{r}'_\alpha = 0
\end{equation} \refstepcounter{subsection}
Y entonces obtenemos las siguientes relaciones
\begin{equation} \label{6.1.1}
    \mathbf{p}_T = M \dot{\mathbf{R}} \ \ \ \ \ \ \mathbf{L}|_O = \mathbf{L}_{\mbox{\small CM}}|_O+\mathbf{L}|_{\mbox{\small CM}}  \ \ \ \ \ \ T = T_{\mbox{\small CM}} + T_{\mbox{\small rel.}}
\end{equation} \refstepcounter{subsection}
\vspace{-30pt}
\subsection{Tensor de Inercia}
Tomamos el sistema de referencia fijo en el sólido, $\pazocal{S}'$, que estará en rotación con respecto a el sistema de referencia inercial, $\pazocal{S}$. El origen es fijo en $\pazocal{S}'$, pero es arbitrario y no es necesario tomar como el origen el CM.

Tenemos una rotación con respecto al origen y a un eje representado por $\vec{\omega}(t)$, entonces el momento ángular con respecto al origen $O$ que hemos elegido es, recordando (6.2.3)
\[
    \mathbf{L} = \sum_\alpha^N \mathbf{r}_\alpha \times m_\alpha \mathbf{v}_\alpha = \sum_\alpha^N m_\alpha \mathbf{r}_\alpha \times (\vec{\omega} \times \mathbf{r}_\alpha) = \sum_\alpha^N m_\alpha \left[\vec{\omega} (\mathbf{r}_\alpha \cdot \mathbf{r}_\alpha) - \mathbf{r}_\alpha (\mathbf{r}_\alpha \cdot \vec{\omega})\right]
\]\vspace{-10pt}
\[
    \mathbf{L} = \sum_\alpha^N m_\alpha \left[\vec{\omega} \mathbf{r}_\alpha^2 - \mathbf{r}_\alpha (\mathbf{r}_\alpha \cdot \vec{\omega})\right] \ \ \ \ \ \ L_i = \sum_\alpha m_\alpha \left[ \sum_{kj} x^2_{\alpha k} \omega_i - x_{\alpha j} \omega_j x_{\alpha i}\right] = 
\]\vspace{-10pt}
\[
    = \sum_\alpha m_\alpha \left[ \sum_{kj} \delta_{ij}x^2_{\alpha k} \omega_j - x_{\alpha j} \omega_j x_{\alpha i}\right] = \sum_j \left(\sum_\alpha m_\alpha \left[\delta_{ij}\sum_k x^2_{\alpha k} - x_{\alpha j} x_{\alpha i}\right]\right) \omega_j
\]
\vspace{-10pt}
\begin{equation} \label{6.1.1}
    I_{ij} = \sum_\alpha m_\alpha \left[\delta_{ij} \sum_k x^2_{\alpha k} - x_\alpha j x_{\alpha i}\right] = \int \left[\delta_{ij}\sum_k x^2_{k} - x_{j} x_{i}\right] \rho dV = I_{ji}
\end{equation} \refstepcounter{subsection}
De esta forma tenemos que
\begin{equation} \label{6.1.1}
    L_i = \sum_j I_{ij} \omega_j \iff \mathbf{L} = \mathbf{I} \vec{\omega}
\end{equation} \refstepcounter{subsection}
$\mathbf{I}$ es un tensor de rango 2 real, y simético y definido positivo. Los elementos de la diagonal se llaman momentos de inercia, y el resto productos de inercia, y el momento de inercia con respecto a una dirección se define como $\hat{n}^T \mathbf{I} \hat{n}$.

\subsubsection{Simetrías}
Si por ejemplo tenemos que una distribución de masa es simétrica bajo una transformación $x_i \mapsto -x_i$, con respecto al plano perpendicular al eje que define $x_i$, de tal forma que $\rho(x_i) = \rho(-x_i)$, haremos una integral de un producto de función par, la densidad de masa, por una función impar, $x_i$, con respecto al intervalo de integración simétrico, entonces el resultado será 0 para todos los productos de inercia donde $I_{ji}$ donde apareza ese índice serán 0.

Si tenemos una simetría axial, tal que $\rho = \rho(r,z)$, es decir, expresada en cilíndicas, no depende del ángulo con respecto al eje. Al igual que en el caso anterior, tendremos una simetría para cualquier plano que contenga al eje y entonces los productos de inercia son 0 y el tensor es diagonal.

Además, si el eje de simetría es $z$, entonces $I_{xx} = I_{yy}$ por que hay simetría axial, y se puede demostrar que las integrales son idénticas por que los intervalos de integración son iguales.
\subsubsection{Teorema de Steiner}
Sea un sólido rígido de masa M, y consideramos dos sistemas de referencia con ejes paralelos y distintos orígenes, $\pazocal{O}$ y $\pazocal{G}$. Sea $\mathbf{a} = \overline{\pazocal{G}\pazocal{O}} = \pazocal{O}-\pazocal{G}$ el vector que relaciona ambos orígenes, tenemos entonces que, donde $\mathbb{E}_3$ es la identidad y $\otimes$ es el producto tensorial
\begin{equation} \label{6.1.1}
    I^\pazocal{G}_{ij} = I^\pazocal{O}_{ij} + M\left(||\mathbf{a}||^2 \delta_{ij} -a_i a_j\right) \iff \mathbf{I}_\pazocal{G} = \mathbf{I}_\pazocal{O} + M\left(||\mathbf{a}||^2 \mathbb{E}_3+\mathbf{a} \otimes \mathbf{a}\right)
\end{equation} \refstepcounter{subsection}
\vspace{-20pt}
\subsubsection{Ejes principales}
Llamamos ejes principales aquellos en los que $\mathbf{L} = \lambda \vec{\omega}$ cuando $\vec{\omega}$ es paralelo al eje.
Siempre existen tres ejes principales, ya que $\mathbf{I}$ es hermítico, entonces siempre existe una base ortonormal de autovectores que verifican la condición anterior.
\section{Dinámica}
\subsection{Energía cinética de rotación}
Tenemos la siguiente expresión de la energía cinética para un sistema de partículas, que ya hemos utilizado previamente
\begin{equation} \label{6.1.1}
    T = \frac{1}{2} \sum_\alpha^N m_\alpha v_\alpha^2 = \frac{1}{2}\int \rho v^2 dV
\end{equation} \refstepcounter{subsection}
Vamos a desarrolar el término de la velocidad al cuadrado, teniendo en cuenta que $\mathbf{v} = \vec{\omega} \times \mathbf{r}$, tal que
\[v^2 =  (\vec{\omega} \times \mathbf{r}) \cdot (\vec{\omega} \times \mathbf{r}) = \sum_{ijklm} \epsilon_{ijk} \epsilon_{ilm} \omega_j \omega_l r_k r_m = \sum_{jklm} (\delta_{jl} \delta_{km}- \delta_{jm} \delta_{kl}) \omega_j \omega_l r_k r_m =\]
\vspace{-10pt}
\begin{equation} \label{6.1.1}
    = \sum_{jk} \omega_j^2 r_k^2 - \sum_{jk}\omega_j \omega_k r_k r_j = \sum_{j} \omega_j \sum_{k} \omega_j |\mathbf{r}|^2 - \omega_k r_k r_j = \sum_{j} \omega_j \sum_{k} \omega_k \left(\delta_{ik}|\mathbf{r}|^2  - r_k r_j\right)
\end{equation} \refstepcounter{subsection}
Entonces, sustituyendo en (8.2.0) tenemos
\begin{equation} \label{6.1.1}
    T = \frac{1}{2} \sum_{jk} \omega_j \omega_k \sum_\alpha^N m_\alpha\left(\delta_{ik}|\mathbf{r}_\alpha|^2 - r_{\alpha k} r_{\alpha j}}\right) = \frac{1}{2} \sum_{jk} \omega_j \omega_k I_{jk} = \frac{1}{2} \sum_{j} \omega_j L_j
\end{equation} \refstepcounter{subsection}
\begin{equation} \label{6.1.1}
    T = \frac{1}{2} \vec{\omega} \cdot \mathbf{L} = \frac{1}{2} \vec{\omega}^T \mathbf{I} \vec{\omega}
\end{equation} \refstepcounter{subsection}
Y en la base de autovectores, donde $\lambda_i$ son los autovalores, tendremos
\begin{equation} \label{6.1.1}
    T = \frac{1}{2} \sum \omega_i^2 \lambda_i \end{equation}\refstepcounter{subsection}
\subsubsection{Precesión de una peonza simétrica}
Tenemos un sistema de referencia inercial, y un sistema de referencia no inercial solidario al trompo, que gira con él. Además los ejes de ese sistema son los ejes principales de la peonza, tal que el momento de inercia es de la forma, pues es simétrica con respecto al un eje
\begin{equation} \label{6.1.1}
    \left[\begin{matrix}
        \lambda_1 && 0 && 0 \\
        0 && \lambda_1 && 0 \\
        0 && 0 && \lambda_3 
    \end{matrix}\right]
\end{equation}\refstepcounter{subsection}
Consideramos el centro de masa de la peonza, con posición $\mathbf{R}$ con respecto al origen de $\pazocal{S}_0$.
Tenemos que la peonza apoya su punta sobre el origen del sistema de referencia inercial, que esta fijo.

Considerando $\vec{\omega} = \omega \mathbf{e}_3$ y $\mathbf{L} = \lambda_3 \omega \mathbf{e}_3$, entonces aplicando la 2 LN, tenemos que, si no consideramos la gravedad
\begin{equation} \label{6.1.1}
    \left(\frac{d\mathbf{L}}{dt}\right)_{\pazocal{S}_0} = \mathbf{\Gamma} = \mathbf{r} \times \mathbf{F} = 0 \implies \mathbf{L} = \mbox{ cte.}
\end{equation}\refstepcounter{subsection}
En cambio, en presencia de gravedad, teniendo que $\mathbf{F}_g = -g \mathbf{e}_z$, entonces tenemos, donde theta es el ángulo que forman $\mathbf{r}$ y $\mathbf{e}_z$
\begin{equation} \label{6.1.1}
    \left(\frac{d\mathbf{L}}{dt}\right)_{\pazocal{S}_0} = \mathbf{\Gamma}_g = \mathbf{r} \times \mathbf{F}_g = gMR \left(\mathbf{e}_z \times \mathbf{e}_3\right); \ \ \ \ |\mathbf{\Gamma}_g| =  gMR \sin\theta 
\end{equation}\refstepcounter{subsection}
Si ahora suponemos que $\mathbf{\Gamma}_g$ es pequeño, es decir $|\mathbf{\Gamma}_g| \ll \lambda_3 \omega^2$ ($\Omega \ll \omega$), $\vec{\omega}$ no cambiará mucho, y entonces 
\begin{equation} \label{6.1.1}
    \left(\frac{d\mathbf{L}}{dt}\right)_{\pazocal{S}_0} \approx \lambda_3 \omega \left(\frac{d\mathbf{e}_3}{dt}\right)_{\pazocal{S}_0} \implies \left(\frac{d\mathbf{e}_3}{dt}\right)_{\pazocal{S}_0} = \frac{gMR}{\lambda_3 \omega} \left(\mathbf{e}_z \times \mathbf{e}_3\right) = \vec{\Omega} \times \mathbf{e}_3
\end{equation}\refstepcounter{subsection}
Es decir, la peonza rotará en torno al eje vertical con velocidad angular $\Omega$ aproximadamente constante, además del giro con respecto a su propio eje.
\subsection{Ecuaciones de Euler}
Usando (6.2.5), podemos llegar a la siguiente expresión, la ecuación de Euler
\begin{equation} \label{6.1.1}
    \left(\frac{d\mathbf{L}}{dt}\right)_{\pazocal{S}_0} = \mathbf{\Gamma} = \left(\frac{d\mathbf{L}}{dt}\right)_{\pazocal{S}} + \omega \times \mathbf{L} = \dot{\mathbf{L}} + \omega \times \mathbf{L}
\end{equation}\refstepcounter{subsection}
que podemos escribir por coordenadas en la base de ejes principales explícitamente usando,
\begin{equation} \label{6.1.1}
    \left(\frac{d \vec{\omega}}{dt}\right)_{\pazocal{S}_0} = \left(\frac{d \vec{\omega}}{dt}\right)_\pazocal{S} + \cancelto{0}{\vec{\omega} \times \vec{\omega}}
\end{equation}\refstepcounter{subsection}
resultando en las ecuaciones de Euler
\begin{equation} \label{6.1.1}
    \left\{\begin{matrix}
        \lambda_1 \dot{\omega}_1 + (\lambda_3-\lambda_2) \omega_3 \omega_2 = \Gamma_1 \\
        \lambda_2 \dot{\omega}_2 + (\lambda_1-\lambda_3) \omega_3 \omega_1 = \Gamma_2\\
        \lambda_1 \dot{\omega}_3 + (\lambda_2-\lambda_1) \omega_1 \omega_2 = \Gamma_3
    \end{matrix}\right.
\end{equation}\refstepcounter{subsection}
\subsubsection{Ejemplos}
Vamos a trabajar en casos en los que $\mathbf{\Gamma} = 0$, de tal forma que tenemos
\begin{equation} \label{6.1.1}
    \left\{\begin{matrix}
        \lambda_1 \dot{\omega}_1 = (\lambda_2-\lambda_3) \omega_3 \omega_2  \\
        \lambda_2 \dot{\omega}_2 = (\lambda_3-\lambda_1) \omega_3 \omega_1 \\
        \lambda_1 \dot{\omega}_3 = (\lambda_1-\lambda_2) \omega_1 \omega_2 
    \end{matrix}\right.
\end{equation}\refstepcounter{subsection}
Vamos a cosiderar un primer caso en el que todos los $\lambda_i$ son distintos, y que tenemos una $\vec{\omega}^0 = \omega^0 \mathbf{e}_3$, tal que $\omega_2 = \omega_1 = 0$, entonces tenemos que $\vec{\omega}$ es constante por (7.3.12).

Ahora vamos a considerar una pequeña perturbación del caso anterior, tal que $\vec{\omega}^0 = \omega_3^0 \mathbf{e}_3 + \epsilon (\omega_1^0 \mathbf{e}_1+\omega_2^0\mathbf{e}_2)$, tal que en (7.3.12) obtenemos
\begin{equation} \label{6.1.1}
    \left\{\begin{matrix}
        \lambda_1 \dot{\omega}_1 = (\lambda_2-\lambda_3) \omega_3^0 \omega_2 \epsilon \\
        \lambda_2 \dot{\omega}_2 = (\lambda_3-\lambda_1) \omega_3^0 \omega_1 \epsilon\\
        \lambda_1 \dot{\omega}_3 = (\lambda_1-\lambda_2) \epsilon^2 \omega_1^0 \omega_2^0 
    \end{matrix}\right.
\end{equation}\refstepcounter{subsection}
A primer orden de $\epsilon$ tenemos que $\omega_3$ es constante, tal que $\omega_3 = \omega_3^0$ y entonces en general $\vec{\omega} = \omega_3 \mathbf{e}_3 + \epsilon (\omega_1(t) \mathbf{e}_1+\omega_2(t)\mathbf{e}_2)$ y su derivada $\dot{\vec{\omega}}= \epsilon(\dot{\omega}_1 \mathbf{e}_1+\dot{\omega}_2\mathbf{e}_2)$, tal que 
\begin{equation} \label{6.1.1}
    \left\{\begin{matrix}
        \dot{\omega}_1 \epsilon = \left(\frac{\lambda_2-\lambda_3}{\lambda_1} \omega_3\right) \omega_2 \epsilon \\
        \dot{\omega}_2 \epsilon = \left(\frac{\lambda_3-\lambda_1}{\lambda_2} \omega_3\right)  \omega_1 \epsilon
    \end{matrix}\right.
\end{equation}\refstepcounter{subsection}
Cancelando los $\epsilon$, derivando una de las ecuaciones y sustituyendo en la otra tenemos 
\begin{equation} \label{6.1.1}
    \ddot{\omega}_1 = -\left[\frac{(\lambda_3-\lambda_2)(\lambda_3-\lambda_1)}{\lambda_1 \lambda_2} \omega_3^2\right] \omega_1 = - \Omega^2 \omega_1 \ \ \ \ \ \ddot{\omega}_2 = - \Omega^2 \omega_2
\end{equation}\refstepcounter{subsection}
De tal forma que cuando $\Omega^2 >0$, las oscilaciones serán estables, esto ocurre cuando $\lambda_3 > \lambda_2$ y $\lambda_3 > \lambda_1$ o $\lambda_3 < \lambda_2$ y $\lambda_3 < \lambda_1$, es decir, cuando $\lambda_3$ es el momento principal mayor o menor. Cuando es el intermedio, $\lambda_1 < \lambda_3 < \lambda_2$ o $\lambda_2 < \lambda_3 < \lambda_1$, tenemos que las oscilaciones no serán inestables.

Si ahora consideramos el mismo caso, pero para un sólido simétrico en el que $\lambda_1 = \lambda_2$, tenemos $\omega_3$ es constante a cualquier orden de $\epsilon$, y las oscilaciones son siempre estables ya que tenemos
\begin{equation} \label{6.1.1}
    \Omega^2 = \frac{(\lambda_3-\lambda_1)^2}{\lambda_1^2} \omega_3^2
\end{equation}\refstepcounter{subsection}
Resolviendo las ecuaciones como un sistema de ecuaciones lineales e imponiendo condiciones iniciales $\omega_1(0)=\omega_0$ y $\omega_2(0)=0$ tenemos que
\[
    \left\{\begin{matrix}
        \dot{\omega}_1  = \left(\frac{\lambda_1-\lambda_3}{\lambda_1} \omega_3\right) \omega_2 = \Omega \omega_2 \\
        \dot{\omega}_2  = -\left(\frac{\lambda_1-\lambda_3}{\lambda_2} \omega_3\right)  \omega_1 = -\Omega \omega_1
    \end{matrix}\right. \rightarrow \left(\begin{matrix}
        \dot{\omega}_1 \\ \dot{\omega}_2
    \end{matrix}\right) = \left(\begin{matrix}
        0 & \Omega \\ -\Omega & 0 
    \end{matrix}\right) \left(\begin{matrix}
        \omega_1 \\ \omega_2
    \end{matrix}\right)
\]
\begin{equation} \label{6.1.1}
    \left(\begin{matrix}
        \omega_1 \\ \omega_2
    \end{matrix}\right) = A \left(\begin{matrix}
        \cos{\Omega t} \\ -\sin{\Omega t}
    \end{matrix}\right) + B \left(\begin{matrix}
        \sin{\Omega t} \\ \cos{\Omega t}
    \end{matrix}\right) \ \ \ \ \omega_2(0) = 0 \implies B=0 \rightarrow \left(\begin{matrix}
        \omega_1 \\ \omega_2
    \end{matrix}\right) = A \left(\begin{matrix}
        \cos{\Omega t} \\ -\sin{\Omega t}
    \end{matrix}\right)
\end{equation}\refstepcounter{subsection}
\begin{equation} \label{6.1.1}
    \vec{\omega} = A(\cos\Omega t \mathbf{e}_1 - \sin\Omega t \mathbf{e}_2) + \omega_3 \mathbf{e}_3
\end{equation}\refstepcounter{subsection}

De tal forma que $\vec{\omega}$, desde el sistema de referencia del sólido, traza un movimiento de precesión circular con sentido horario con respecto a $\mathbf{e}_3$, y lo mismo con $\mathbf{L}$, que además es coplanar a $\vec{\omega}$ tal que
\begin{equation} \label{6.1.1}
    \mathbf{L} = A\lambda_1(\cos\Omega t \mathbf{e}_1 - \sin\Omega t \mathbf{e}_2) + \lambda_3 \omega_3 \mathbf{e}_3
\end{equation}\refstepcounter{subsection}
Como $\mathcal{\Gamma} = 0$, entonces $\mathbf{L}$ es constante en el sistema de referencia inercial, y entonces son $\vec{\omega}$ y $\mathbf{e}_3$ los que precesan en torno a $\mathbf{L}$.
\subsection{Ángulos de Euler (II)}
De los dibujos que pueden verse cuando fueron definidos los ángulos de euler, puede verse que 
\begin{equation} \label{6.1.1}
    \vec{\omega} = \dot{\phi} \mathbf{e}_z + \dot{\theta} \mathbf{e}_2'+\dot{\psi} \mathbf{e}_3
\end{equation}\refstepcounter{subsection}
Vamos a considerar sólidos con simetría axial, entonces $\mathbf{e}_2'$ es un eje principal perfectamente válido, como queremos tenerlo todo en la misma base, lo pondremos en la base del sólido, tal que, donde $\mathbf{e}_1''$ es otro eje principal perfectamente válido
\begin{equation} \label{6.1.1}
    \mathbf{e}_z = \cos\theta \mathbf{e}_3 - \sin\theta \mathbf{e}_1'' \ \ \ \ \ \vec{\omega} = -\dot{\phi}\sin\theta \mathbf{e}_1 + \dot{\theta} \mathbf{e}_2+(\dot{\psi}+\dot{\phi}\cos\theta)\mathbf{e}_3
\end{equation}\refstepcounter{subsection}
Y en consecuencia tenemos que el momento angular puede expresarse como
\begin{equation} \label{6.1.1}
    \mathbf{L} = -\lambda_1\dot{\phi}\sin\theta \mathbf{e}_1 + \lambda_1 \dot{\theta} \mathbf{e}_2+\lambda_3(\dot{\psi}+\dot{\phi}\cos\theta)\mathbf{e}_3
\end{equation}\refstepcounter{subsection}
Así, la energía cinética toma la forma
\begin{equation} \label{6.1.1}
    T = \frac{1}{2}\lambda_1\left(\dot{\phi}^2\sin^2\theta+ \dot{\theta}^2\right)+\frac{1}{2}\lambda_3\left(\dot{\psi}+\dot{\theta}\cos\theta\right)^2
\end{equation}\refstepcounter{subsection}
Por otro lado, podemos expresar $\vec{\omega}$ y $\mathbf{L}$ en términos de la base del sistema inercial, tal que
\begin{equation} \label{6.1.1}
    \vec{\omega} = \left(\begin{matrix}
    \dot{\psi}\sin\theta\cos\phi - \dot{\theta}\sin\phi\\
    \dot{\psi}\sin\theta\sin\phi - \dot{\theta}\cos\phi\\
    \dot{\phi} + \dot{\psi}\cos\theta
    \end{matrix}\right)_{(\mathbf{e}_x,\mathbf{e}_y,\mathbf{e}_z)}
\end{equation}\refstepcounter{subsection}
\begin{equation} \label{6.1.1}
    \mathbf{L} = \left(\begin{matrix}
    \lambda_3(\dot{\psi}+\dot{\phi}\cos\theta)\sin\theta \cos\phi-\lambda_1\dot{\theta}\sin\theta -\lambda_1 \dot{\phi} \sin\theta\cos\theta\cos\phi\\
    \lambda_3(\dot{\psi}+\dot{\phi}\cos\theta)\sin\theta \sin\phi+\lambda_1\dot{\theta}\cos\theta-\lambda_1 \dot{\phi} \sin\theta\cos\theta\sin\theta\\
    \lambda_3(\dot{\psi}+\dot{\phi}\cos\theta)\cos\theta+\lambda_1 \dot{\phi} \sin^2\theta
    \end{matrix}\right)_{(\mathbf{e}_x,\mathbf{e}_y,\mathbf{e}_z)}
\end{equation}\refstepcounter{subsection}
Podemos observar que $L_z = L_3 \cos\theta + \lambda_1 \dot{\phi} \sin^2\theta$, de tal forma que veremos que tanto $L_3$ como $L_z$ se conservan en deteminadas circumstancias y podemos obtener la relación
\begin{equation} \label{6.1.1}
    \dot{\phi}(\theta) = \frac{L_z - L_3 \cos\theta}{\sin^2\theta}
\end{equation}\refstepcounter{subsection}
En esos casos en los que $L_3$ y $L_z$ se conservan, tenemos que en función de los valores de ambas, un sólido simético va a realizar, además de la precesión con respecto a $\mathbf{e}_z$ que ya hemos visto antes, un movimiento de nutación, es decir una oscilación de $\theta$ entre dos valores, aplicando las ecuaciones de Euler-Lagrange.

Si (7.3.26) no se anula, se describe un movimiento sinusoidal, mientras que si se anula en lo límites de $\theta$, describe un movimiento similar al anterior pero con picos o cúspides, y si no se anula en los límites de $\theta$, sino en el intervalo, va a haber momentos de retroceso y va a formar una serie de bucles.
\begin{tikzpicture}[remember picture, overlay]
    \node [shift={(-18.5cm, -22cm)}] at (current page.north east)
        { \normalsize
        \def\svgwidth{4 cm}
        \input{images/peonza.pdf_tex} };
\end{tikzpicture}

\pagelayout{wide} % No margins
\addpart[Oscilaciones y Ondas]{\fontsize{55pt}{0pt} \setstretch{2} \textbeuron{Oscilaciones y Ondas}}
\pagelayout{margin} % Restore margins

\input{chapters/oscilaciones.tex}
\chapter{Modos normales}
\refstepcounter{section}
\refstepcounter{subsection}
Vamos a suponer un caso más general formado por $N$ partículas que intectúan entre sí con un potencial $U$, que siguiendo las mísmas líneas que en (8.0.1) podemos expresar el potencial a pequeñas oscilaciones de la posición de equilibrio sin pérdida de generalidad estableciendo el mínimo y el origen de potencial en el origen de las coordenadas generalizadas $q_j$
\begin{equation} \label{6.1.1}
    U({q_j}) \approx \frac{1}{2}\mathbf{q}^{\mbox{\tiny T}} \mathbb{H}_{\mbox{\tiny U}}(0)\mathbf{q} = \frac{1}{2} \sum_{ij}^N \left.\frac{\partial^2 U}{\partial q_i q_j}\right|_0 q_i q_j \ \ \ \ \nabla_{\mathbf{q}} U \approx \mathbb{K} \mathbf{q} \ \ \ \ \mathbb{K} = \mathbb{H}_{\mbox{\tiny U}}(0)
\end{equation}\refstepcounter{subsection}
Por otro lado, usando el teorema de la energía cinética cuando las coordendas generalizadas no dependen explícitamente del tiempo, y aproximando la matriz a orden constante para tener orden cuadrático en T.
\begin{equation} \label{6.1.1}
     T \approx \sum_{j,k}^s{\left(\sum_{\alpha,i}^{N,d} \frac{1}{2} m_\alpha \left.\frac{\partial x_{\alpha i}}{\partial q_j}\right|_0\left.\frac{\partial x_{\alpha i}}{\partial q_k}\right|_0\right)\dot{q}_j\dot{q}_k} = \frac{1}{2}\dot{\mathbf{q}}^{\mbox{\tiny T}} \mathbb{M}\dot{\mathbf{q}} \ \ \ \ \nabla_{\dot{\mathbf{q}}} T \approx \mathbb{M}\dot{\mathbf{q}}
\end{equation}\refstepcounter{subsection}
Entones aplicando (2.2.1) (E-L) llegamos a la siguiente ecuación que queremos resolver
\begin{equation} \label{6.1.1}
    \mathbb{M}\ddot{\mathbf{q}} = - \mathbb{K} \mathbf{q} \ \ \ \ M_{ij} = \sum_{\alpha,k}^{N,d} \frac{1}{2} m_\alpha \left.\frac{\partial x_{\alpha k}}{\partial q_j}\right|_0\left.\frac{\partial x_{\alpha k2}}{\partial q_k}\right|_0 \ \ \ \ K_{ij} = \left.\frac{\partial^2 U}{\partial q_i q_j}\right|_0
\end{equation}\refstepcounter{subsection}
En la práctica, lo que se hace es calcularte $T$ y $U$, aproximarlas a orden cuadrático de $\dot{\mathbf{q}}$ y $\mathbf{q}$ respectivamente, y sacar las matrices de las formas cuadráticas correspondientes, que serán $\mathbb{M}$ y $\mathbb{K}$, respectivamente. En ocasiones realizar las aproximaciones se vuelve complicado y es mejor sacar directamente $\mathbb{H}_{\mbox{\tiny U}}(0)$ de su definición.

En ocasiones también salen términos lineales en el potencial, lo cual significa que las coordenadas generalizadas que hemos escogido no tienen un mínimo en 0, para arreglarlo podemos utilizar completación de cuadrados en muchas circumstancias y cambiar a unas variables donde sí sea cuadrática, y si eso no funciona siempre se puede calcular el gradiente de $U$ y obtener los puntos donde se anula y centrar ahí las coordenadas.
\vspace{-15pt}
\subsection{Soluciones de la ecuación}
\refstepcounter{subsection}
Para resolver esa ecuación diferencial, hacemos la conjetura $\mathbf{q} = \vec{\xi} e^{i\omega t}$, tal que $\ddot{\mathbf{q}} = -\omega^2 \vec{\xi} e^{i\omega t}$, tal que nos queda la siguiente ecuación de autovalores
\begin{equation} \label{6.1.1}
    (\mathbb{K}-\omega^2 \mathbb{M})\vec{\xi} = 0
\end{equation}\refstepcounter{subsection}
Como se observa en (9.0.3), $\mathbb{M}$ y $\mathbb{K}$ son simétricas y definidas positivas, por lo tanto, por el teorema espectral, sabemos que exite una matriz $\mathbb{O} \mathbb{O}^{\mbox{\tiny T}} = \mathbb{I}$, de tal forma que $\mathbb{M} = \mathbb{O} \mathbb{D}_\mathbb{M} \mathbb{O}^{\mbox{\tiny T}}$, entonces si las entradas de $\mathbb{D}_\mathbb{M}$ son los autovalores positivos $\lambda_i$, podemos escribirla como el de una matriz diagonal cuyas entradas son $\sqrt{\lambda_i}$, $\mathbb{D}_\mathbb{M} = \mathbb{S} \mathbb{S}$, por lo tanto $\mathbb{M} = \mathbb{O} \mathbb{S} \mathbb{S} \mathbb{O}^{\mbox{\tiny T}} = \mathbb{R}^{\mbox{\tiny T}} \mathbb{R}$.

Entonces podemos hacer las siguientes manipulaciones, con $\vec{\zeta} = \mathbb{R}\vec{\xi}$ y $\mathbb{A} = (\mathbb{R}^{\mbox{\tiny T}})^{-1}\mathbb{K}\mathbb{R}^{-1}$, para llegar a 
\begin{equation} \label{6.1.1}
    (\mathbb{R}^{\mbox{\tiny T}})^{-1}(\mathbb{K}-\omega^2 \mathbb{M})\mathbb{R}^{-1} \mathbb{R}\vec{\xi} = (\mathbb{A}-\omega^2 \mathbb{I}) \vec{\zeta}= 0
\end{equation}\refstepcounter{subsection}
De esta forma, usando el hecho de que la operación inversa y transpueta conmutan, $\mathbb{A}^{\mbox{\tiny T}} = ((\mathbb{R}^{\mbox{\tiny T}})^{-1}\mathbb{K}\mathbb{R}^{-1})^{\mbox{\tiny T}}= (\mathbb{R}^{\mbox{\tiny T}})^{-1}\mathbb{K}^{\mbox{\tiny T}} \mathbb{R}^{-1}$ es simétrica, puesto que $\mathbb{K}$ lo es. Por lo tanto, por el teorema espectral existe una matriz $\mathbb{Z}\mathbb{Z}^{\mbox{\tiny T}} = \mathbb{I}$ tal que $\mathbb{Z}^{\mbox{\tiny T}} \mathbb{A} \mathbb{Z} = \mathbb{Z}^{\mbox{\tiny T}} (\mathbb{R}^{\mbox{\tiny T}})^{-1}\mathbb{K}\mathbb{R}^{-1} \mathbb{Z}=\mathbb{X}^{\mbox{\tiny T}} \mathbb{K} \mathbb{X} = \mathbb{D}$.

Se observa que $\mathbb{X} \mathbb{M} \mathbb{X}^{\mbox{\tiny T}} =\mathbb{Z} (\mathbb{R}^{\mbox{\tiny T}})^{-1}\mathbb{R}^{\mbox{\tiny T}} \mathbb{R}\mathbb{R}^{-1} \mathbb{Z}^{\mbox{\tiny T}} =\mathbb{I}$ y que $\mathbb{X}\mathbb{X}^{\mbox{\tiny T}} = \mathbb{Z} (\mathbb{R}^{\mbox{\tiny T}})^{-1} \mathbb{R}^{-1} \mathbb{Z}^{\mbox{\tiny T}} = \mathbb{Z} \mathbb{S}^{-1} \mathbb{O}^{-1} (\mathbb{O}^{\mbox{\tiny T}})^{-1} \mathbb{S}^{-1} \mathbb{Z}^{\mbox{\tiny T}} =  \mathbb{Z} (\mathbb{S}^{-1})^2 \mathbb{Z}^{\mbox{\tiny T}} = \mathbb{Z} \mathbb{D}_\mathbb{M}^{-1} \mathbb{Z}^{\mbox{\tiny T}}\neq \mathbb{I}$, por lo tanto, como se verá a continuación, los vectores $\vec{\xi}$ en general no van a ser ortogonales respecto a la identidad, sino respecto al la forma bilineal simétrica definida por $\mathbb{M}$.

Tenemos que $\vec{\xi} = \mathbb{R}^{-1} \vec{\zeta}$ es equivalente a $\mathbb{X} = \mathbb{R}^{-1} \mathbb{Z}$, puesto que $\mathbb{X}$ y $\mathbb{Z}$ son las matrices que contienen a los vectores $\vec{\xi}$ y $\vec{\zeta}$, respectivamente, como columnas. De esta forma, aplicando varias veces el torema espectral, hemos demostrado que la ecuación siempre tiene soluciones y hemos encontrado una expresión explícita para ellas.

Otra forma de ver todo esto que hemos hecho es aplicar directamente el Teorema espectral para $(\mathbb{M}^{-1}\mathbb{K}-\omega^2 \mathbb{I})\vec{\xi} = 0$. Un operador, en este caso $\mathbb{P} = \mathbb{M}^{-1}\mathbb{K}$, es autoadjunto con respecto a una forma bilineal expresada en una misma base, en este caso $\mathbb{M}$, si $\mathbb{P} = \mathbb{M}^{-1} \mathbb{P}^T \mathbb{M} = \mathbb{M}^{-1} (\mathbb{M}^{-1}\mathbb{K})^T \mathbb{M}= \mathbb{M}^{-1} \mathbb{K}^T (\mathbb{M}^{-1})^T\mathbb{M} = \mathbb{M}^{-1} \mathbb{K} \mathbb{M}^{-1}\mathbb{M} = \mathbb{P}$, por lo tanto el Teorema espectral se aplica y existe una matriz $\mathbb{X}$ que cumple que $\mathbb{X}^T \mathbb{M}\mathbb{X} = \mathbb{I}$ tal que $\mathbb{X}^{-1} \mathbb{P} \mathbb{X} = \mathbb{D}$, de tal forma que es diagonalizable y $\mathbb{X}$ es un cambio de base ortogonal de la forma bilineal desde $\mathbb{M}$ a $\mathbb{I}$.

$\mathbb{X}$ es la matriz de cambio de base de la base que diagonaliza a $\mathbb{K}$ hasta la base canónica de $\mathbf{q}$, cuyas columnas estan formadas por los autovectores de la ecuación (9.0.4), de tal forma que si expresamos $\mathbf{\mathbf{q}} = \mathbb{X} \vec{\varsigma}$ entonces la ecuación original (9.0.3) nos queda 
\begin{equation} \label{6.1.1}
    \mathbb{M}\mathbb{X} \ddot{\vec{\varsigma}} + \mathbb{K} \mathbb{X} \vec{\varsigma} = \mathbb{X}^{\mbox{\tiny T}}\mathbb{M}\mathbb{X} \ddot{\vec{\varsigma}} + \mathbb{X}^{\mbox{\tiny T}}\mathbb{K} \mathbb{X} \vec{\varsigma} = \mathbb{I}\ddot{\vec{\varsigma}} + \mathbb{D}\vec{\varsigma} = 0 \iff \ddot{\varsigma_i} + \omega_i^2 \varsigma_i = 0
\end{equation}\refstepcounter{subsection}
Las soluciones de cada una de esas ecuaciones, ya en forma real, son $\varsigma_i = A_i \cos{(\omega_i t -\delta_i)} = A_i \cos{(\omega_i t)}+B_i \sin{(\omega_i t)}$, dónde $A_i$ y $\delta_i$ dependen de las condiciones iniciales, de tal forma que la solución final es $\mathbf{q} = \sum_i^s A_i \vec{\xi}_i \cos{(\omega_i t -\delta_i)}$, que coincide con tomar la parte real de una combinación lineal arbitraria de las conjeturas $\mathbf{q} = \vec{\xi} e^{i\omega t}$.

Hay que tener en cuenta que si obtenemos $\omega_i = 0$ en alguno de los autovalores, eso significa que los movimientos asociados a ese autovalor mantienen al potencial $U$ en un mínimo, además la solución de $\ddot{\varsigma_i}= 0$ es distinta a las otras, es $\varsigma_i = A+Bt$.


\vspace{-15pt}
\subsection{Energía}
\refstepcounter{subsection}
Utilizando las expresiones de (9.0.1) y (9.0.2) tenemos que 
\begin{equation} \label{6.1.1}
    T = \frac{1}{2} \dot{\mathbf{q}}^{\mbox{\tiny T}}\mathbb{M} \dot{\mathbf{q}} = \frac{1}{2} \dot{\vec{\varsigma}}^{\mbox{\tiny T}} \mathbb{X}^{\mbox{\tiny T}} \mathbb{M} \mathbb{X}\dot{\vec{\varsigma}} =\frac{1}{2} \dot{\vec{\varsigma}}^{\mbox{\tiny T}} \mathbb{I}\dot{\vec{\varsigma}} = \frac{1}{2} \sum_i^s \dot{\varsigma}_i^2
\end{equation}\refstepcounter{subsection}
\vspace{-15pt}
\begin{equation} \label{6.1.1}
    U = \frac{1}{2} \mathbf{q}^{\mbox{\tiny T}}\mathbb{K} \mathbf{q} = \frac{1}{2} \vec{\varsigma}^{\mbox{\tiny T}} \mathbb{X}^{\mbox{\tiny T}} \mathbb{K} \mathbb{X}\vec{\varsigma} = \frac{1}{2} \vec{\varsigma}^{\mbox{\tiny T}} \mathbb{D}\vec{\varsigma} =\frac{1}{2} \sum_i^s \omega_i^2\varsigma_i^2
\end{equation}\refstepcounter{subsection}
\vspace{-15pt}
\begin{equation} \label{6.1.1}
    E = T+U = \frac{1}{2}\sum_i^2 \left(\dot{\varsigma}_i^2 + \omega_i^2\varsigma_i^2\right)
\end{equation}\refstepcounter{subsection}
\newpage
\subsection{Péndulo Doble}
\refstepcounter{subsection}
Se va a hacer como ejemplo el péndulo doble para pequeñas oscilaciones en torno a la posición de equilibrio para dos masas iguales y longitudes de los péndulos iguales, por simplificar.
\[
    \begin{matrix}
        x_1 = l \sin\theta_1 && x_2 = l (\sin\theta_1 + \sin\theta_2) \\
        y_1 = l \cos\theta_1 && y_2 = l (\cos\theta_1 + \cos\theta_2)
    \end{matrix}
\]
\[
    T = \frac{1}{2} m \left(\dot{x}_1^2+\dot{y}_1^2+\dot{x}_2^2+\dot{y}_2^2\right) = \frac{1}{2} m l^2 \left(2\cos^2\theta_1 \dot{\theta}_1^2 +\cos^2\theta_2 \dot{\theta}_2^2 + \right.
\]\[
    \left.+2\cos\theta_1\cos\theta_2 \dot{\theta}_1 \dot{\theta}_2 + 2\sin^2\theta_1 \dot{\theta}_1^2 +\cos^2\theta_2 \dot{\theta}_2 - 2\sin\theta_1\cos\theta_2 \dot{\theta}_1 \dot{\theta}_2\right)  \approx  
\]\[
    \approx \frac{1}{2} m l^2 \left(2 \dot{\theta}_1^2 + \dot{\theta}_2^2 + 2 \dot{\theta}_1 \dot{\theta}_2\right) \implies \mathbb{M} = ml^2 \left[\begin{matrix}
        2 && 1 \\ 1 && 1
    \end{matrix}\right]
\]
\[U = -mg (y_1+y_2) = -mgl (2 \cos\theta_1+\cos\theta_2) \approx \frac{1}{2}mgl\left(2 \theta_1^2 + \theta_2^2\right)+ U_0 \implies \mathbb{K} = mgl \left[\begin{matrix}
    2 && 0 \\ 0 && 1
\end{matrix}\right]\]
Ahora resolvemos $\det(\mathbb{K}-\omega^2 \mathbb{M}) = 0$
\[\det\left[\begin{matrix}
    2g/l-2\omega^2 && -\omega^2 \\ -\omega^2 && g/l-\omega^2
\end{matrix}\right] = 0 \implies (\omega^2)^2=(2g/l-2\omega^2)(g/l-\omega^2)\]
Esta ecuación es bicuadrada y las soluciones positivas que obtenemos son
\[\omega_1 = \sqrt{\frac{g}{l}(2-\sqrt{2})} \ \ \ \ \ \omega_2 = \sqrt{\frac{g}{l}(2+\sqrt{2})}\]
Ahora obtenemos los autovectores
\[\left[\begin{matrix}
    2g/l-2g/l(2-\sqrt{2}) && -2g/l(2-\sqrt{2}) \\ -2g/l(2-\sqrt{2}) && g/l-2g/l(2-\sqrt{2})
\end{matrix}\right] \left(\begin{matrix}
    \xi_{11} \\ \xi_{21}
\end{matrix}\right) = 0 \implies \xi_{21} = \sqrt{2}\xi_{11}\]
\[\left[\begin{matrix}
    2g/l-2g/l(2+\sqrt{2}) && -2g/l(2-\sqrt{2}) \\ -2g/l(2+\sqrt{2}) && g/l-2g/l(2-\sqrt{2})
\end{matrix}\right] \left(\begin{matrix}
    \xi_{12} \\ \xi_{22}
\end{matrix}\right) = 0 \implies \xi_{22} = -\sqrt{2}\xi_{21}\]
Entonces, la solución será
\[\left(\begin{matrix}
    \theta_1 \\ \theta_2
\end{matrix}\right) = A_1 \left(\begin{matrix}
    1 \\ \sqrt{2}
\end{matrix}\right) \cos{(\omega_1 t -\delta_1)}+A_2 \left(\begin{matrix}
    1 \\ -\sqrt{2}
\end{matrix}\right) \cos{(\omega_2 t -\delta_2)}\]

\chapter{Ondas}
\refstepcounter{subsection}
\section{Cueda discreta infinita}
Vamos a tener un monton de masas unidas por cuerdas que crean tensión, supondremos que solo hay movimiento vertical de las másas y que para $y_n = 0$ sufren una tensión $\tau$ que mantiene al sistema en equilibrio. A su vez, vamos a asumir que las perturbaciones son pequeñas y que lo ángulos con la horizontal también.

De esta forma, la fuerza que sufre una de las masas es 
\begin{equation} \label{6.1.1}
    m \frac{d^2 y_n}{dt^2} = \tau_n \sin\theta_n + \tau_{n+1} \sin\theta_{n+1}
\end{equation}\refstepcounter{subsection}
Dónde $\tau_n$ es la tensión que genera la cuerda izquierda, $\theta_n$ es el ángulo que forma la cuerda izquierda por la horizontal que pasa por $y_n$, y de forma similar para $n+1$. En general $\tau_n = \tau_n (y_{n-1},y_n)$, pero expandiendo por Taylor, vamos a asumir que la variación es pequeña, entonces nos quedan términos de orden cuadráticos que despreciatemos, de tal forma que $\tau_n \approx \tau$.

Por otro lado, como vamos a asumir que las perturbaciones de ángulos, tal que $\sin\theta_n \approx \tan\theta_n = (y_{n-1}-y_n)/\delta$ y $\sin\theta_{n+1} \approx \tan\theta_{n+1} = (y_{n+1}-y_n)/\delta$, dónde $\delta$ es la distancia horizontal entre masas.

Así, tenemos finalmente una fuerza por parícula, y una energía potencial del sistema
\begin{equation} \label{6.1.1}
    \frac{d^2 y_n}{dt^2} \approx \frac{\tau}{m \delta} \left(y_{n+1}-2y_n+y_{n-1}\right) \ \ \ \ U = \frac{1}{2} \sum_n^\infty \frac{\tau}{\delta} (y_{n+1}-y_n)^2
\end{equation}\refstepcounter{subsection}
% La ecuación (10.2.1) puede expresarse como un conjunto infinito de osciladores acoplados
% \begin{equation} \label{6.1.1}
%     \mathbb{M} \mathbf{y}-\mathbb{I}\ddot{\mathbf{y}}=0 \ \ \ \ \mathbb{M} = \frac{\tau}{m \delta} \left[\begin{matrix}
%         \ddots & \vdots & \vdots  &  \vdots & \vdots & \vdots & \ddots\\
%         \hdots & 1 & -2 & 1 & 0 & 0 & \hdots \\
%         \hdots & 0 & 1 & -2 & 1 & 0 & \hdots \\
%         \hdots & 0 & 0 & 1 & -2 & 1 & \hdots \\
%         \ddots & \vdots & \vdots & \vdots & \vdots & \vdots & \ddots \\
%     \end{matrix}\right]
% \end{equation}\refstepcounter{subsection}
% Si proponemos, como en modos normales, la conjetura $\mathbf{y} = \vec{\xi} e^{i\omega t}$, entonces obtenemos la siguiente ecuación de autovalores en un espacio de dimensión infinita.
% \begin{equation} \label{6.1.1}
%     (\mathbb{M} +\omega^2\mathbb{I})\vec{\xi}=0
% \end{equation}\refstepcounter{subsection}
% Pero $\mathbb{M} +\omega^2\mathbb{I}$ es siempre una matriz tridiagonal no nula, por lo que su núcleo (el conjunto de soluciones de (10.2.3)) es el conjunto vacío, por lo tanto la conjetura que hemos usado no nos sirve para resolver la ecuación.
Las soluciones de la ecuación son de la forma $y_n = A e^{i(\omega t - k n \delta)}$, que sustituyendo en (10.2.1) llegamos a la relación de dispersión
\[
    -\omega^2 A e^{i(\omega t - k n \delta)}= A\frac{\tau}{m \delta} e^{i(\omega t - k n \delta)} \left(e^{-ik\delta}+e^{ik\delta}-2\right) = 2A\frac{\tau}{m \delta} e^{i(\omega t - k n \delta)} \left(\cos(k \delta) - 1\right)
\]\[
    -\omega^2 = \frac{2 \tau}{m \delta}\left(\cos(k \delta) - 1\right) = -4 \frac{\tau}{m \delta} \sin^2{\left(\frac{k\delta}{2}\right)}
\]
\begin{equation} \label{6.1.1}
    \omega (k) = 2 \sqrt{\frac{\tau}{m\delta}} \left|\sin{\left(\frac{k\delta}{2}\right)\right|
\end{equation}\refstepcounter{subsection}

Hemos obtenido por tanto como solución una onda dispersiva discreta. Como se verá mas adelante, la solución general va a venir dada en forma de Transformadas de Fourier (discreta en este caso), y como también veremos, en el caso de que el sistema este confinado, en forma de Serie de Fourier.

\newpage
\subsection{Paso al contínuo}
\refstepcounter{subsection}
Ahora vamos a tomar el límite cuando $\lambda = 2\pi / k \gg \delta$, que es equivalente a decir que las distancias entre las masas son pequeñas, y que $k\delta \ll 1$, por lo tanto $\sin{\left(k \delta/2\right)} \approx k \delta/2$ por lo tanto la relación de dispersión nos queda 
\begin{equation} \label{6.1.1}
    \omega (k) \approx  \sqrt{\frac{\tau}{m/\delta}} k \ \ \ \ \frac{m}{\delta} = \mu
\end{equation}\refstepcounter{subsection}
Por otro lado, si tomamos los mismos limites para (10.2.1), cambiando la notación a $y_{n} (t) = y(x,t)$, $y_{n+1}(t) = y(x+\delta,t)$, $y_{n-1}(t) = y(x-\delta,t)$, y expandiendo los dos últimos en serie de Taylor tenemos
\[\frac{\partial^2 y}{\partial t^2} \approx \frac{\tau}{m/\delta} \frac{1}{\delta^2}\left(y(x,t)+\dot{y}(x,t)\delta + \frac{1}{2}\ddot{y} \delta^2-2y(x,t)+y(x,t)-\dot{y}(x,t)\delta + \frac{1}{2}\ddot{y} \delta^2+ O(\delta^3)\right)\]
Entonces quitando los términos que se cancelan, metiendo el $1/\delta^2$ y al tomar límites nos quitamos los términos de orden $\delta$ o mayor llegamos a la ecuación de ondas no dispersivas.
\begin{equation} \label{6.1.1}
    \frac{\partial^2 y}{\partial t^2} = \frac{\tau}{\mu} \frac{\partial^2 y}{\partial x^2} \ \ \ \ v = \sqrt{\frac{\tau}{\mu}}
\end{equation}\refstepcounter{subsection}
También podemos tomar el límite para la energía potencial de (10.1.1), tal que
\[ U = \frac{1}{2} \sum_n^\infty \tau \left(\frac{y(x+\delta,t)-y(x,t)}{\delta}\right)^2 \delta \rightarrow U = \int_{-\infty}^\infty \frac{1}{2} \tau \left(\frac{\partial y}{\partial x}\right)^2 dx\]
Podemos hacer algo similar para la energía cinética
\[ T = \frac{1}{2} \sum_n^\infty \frac{m}{\delta} \left(\frac{d y_n}{dt}\right)^2 \delta \rightarrow T = \int_{-\infty}^\infty \frac{1}{2} \mu \left(\frac{\partial y}{\partial t}\right)^2 dx\]
Podemos construirnos entonces un lagrangiano
\begin{equation} \label{6.1.1}
    \pazocal{L} = \int_{-\infty}^\infty \left[\frac{1}{2} \mu \left(\frac{\partial y}{\partial t}\right)^2 - \tau \left(\frac{\partial y}{\partial x}\right)^2\right] dx= \int_{-\infty}^\infty \mathfrak{L}dx \ \ \ \ \ S = \int \mathfrak{L}(y,\partial_t y,\partial_x y,t) dxdt
\end{equation}\refstepcounter{subsection}
Se puede comprobar que aplicando el cálculo variacional para minimizar la acción de (10.1.6) se obtiene de nuevo (10.1.5). Además se pueden obtener unas ecuaciones equivalentes a las de E-L para medios contínuos generales que involucran la densidad lagrangiana $\mathfrak{L}$.

\section{La ecuación de ondas}
\refstepcounter{subsection}
\vspace{-20pt}
\begin{equation} \label{6.1.1}
    \frac{\partial^2 \eta}{\partial t^2} = v^2 \frac{\partial^2 \eta}{\partial x^2}
\end{equation}\refstepcounter{subsection}
Es facil demostrar sustituyendo que la solución general de (10.2.1) es de la siguiente forma, donde la primera solución se corresponde a una onda que se desplaza a la derecha y la segunda a la izquierda.
\begin{equation} \label{6.1.1}
    \eta(x,t) = f(x-vt) + g(x+vt)
\end{equation}\refstepcounter{subsection}
La solución particular de la cuerda discreta una vez hecho el paso al contínuo, y por lo tanto, solución partícular de (10.2.1) son las ondas harmónicas
\begin{equation} \label{6.1.1}
    \eta(x,t) = A e^{i(\omega t \mp k x)} = A e^{ ik (x \mp vt)}  \ \ \ \ \omega = vk
\end{equation}\refstepcounter{subsection}
En general una onda es toda aquella función que sea superposición de ondas harmónicas de la forma
\begin{equation} \label{6.1.1}
    \eta(x,t) = \sum_i A_i e^{i(\omega(k_i) t \pm k_i x)} \ \ \mbox{  ó  } \ \ \eta(x,t) = \int a(k) e^{i(\omega(k) t \pm k x)}dk
\end{equation}\refstepcounter{subsection}
Se dice que una onda es \textbf{no dispersiva} si $\omega(k)$ es lineal es $k$, y entonces verifica la ecuación (10.2.1), de lo contrario es  \textbf{dispersiva} y verfica otra ecuación de onda distinta.
\subsubsection{Ondas en 2D y 3D}
La ecuación de ondas en más dimensiones se generaliza como
\begin{equation} \label{6.1.1}
    \frac{\partial^2 \eta}{\partial t^2} = v^2 \nabla^2 \eta
\end{equation}\refstepcounter{subsection}
Una solución particular en 2 y 3 dimensiones son las ondas planas, que se escriben como
\begin{equation} \label{6.1.1}
    \eta(\mathbf{r},t) = A e^{i(\omega t - \mathbf{k} \cdot \mathbf{r})}  \ \ \ \ \mathbf{k} = |\mathbf{k}| \hat{\mathbf{n}}
\end{equation}\refstepcounter{subsection}
De esta forma, los frentes de onda, es decir, el lugar geométrico de todos los puntos con la misma fase en un instante concreto del tiempo, verifican que $\mathbf{k}\cdot (\mathbf{r}_2-\mathbf{r}_1) = 0$, es decir, rectas, que son perpendiculares a $\hat{\mathbf{n}}$, la dirección de propagación de la onda.

La longitud de onda, $\lambda$, es la distancia entre dos frentes de onda y viene dada por $\lambda = 2\pi/|\mathbf{k}|$, también puede definirse un periodo, al igual en 1D, como $T = 2\pi/\omega$, que es lo que tarda la onda en una posición del espacio en volver a adquirir la misma fase.

Otra solución particular, en este caso para ondas tridimensionales, son las ondas esféricas, o al menos un tipo de ellas, pues en coordenadas esféricas con solo dependencia radial la ecuación (10.2.5) toma la forma
\begin{equation} \label{6.1.1}
    \frac{\partial^2 \eta}{\partial t^2} = v^2 \frac{1}{r^2}\frac{\partial}{\partial r}\left(r^2 \frac{\partial \eta}{\partial r}\right)
\end{equation}\refstepcounter{subsection}
Se puede simplificar suponiendo $\eta(r,t) = \frac{1}{r} F(r,t)$, de tal forma que sustituyendo al final obtenemos (10.2.1) para $F$, así las ondas harmónicas en esféricas serán
\begin{equation} \label{6.1.1}
    \eta(r,t) = \frac{1}{r}A e^{i(\omega t \mp kr)}
\end{equation}\refstepcounter{subsection}
Y en general tendremos que para ondas de este tipo la solución general es
\begin{equation} \label{6.1.1}
    \eta(r,t) = \frac{1}{r}\left(f(r-vt) + g(r+vt)\right)
\end{equation}\refstepcounter{subsection}
La energía es proporcional a la amplitud al cuadrado, que en este caso va como $1/r^2$, entonces si calculamos la energía en esferas, esta se conserva, pues el área de la esfera va como $r^2$, esta propiedad es muy importante para campos electromagnéticos.

\section{Condiciones de frontera y ondas estacionarias}
\refstepcounter{subsection}
Si sumamos dos ondas harmónicas con misma $k$ y $A$ pero sentido opuesto, obtenemos 
\begin{equation} \label{6.1.1}
    \eta(x,t) = Ae^{i(\omega t - kx)}+Ae^{i(\omega t + kx)} = A e^{i\omega t}(e^{-ikx}+e^{ikx}) = 2A e^{i\omega t} \cos{kx}
\end{equation}\refstepcounter{subsection}
De esta forma, se 'desacoplan' las dependencias espaciales y temporales y tenemos lo que se conoce como una onda estacionaria, con unos nodos fijos separados por $\Delta x_n = \lambda/2$.

Si ahora consideramos un caso similar a (10.3.1), pero con una de las ondas ligeramente desfasadas, llegamos a
\[\eta(x,t) = A e^{i\omega t}(e^{-ikx}+e^{i(kx + \phi)}) = A e^{i(\omega t +\phi/2)} (e^{-i(kx+\phi/2)}+e^{i(kx + \phi/2)}) = \]
\begin{equation} \label{6.1.1}
    = 2A e^{i(\omega t +\phi/2)} \cos(kx+\phi/2)
\end{equation}\refstepcounter{subsection}
Ahora, en función del problema con el que estemos tratando, podemos aplicar condiciones de contorno distintas, si aplicamos que $\eta(x_0,t) = c$, estas se conocen como condiciones de Dirichlet, por ejemplo en (10.3.2) si imponemos $\eta(0,t) = 0$, entonces $\phi = \pi +2\pi l$, $l\in \mathbb{Z}$. Por otro lado, podríamos aplicar que $\partial_x \eta|_{(x_0,t)} =c$, condiciones de Neumann, por ejemplo si imponemos que $\partial_x \eta|_{(0,t)} =0$, entonces $\phi = 0 +2\pi l$.

La primera condición esta relacionada con una onda que se ve reflejada al chochar con un obstaculo, el ejemplo que hemos puesto es una onda que se ve completamente reflejada y no pierde energía, para que se conserve la energía y el momento, la onda reflejada debe tener la misma amplitud y fase contraria (puesto que la velocidad ha cambiado de signo), y por lo tanto se forma un nodo, si se transmite energía al obstaculo, entonces, la amplitud es menor y la fase no es exactamente la contraria y entonces no se anulan y no se forma un nodo.

Para ondas planas en más dimensiones, lo que se refleja es la componente perpendicular al punto de colisión, de esta forma
\begin{equation} \label{6.1.1}
\eta(\mathbf{r},t) = A e^{i(\omega t-\mathbf{k}_{\mbox{\tiny ||}}\cdot\mathbf{r}_{\mbox{\tiny ||}})}(e^{-ik_{\mbox{\tiny $\perp$}}r_{\mbox{\tiny $\perp$}}}+e^{i(k_{\mbox{\tiny $\perp$}}r_{\mbox{\tiny $\perp$}} + \phi)}) = 2A e^{i(\omega t -\mathbf{k}_{\mbox{\tiny ||}}\cdot\mathbf{r}_{\mbox{\tiny ||}} +\phi/2)} \cos(k_{\mbox{\tiny $\perp$}}r_{\mbox{\tiny $\perp$}} +\phi/2)
\vspace{-25pt}
\end{equation}\refstepcounter{subsection}
\subsection{Modos normales y series de Fourier}
Si tenemos una cuerda finita con extremos fijos en $x=0$ y $x=L$, de tal forma que $\eta(0,t) = 0$ y $\eta(L,t) =0$, tomaremos como solución dos ondas harmónicas en sentidos opuestos, inspirándonos en el apartado anterior, y con amplitudes distintas, pues necesitamos determinar dos constantes si tenemos dos condiciones de contorno.
\[\eta(x,t) = A e^{i(\omega t - kx)}+B e^{i(\omega t + kx)}\]
Aplicando la primera condición tenemos que $B=-A$, y aplicando la segunda tenemos que $e^{-ikL} - e^{ikL} = 0$, lo cual se traduce en la siguiente condición para $k$
\begin{equation} \label{6.1.1}
    \sin(kL) = 0 \implies k_n L = \pi n \ \ \ \ \ n\in \mathbb{N} \ \rightarrow k_n = \frac{\pi n}{L} \ \ \ \lambda_n = \frac{2L}{n}
\end{equation}\refstepcounter{subsection}
Si la onda esta centrada en el origen, se obteniene una relación similar pero con un coseno que nos da valores impares de $k_n$. Además, también se puede dejar uno de los extremos suelto y hay que aplicar una condición de Neumann, y se obtiene otra relación distinta para $k_n$.

La onda resultante de (10.3.4) es la misma que en (10.3.1) pero con un seno, de tal forma que la solución general de una onda en esta cuerda es de la forma, donde $\omega_n = v k_n$ para ondas no dispersivas y $\omega_n=\omega(k_n)$ para ondas dispersivas
\begin{equation} \label{6.1.1}
    \eta(x,t) = \sum_{n=1}^\infty C_n e^{i\omega_n t} \sin{k_n x} = \sum_{n=1}^\infty \sin{k_n x}\left(A_n \cos(\omega_n t)+B_n \sin(\omega_n t)\right)
\end{equation}\refstepcounter{subsection}
De esta forma, para estas condiciones de contorno, si queremos saber la solución de la onda dadas unas condiciones iniciales $\eta(x,0)$ y $\partial_t \eta |_{(x,0)}$, tenemos que
\begin{equation} \label{6.1.1}
    \eta(x,0) =  \sum_{n=1}^\infty A_n \sin{k_n x} \ \ \ \ \partial_t \eta |_{(x,0)} = \sum_{n=1}^\infty B_n \omega_n \sin{k_n x}
\end{equation}\refstepcounter{subsection}
Entonces simplemente descomponiendo en series de Fourier las condiciones iniciales, y obteniendo $A_n$ y $B_n$, ya tenemos la solución general usando (10.3.5)

Se puede hacer la descomposición haciendo integrales ortogonales, como se detalla en (8.4), pero para ahorrarnos tener que hacer las integrales los resultados finales son
\begin{equation} \label{6.1.1}
    A_m = \frac{2}{L}\int_{0}^{L} \eta(x,0)\cos(k_m x)dx \ \ \ \ \ B_m = \frac{2}{L \omega_m}\int_{0}^{L} \partial_t \eta |_{(x,0)}\sin(k_m x)dx
\end{equation}\refstepcounter{subsection}
Si el intervalo esta centrado en el origen se pueden usar unas expresiones equivalentes a las de (8.4.2).
\section{Solución general, Transformada de Fourier}
\refstepcounter{subsection}
Si tenemos una onda que abarca todo el espacio que tiene una relación de dispersión $\omega(k)$, y conocemos las condiciones iniciales $\eta(x,0)$ y $\partial_t \eta |_{(x,0)}$, vamos a poder hallar la solución general.

La solución general para una onda, por definición, como se dijo en (10.2.4) se puede escribir como una Transformada de Fourier, el límite contínuo de la serie de Fourier, que es como sumar a todas las posibles longitudes de onda, cada una con su amplitud (compleja) $\hat{A}(k)$.
\begin{equation} \label{6.1.1}
    \eta(x,t) = \frac{1}{\sqrt{2\pi}} \int_{-\infty}^\infty \hat{A}(k) e^{i(\omega(k)t + kx)}dk
\end{equation}\refstepcounter{subsection}
Entonces tenemos que para $t=0$ tenemos
\begin{equation} \label{6.1.1}
    \eta(x,0) = \frac{1}{\sqrt{2\pi}} \int_{-\infty}^\infty \hat{A}(k) e^{ikx}dk
\end{equation}\refstepcounter{subsection}
Si ahora multiplicamos ambos lados de la ecuación $1/\sqrt{2pi} \int e^{-i\tilde{k}x} dx$ tenemos
\[\frac{1}{\sqrt{2\pi}}\int_{-\infty}^\infty \eta(x,0) e^{-i\tilde{k}x}dx = \frac{1}{2\pi} \int_{-\infty}^\infty \hat{A}(k) e^{i(k-\tilde{k})x}dkdx\]
\newpage
Ahora, se puede ver que la siguiente función se comporta como una delta de Dirac
\begin{equation} \label{6.1.1}
    \frac{1}{2\pi} \int_{-\infty}^\infty e^{i(k-\tilde{k})x}dx = \delta(k-\tilde{k})
\end{equation}\refstepcounter{subsection}
Puesto que si $k = \tilde{k}$, entonces el integrando es 1 y la integral diverge a $\infty$, sin embargo, si $k \neq \tilde{k}$, entonces, aplicando que la función es holomorfa, tenemos que 
\[\int_{-\infty}^\infty e^{iax}dx = \lim_{R\rightarrow \infty }\int_{c_R} e^{iaz} dz\]
Donde $c_R$ es un contorno semicircular ubicado en el semiplano inferior o superior en función del signo de $a$. Aplicando el Lema de Jordan tenemos que
\[\left|\int_{c_R} e^{iaz} dz\right| \leq \frac{\pi}{R} \implies \lim_{R\rightarrow \infty }\int_{c_R} e^{iaz} dz = 0\]
Por lo tanto cuando $k \neq \tilde{k}$, la integral vale 0, así hemos comprobado que se comporta como una Delta de Dirac, el factor $1/2\pi$ esta ahí para que su integral a todos los $k$ sea 1.

Ahora aplicando las propiedades de la de la Delta de dirac llegamos a que 
\begin{equation} \label{6.1.1}
    \hat{A}(\tilde{k}) = \frac{1}{\sqrt{2\pi}}\int_{-\infty}^\infty \eta(x,0) e^{-i\tilde{k}x}dx
\end{equation}\refstepcounter{subsection}

Así, sacando $\hat{A}(\tilde{k})$ a partir de las condiciones iniciales, podemos obtener la solución general de la onda usando (10.4.1). Si además la onda es no dispersiva entonces tenemos
\begin{equation} \label{6.1.1}
    \eta(x,t) = \frac{1}{\sqrt{2\pi}} \int_{-\infty}^\infty \hat{A}(k) e^{ik(x+\mbox{\tiny sign}(k) v t)}dk
\end{equation}\refstepcounter{subsection}
\subsection{Velocidad de grupo}
Vamos a suponer que tenemos una onda tal que $\hat{A}(k)$ sigue una distribución normal, donde $k_0$ es la media y $\Delta k$ la desviación típica (anchura). A esto se le conoce como paquete
\[\hat{A}(k) = A_0 e^{-\frac{(k-k_0)^2}{2\Delta k^2}}\]
Entonces la solución de esa onda es
\[\eta(x,t) = \frac{A_0 }{\sqrt{2\pi}} \int_{-\infty}^\infty e^{-\frac{(k-k_0)^2}{2\Delta k^2}} e^{i(\omega(k)t + kx)}dk\]
En esta forma, la solución depende de $\omega(k)$, pero vamos a expandir esta función en Taylor y vamos a suponer que $\delta k = k-k_0$ es pequeño, de tal forma que nos quedamos a primer orden, tal que $\omega(k) \approx \omega(k_0) + \omega'(k_0) \Delta k = \omega_0 + u \delta k$.
\[\eta(x,t) = \frac{A_0 }{\sqrt{2\pi}} e^{i(\omega_0 t - k_0 x)}\int_{-\infty}^\infty \exp\left[-i\frac{\delta k^2}{2\Delta k^2}-i\delta k(x-ut)\right]dk\]
Ahora podemos completar cuadrados en la exponencial de la integral y tenemos
\[\eta(x,t) = \frac{A_0 }{\sqrt{2\pi}} e^{i(\omega_0 t - k_0 x)}e^{-\frac{\Delta k^2}{2}(x-ut)^2}\int_{-\infty}^\infty \exp\left[-\frac{1}{2\Delta k^2}\left(\delta k+ i\Delta k^2 (x-ut)\right)^2\right]dk\]
La integral de la izquierda la podemos escribir como una gaussiana, donde $a =  k_0 -i\Delta k^2 (x-ut)$
\[I = \int_{-\infty}^\infty \exp\left[-\frac{1}{2\Delta k^2}\left(k-a\right)^2\right]dk = \Delta k \sqrt{2\pi}\]
Esa solución vale incluso si $a\in \mathbb{C}$, para ello basta ver que derivando respecto a $a$ y haciendo el cambio de variable $u = -(k-a)^2/2\Delta k ^2$ sale 0
\[\frac{dI}{da} = \int_{-\infty}^\infty \frac{k-a}{\Delta k^2} \exp\left[-\frac{1}{2\Delta k^2}\left(k-a\right)^2\right] dk = \int e^u du = \left|\exp\left[-\frac{1}{2\Delta k^2}\left(k-a\right)^2\right]\right|_{-\infty}^\infty = 0\]
Así llegamos finalmente a la solución (aproximada) de la onda
\begin{equation} \label{6.1.1}
    \eta(x,t) = A_0 \Delta k e^{i(\omega_0 t - k_0 x)}e^{-\frac{\Delta k^2}{2}(x-ut)^2}
\end{equation}\refstepcounter{subsection}
De esta forma tenemos una onda con forma de paquete gaussiano con una anchura $1/\Delta k$ que se mueve a velocidad $u$, lo que denominamos la velocidad de grupo. Y esta envolvente modula a una onda viajera de longitud de onda $k_0$ y velocidad $v_0 = \omega_0/k_0$, la velocidad de cresta o fase.

Si la onda es no dispersiva, $u = v$, y el paquete se mueve a la misma velocidad que la fase, pero si es dispersiva, por lo general las velocidades van a ser distintas y el paquete va a ir atradasado o adelantado a la fase.

Si tenemos en cuenta más términos de $\omega(k)$, se pueden observar más efectos de la dispersión, únicos de las ondas dispersivas, como por ejemplo que la anchura del paquete crece.

Se observa también un esbozo del principio de incertidumbre, si $\Delta k$ es pequeño, quiere decir que la onda esta formada por pequeño intervalo de longitudes de onda, pero entonces eso va a implicar que la anchura de la onda, que va como $1/\Delta k$, sea muy grande, y viceversa.

\pagelayout{wide} % No margins
\addpart[Dispersión]{\fontsize{55pt}{0pt} \setstretch{2} \textbeuron{Dispersion}}
\pagelayout{margin} % Restore margins

\chapter{Teoría de la dispersión}
\refstepcounter{subsection}

%----------------------------------------------------------------------------------------

\backmatter % Denotes the end of the main document content
\setchapterstyle{plain} % Output plain chapters from this point onwards

%----------------------------------------------------------------------------------------
%	BIBLIOGRAPHY
%----------------------------------------------------------------------------------------

% The bibliography needs to be compiled with biber using your LaTeX editor, or on the command line with 'biber main' from the template directory

%\defbibnote{bibnote}{Here are the references in citation order.\par\bigskip} % Prepend this text to the bibliography
%\printbibliography[heading=bibintoc, title=Bibliography, prenote=bibnote] % Add the bibliography heading to the ToC, set the title of the bibliography and output the bibliography note

%----------------------------------------------------------------------------------------
%	INDEX
%----------------------------------------------------------------------------------------

% The index needs to be compiled on the command line with 'makeindex main' from the template directory

\printindex % Output the index

\end{document}
