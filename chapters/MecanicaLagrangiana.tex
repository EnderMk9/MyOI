\chapter{Mecánica Lagrangiana}
\labch{intro}
%-------------------------------------------------------------------------------------------
Ahora la variable independiente sobre la que vamos a trabajar va a ser el tiempo, $t$, y las variables dependientes son las coordenadas cartesianas $\{x_{\alpha i}\}$, donde $\alpha$ indica la partícula y $i$ indica la componente de la posición.
Definimos además las derivadas totales temporales como $\{\dot{x}_{\alpha i}\}$.

\section{Principio de Hamilton}\refstepcounter{subsection}

Definimos una función llamada \textbf{Lagrangiano}\sidenote{La definición de \textbf{Lagrangiano} dependerá de la configuración del sistema físico, pero como norma géneral en mecánica clásica (2.1.1) es la expresión más común de la función que verifica (2.1.3).}
\begin{equation} \label{2.1.1}
    \pazocal{L}(\{x_{\alpha i},\dot{x}_{\alpha i}\};t)=T-U
\end{equation} \refstepcounter{subsection}
Dónde $T$ es la energía cinética del sistema y $U$ es la energía potencial (conservativa o no), de tal forma que definimos el siguiente funcional llamado \textbf{acción}
\begin{equation} \label{2.1.2}
    S \equiv \int_{t_A}^{t_B}{\pazocal{L}(\{x_{\alpha i},\dot{x}_{\alpha i}\};t)dt}
\end{equation} \refstepcounter{subsection}
\textbf{Principio de Hamilton o de mínima acción.} La evolución temporal de un sistema físico es aquella que extremiza la acción, es decir que $\delta S = 0$ para la evolución real del sistema, lo cual es equivalente a
\begin{equation} \label{2.1.3}
    \frac{\partial \pazocal{L}}{\partial x_{\alpha i}} -\frac{d}{dt}\left(\frac{\partial \pazocal{L}}{\partial \dot{x}_{\alpha i}}\right) =0
\end{equation} \refstepcounter{subsection}
Para la mecánica clásica, este principio es equivalente a las leyes de \textit{Newton}, cuando $\pazocal{L}$ toma la forma de (2.1.1) con ligeras modificaciones que discutiremos en las próximas secciones.
%-------------------------------------------------------------------------------------------
\subsubsection{Muelle elástico}
Un sencillo ejemplo para aplicar este principio es el de un muelle elástico en una dirección, donde $T=m\dot{x}^2/2$ y $U = kx^2/2$ (el término $mgh$ es constante y puede ser ignorado), si $\pazocal{L} = T-U$, entonces
\[\frac{\partial \pazocal{L}}{\partial x}=-kx \ \ \ \ \frac{\partial \pazocal{L}}{\partial \dot{x}}=m\dot{x} = p \ \ \ \ \frac{d p}{dt} = m\ddot{x} \rightarrow m\ddot{x} = -kx \iff F=-kx=ma\]
%-------------------------------------------------------------------------------------------
\section{Coordendas generalizadas} \refstepcounter{subsection}
Podemos realizar un cambio de variables para poder expresar $x_{\alpha i}$ en función de otras variables $q_j$, las cuales pueden resultarnos más sencillas para resolver un problema, tal que $x_{\alpha i}=x_{\alpha i}(\{q_j\};t)$. Esta transformación será invertible cuando
\vspace{-10pt}
\[J_l^k=\frac{\partial x_k}{\partial q_l}\]
el determinante de esa matriz, el jacobiano, sea no nulo, tal que existe la transformación $q_j = q_j(\{x_{\alpha i}\};t)$.

Usando la regla de la cadena podemos ver la dependencia de las velociades entre sí, $\dot{x}_{\alpha i}=\dot{x}_{\alpha i}(\{q_j,\dot{q}_j\};t)$ y que $\dot{q}_j = \dot{q}_j(\{x_{\alpha i},\dot{x}_{\alpha i}\};t)$.

De esta forma, podemos expresar $\pazocal{L}$ en función de las coordenadas y velocidades generalizadas, tal que $\pazocal{L}=\pazocal{L}(\{q_j,\dot{q}_j\};t)$ de tal forma que (2.1.3) queda como 

\vspace{-10pt}
\Large\begin{equation} \label{2.2.1}
    \boxed{\frac{\partial \pazocal{L}}{\partial q_j} -\frac{d}{dt}\left(\frac{\partial \pazocal{L}}{\partial \dot{q}_j}\right) =0}
\end{equation} \refstepcounter{subsection}\normalsize

Definimos además el \textbf{momento generalizado}, que para cartesianas es el momento lineal y para polares es el momento angular. También definimos la \textbf{fuerza generalizada}, que es la proyección del vector cartesiano en el sistema de vectores asociado a las coordenadas generalizadas.
\begin{equation} \label{2.2.2}
    p_j = \frac{\partial \pazocal{L}}{\partial \dot{q}_j} \ \ \ \ Q_j = -\frac{\partial U}{\partial q_j}=\sum_\alpha^N{\mathbf{F}_\alpha\cdot \frac{\partial \mathbf{r}_{\alpha}}{\partial q_j}}
\end{equation} \refstepcounter{subsection}
%-------------------------------------------------------------------------------------------
\section{Ligaduras}\refstepcounter{subsection}
Al igual que en la sección \textit{Ligaduras} de la sección 1.3, tendremos $M$ ecuaciones de ligadura, los tipos de las cuales se datallarán a continuación, pero antes definimos lo que vamos a denominar \textbf{grados de libertad}, que indica el número mínimo de parámetros que es necesario para especifcar la configuración del sistema en un tiempo dado, tal que $s = N\cdot d- M \label{2.3.1} \inlineeqnum$, donde $s$ son los \textbf{grados de libertad}, $N$ el número de partículas del sistema, y $d$ la dimensión del espacio.
%-------------------------------------------------------------------------------------------
\subsubsection{Tipos de ligaduras}
Cuando las ecuaciones de ligadura no dependen de las veclocidades, $G_i(\{q_j\})=0$, se denominan ligaduras \textbf{holónomas} y son con las que vamos a trabajar. Si las ecuaciones de ligadura dependen de la velocidad,$ G_i(\{q_j,\dot{q}_j\})=0$, se denominan \textbf{no holónomas} y salvo que sean integrables no trabjaremos con ellas. Son \textbf{integrables} cuando son de la forma siguiente donde $h=h(\{q_i\};t)$ tal que
\begin{equation} \label{2.3.2}
    \sum_j^{N\cdot d}{A_j(\{q_i\};t)\dot{q}_j} + B(\{q_i\};t)=0; \ \ \ \ A_j=\frac{\partial h}{\partial q_j}; \ \ \ \ B_j=\frac{\partial h}{\partial t}
\end{equation} \refstepcounter{subsection}
Entonces podemos ver que nos queda la regla de la cadena e integramos
\begin{equation} \label{2.3.3}
    \sum_j^{N\cdot d}{\frac{\partial h}{\partial q_j}\dot{q}_j} + \frac{\partial h}{\partial t}=\frac{d h}{dt} =0 \iff h(\{q_i\};t) - C = 0 \mbox{ (Holónoma)}
\end{equation} \refstepcounter{subsection}
Luego a parte si la ligadura depende explícitamente del tiempo se llama \textbf{forzada} o \textbf{reónoma}, si no depende explcítamente del tiempo, se denominan \textbf{naturales} o \textbf{esclerónomas}.
%-------------------------------------------------------------------------------------------
\subsection{Sistema holonómico}
Decimos que un sistema es \textbf{holonómico} cuando podemos resolver (o bien en cartesianas o en generalizadas) las ecuaciones de ligadura (holónomas) y expresar $m$ coordenadas como explícitamente dependientes de $s$ coordenadas independientes, reduciendo el sistema a $s$ variables que podemos resolver usando (2.2.1)  (\textit{E-L}).
%-------------------------------------------------------------------------------------------
\subsection{Multiplicadores de Lagrange}
En el caso en el que el sistema no sea holonómico, y no podamos resolver las ecuaciones de ligadura, al igual que en la sección 1.3 tenemos que recurrir a multiplicadores de multiplicadores de \textit{Lagrange}, podemos obtener una expresión equivalente a (1.3.15) modificando el lagrangiano de la siguiente forma
\begin{equation} \label{2.3.4}
    \pazocal{L}^* = \pazocal{L}+\sum^m{\lambda_i G_i}
\end{equation} \refstepcounter{subsection}
Que aplicando las ecuaciones de \textit{Euler-Lagrange} tanto para $q_j$ como para $\lambda_i$ resulta

\vspace{-10pt}
\Large\begin{equation} \label{2.3.5}
    \frac{\partial \pazocal{L}}{\partial q_j} -\frac{d}{dt}\left(\frac{\partial \pazocal{L}}{\partial \dot{q}_j}\right) +\underbrace{\sum^m\lambda_i(t)\frac{\partial G_i}{\partial q_j}}_{Q^L_j}=0
\end{equation} \refstepcounter{subsection}\normalsize
Donde $Q^L_j$ es la componente $j$ de la fuerza de ligadura total, tal que $\mathbf{F}^L=\sum{Q^L_j \hat{u}_{q_j}}$, que cumplen que $dW^L = \mathbf{F}^L\cdot d\mathbf{r}=0$ son fuerzas que o siempre perpendiculares a la ligadura, o que provocan que $d\mathbf{r}=0 \iff \mathbf{v}=0$ en el punto de contacto.

%-------------------------------------------------------------------------------------------
\section{Teorema de la Energía Cinética}\refstepcounter{subsection}
%-------------------------------------------------------------------------------------------
\subsection{Función k-homogénea}
Una función k-homogéna cumple la siguiente expresión, donde $\lambda$ es un parámetro arbitrario cualquiera
\begin{equation} \label{2.4.1}
    f(\{\lambda x_i\})=\lambda^k f(\{x_i\})
\end{equation} \refstepcounter{subsection}
\vspace{-35pt}
\subsubsection{Teorema de Euler}
Podemos derivar cada lado de (2.4.1) con respecto al parámetro
\[\frac{\partial}{\partial \lambda} f(\{\lambda x_i\})= \frac{\partial}{\partial \lambda} \lambda^k f(\{x_i\})\]
En el primer miembro hacemos la regla de la cadena y en el segundo es la derivada de una potencia
\[\sum_j^N \frac{\partial f(\{\lambda x_i\})}{\partial (\lambda x_j)} \frac{d (\lambda x_j)}{d \lambda}= \sum_j^N \frac{\partial f(\{\lambda x_i\})}{\partial (\lambda x_j)} x_j=  k \lambda^{k-1} f(\{x_i\})\]
Como $\lambda$ es un parámetro arbitrario, podemos tomar $\lambda=1$ y tenemos
\begin{equation} \label{2.4.2}
    \sum_j^N \frac{\partial f(\{x_i\})}{\partial x_j} x_j=  k f(\{x_i\})
\end{equation} \refstepcounter{subsection}
%------------------------------------------------------------------------------------------
\subsection{Forma cuadrática}
Una forma cuadrática es una función 2-homogénea de la siguiente forma
\begin{equation} \label{2.4.3}
    f(\{x_i\})=\sum_{i,j}^N{{a_{jk} x_j x_k}}=\mathbf{x} A \mathbf{x}^T
\end{equation} \refstepcounter{subsection}
Donde $a_{jk}$ no tienen por que ser constantes, pueden ser funciones de otras variables, pero no de $x_i$.
%------------------------------------------------------------------------------------------
\subsection{Teorema}
En coordenadas cartesianas la expresión de la energía cinética es una forma cuadrática que solo depende de las velocidades que tiene la siguiente forma
\begin{equation} \label{2.4.4}
    T=T(\{x_{\alpha i}\})=\frac{1}{2} \sum_{\alpha, i}^{N,d} m_\alpha \dot{x}_{\alpha i}^2
\end{equation} \refstepcounter{subsection}
\vspace{-25pt}

Si $x_{\alpha i}(\{q_j\};t)$, entonces
\begin{equation} \label{2.4.5}
    \dot{x}_{\alpha i}=\sum_j^s \frac{\partial x_{\alpha i}}{\partial q_j}\dot{q}_j + \frac{\partial x_{\alpha i}}{\partial t}=\dot{x}_{\alpha i}(\{q_j,\dot{q}_j\};t)
\end{equation} \refstepcounter{subsection}
Elevando (2.4.5) al cuadrado tenemos
\[\dot{x}_{\alpha i}^2= \left(\sum_j^s \frac{\partial x_{\alpha i}}{\partial q_j}\dot{q}_j \right) \left(\sum_k^s \frac{\partial x_{\alpha i}}{\partial q_k}\dot{q}_k \right) + 2\frac{\partial x_{\alpha i}}{\partial t} \sum_j^s \frac{\partial x_{\alpha i}}{\partial q_j}\dot{q}_j  + \left(\frac{\partial x_{\alpha i}}{\partial t}\right)^2 = \]
\begin{equation} \label{2.4.6}
    = \sum_{j,k}^s{\frac{\partial x_{\alpha i}}{\partial q_j}\frac{\partial x_{\alpha i}}{\partial q_k}\dot{q}_j\dot{q}_k} + 2\frac{\partial x_{\alpha i}}{\partial t} \sum_j^s \frac{\partial x_{\alpha i}}{\partial q_j}\dot{q}_j + \left(\frac{\partial x_{\alpha i}}{\partial t}\right)^2
\end{equation} \refstepcounter{subsection}
Sustituyendo (2.4.6) en (2.4.4)
\[ T = \frac{1}{2} \sum_{\alpha, i}^{N,d} m_\alpha \left[\sum_{j,k}^s{\frac{\partial x_{\alpha i}}{\partial q_j}\frac{\partial x_{\alpha i}}{\partial q_k}\dot{q}_j\dot{q}_k} + 2\frac{\partial x_{\alpha i}}{\partial t} \sum_j^s \frac{\partial x_{\alpha i}}{\partial q_j}\dot{q}_j + \left(\frac{\partial x_{\alpha i}}{\partial t}\right)^2\right]\]
Usando que los sumatorios conmutan y son lineales llegamos a
\begin{equation}  
    T = \sum_{j,k}^s{\left(\sum_{\alpha,i}^{N,d}{ \frac{1}{2} m_\alpha \frac{\partial x_{\alpha i}}{\partial q_j}\frac{\partial x_{\alpha i}}{\partial q_k}\right)\dot{q}_j\dot{q}_k}} + \sum_j^s{\left( \sum_{\alpha,i}^{N,d}{m_\alpha \frac{\partial x_{\alpha i}}{\partial t}\frac{\partial x_{\alpha i}}{\partial q_j}\right)}\dot{q}_j} + \sum_{\alpha,i}^{N,d}{\frac{1}{2} m_\alpha\left(\frac{\partial x_{\alpha i}}{\partial t}\right)^2} \label{2.4.7}
\end{equation} \refstepcounter{subsection}
De una forma más reducida obtenemos
\begin{equation} \label{2.4.8}
    T = T(\{q_j,\dot{q}_j\};t) = \sum_{j,k}^sA_{ij}\dot{q}_j\dot{q}_k + \sum^s_j B_j \dot{q}_j + C
\end{equation} \refstepcounter{subsection}
Así, fijándonos en (2.4.7), si el cambio de coordenadas no depende explícitamente del tiempo, $B_j$ y $C$ se anulan, y entonces $T$ es una forma cuadrática en los $\dot{q}_j$.

\textbf{Teorema de la Energía cinética.} Si las coordenadas no dependen explícitamente del tiempo, entonces $T$ es una forma cuadrática en los $\dot{q}$.

Si ahora partimos de este supuesto y hacemos la parcial de $T$ con respecto a un $\dot{q}_l$ dado, obtenemos
\[\frac{\partial T}{\partial \dot{q}_l} = \cancelto{0}{\frac{\partial}{\partial \dot{q}_l} \sum_{j,k\neq l}^s{A_{jk}^s\dot{q}_j\dot{q}_k}}+\frac{\partial}{\partial \dot{q}_l} \sum_{j=l, k\neq l}^s{A_{lk}\dot{q}_l\dot{q}_k} + \frac{\partial}{\partial \dot{q}_l} \sum_{j\neq l, k = l}^s{A_{jl}\dot{q}_j\dot{q}_l + \frac{\partial}{\partial \dot{q}_l} \left(A_{ll} \dot{q}_l^2\right) = \]
\begin{equation} \label{2.4.9}
    =\sum_{j=l, k\neq l}^s{A_{lk}\dot{q}_k} + \sum_{j\neq l, k = l}^s{A_{jl}\dot{q}_j} + 2A_{ll} \dot{q}_l =2 \sum_i^s A_{li} \dot{q}_i=2 \sum_i^s A_{il} \dot{q}_i
\end{equation} \refstepcounter{subsection}
Si ahora hacemos lo siguiente usando (2.4.9), vemos que se verifica (2.4.2)
\begin{equation} \label{2.4.10}
    \sum_j^s \frac{\partial T}{\partial \dot{q}_j} \dot{q}_j = 2 \sum_{i,k}^s{A_{kj}\dot{q}_j\dot{q}_k=2 T}
\end{equation} \refstepcounter{subsection}