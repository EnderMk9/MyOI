\chapter{Sistemas de referencia no inerciales} 
\refstepcounter{subsection}
Llamemos $\pazocal{S}_0$, al sistema de referencia inercial, que lleva asociado un origen espacial $\pazocal{O}_0 = \mathbf{O}_{\pazocal{S}_0}$, un origen temporal $\pazocal{O}_{t_0}=0_{\pazocal{S}_0}$, cuyos ejes cartesianos se encuentran fijos, con coordenadas $(x_0,y_0,z_0)$. Lo representamos por $\pazocal{S}_0 = \left\{\pazocal{O}_0,\pazocal{O}_{t_0},(\mathbf{e}_{x_0},\mathbf{e}_{y_0},\mathbf{e}_{z_0})\right\}$.

Tenemos otro sistema de referencia $\pazocal{S}$ no inercial, con sus orígenes, $\pazocal{O}$ y $\pazocal{O}_{t}$ que pueden ser iguales o distintos a los de $\pazocal{S}_0$, con sus ejes cartesianos fijos en él mismo, con coordenadas $(x,y,z)$, tal que $\pazocal{S} = \left\{\pazocal{O},\pazocal{O}_{t},(\mathbf{e}_x,\mathbf{e}_y,\mathbf{e}_z)\right\}$.

Definimos formalmente que los ejes estan fijos con respecto a un sistema de referencia cuando 
\begin{equation} \label{6.1.1}
    \left(\frac{d}{dt}\mathbf{e}_{j_0}\right)_{\pazocal{S}_0} = 0 \ \ \ \ \ \left(\frac{d}{dt}\mathbf{e}_j\right)_{\pazocal{S}} = 0 \ \ \ \ \ \forall j
\end{equation} \refstepcounter{subsection}
Entonces un sistema de referencia es no inercial cuando, para algún $j$,
\begin{equation} \label{6.1.1}
    \left(\frac{d}{dt}\mathbf{e}_j\right)_{\pazocal{S}_0} \neq 0 \iff \left(\frac{d}{dt}\mathbf{e}_{j_0}\right)_{\pazocal{S}} \neq 0 
\end{equation} \refstepcounter{subsection}
es decir, alguno de los vectores de $\pazocal{S}$ no es constante respecto a $\pazocal{S}_0$. También es un sistema no inercial en general cuando $\pazocal{O}$ este acelerado con respecto a un $\pazocal{S}_0$.
\section{Aceleración rectilínea}
\refstepcounter{subsection}
Si tenemos que $\pazocal{S}$ esta se mueve de forma rectilínea respecto a $\pazocal{S}_0$ con aceleración $\mathbf{A}=\dot{\mathbf{V}}$, donde $\mathbf{V}$ es la velocidad de $\pazocal{S}$ respecto a $\pazocal{S}_0$.

Si $\mathbf{r}_0$ y $\mathbf{r}$ son vectores de posición equivalentes en cada sistema de referencia, estos se relacionan por
\begin{equation} \label{6.1.1}
    \dot{\mathbf{r}}_0=\dot{\mathbf{r}}+\mathbf{V} \rightarrow \ddot{\mathbf{r}}_0=\ddot{\mathbf{r}}+\mathbf{A}
\end{equation} \refstepcounter{subsection}
Si ahora tenemos la 2LN, llegamos a que aparece lo que llamamos '\textit{fuerza inercial}' en el sistema no inercial
\begin{equation} \label{6.1.1}
    \mathbf{F} = m\ddot{\mathbf{r}}_0 = m\ddot{\mathbf{r}} + m \mathbf{A} \implies m\ddot{\mathbf{r}} = \mathbf{F}-m \mathbf{A} = \mathbf{F}+\mathbf{F}_{\mbox{\small iner.}}
\end{equation} \refstepcounter{subsection}
\vspace{-25pt}
\subsection{Mareas}
Vamos a asumir que ha habido un diluvio universal y la Tierra esta cubierta de océano. Las mareas son un fenómeno en el que la altura del mar varía con periodo de unas 12 horas.

Si tenemos un objeto $m$ en la superficie del océano. Considerando que la rotación de la Luna sobre la Tierra tiene un periodo mucho mayor que las mareas, podremos considerar el sistema de referencia de la Tierra como un sistema no inercial con aceleración rectilínea y que la Tierra y la Luna no rotan en ese intervalo temporal.
\begin{figure}[H]
    \def\svgwidth{15 cm}
    \normalsize
	\input{images/marea.pdf_tex}
	\labfig{margin2}
    \vspace{-75pt}
    \caption{Diagrama del sistema}
\end{figure}
\vspace{15pt}
La aceleración de la Tierra con respecto al CM de ambos cuerpos es la fuerza gravitatoria que sufre la tierra debida a la masa de la luna partido por la masa de la tierra, dirección a la Luna.
\begin{equation} \label{6.1.1}
    \mathbf{A} = -\frac{G M_L}{d_0^2} \mathbf{e}_{d_0}
\end{equation} \refstepcounter{subsection}

Entonces por (6.1.2) la fuerza inercial que sufre $m$ en su sistema de referencia es
\begin{equation} \label{6.1.1}
    \mathbf{F}_{\mbox{\small iner.}} = \frac{GmM_L}{d_0^2}\mathbf{e}_{d_0}
\end{equation} \refstepcounter{subsection} 
Además, $m$ sufre las fuerzas gravitatorias de la Tierra, la Luna, y la fuerza de Arquímedes del agua, tal que
\begin{equation} \label{6.1.1}
    \mathbf{F}_T = m\mathbf{g} \ \ \ \ \ \mathbf{F}_L = -\frac{GmM_L}{d^2}\mathbf{e}_d
\end{equation} \refstepcounter{subsection}
También deberíamos tener en cuenta la fuerza centrífuga de la rotación de la Tierra sobre su propio eje, pero la vamos a despreciar por que no es muy grande en comparación al resto.
Entonces, usando (6.1.2), la 2LN para $m$ nos queda
\begin{equation} \label{6.1.1}
    m \ddot{\mathbf{r}} = m\mathbf{g} + \mathbf{F}_A - GmM_L\left(\frac{\mathbf{e}_d}{d^2}-\frac{\mathbf{e}_{d_0}}{d_0^2}\right) = m\mathbf{g} + \mathbf{F}_A + \mathbf{F}_{\mbox{\small marea}}
\end{equation} \refstepcounter{subsection}
Los casos en los que $m$ se encuentra en los puntos $A$ y $B$, donde $\mathbf{e}_d=\mathbf{e}_{d_0}$, en el primer caso $d<d_0$ y en el segundo $d>d_0$, así podemos obtener el sentido de la fuerza de marea en esos puntos usando (6.1.6), como se observa en el diagrama 
\begin{equation} \label{6.1.1}
    \mathbf{F}_{\mbox{\small marea}}=-GmM_L\left(\frac{1}{d^2}-\frac{1}{d_0^2}\right)\mathbf{e}_d \ \ \ \ \ \begin{matrix}
        \mathbf{F}^A_{\mbox{\small marea}} \propto -\mathbf{e}_d \\
        \mathbf{F}^B_{\mbox{\small marea}} \propto \ \ \ \mathbf{e}_d
    \end{matrix}
\end{equation} \refstepcounter{subsection}
Cuando $m$ se encuentra en $C$ o $D$, $d^2=r^2+d_0^2$, y podemos hacer la siguiente aproximación por series de Taylor cuando $r<<d_0$, lo cual es cierto en este caso.
\begin{equation} \label{6.1.1}
    d = d_0\left[1+\left(\frac{r}{d_0}\right)^2\right]^{1/2} = d_0+ O\left(\left(\frac{r}{d_0}\right)^2\right)
\end{equation} \refstepcounter{subsection}
Por otro lado $d\mathbf{e}_d = d_0\mathbf{e}_{d_0}+ \mathbf{r}$, usando (6.1.7) llegamos a que
\begin{equation} \label{6.1.1}
    \mathbf{e}_d \approx \mathbf{e}_{d_0}+ \frac{\mathbf{r}}{d_0}
\end{equation} \refstepcounter{subsection}
Entonces usando (6.1.6), (6.1.8) y (6.1.9) llegamos a que, como se observa en el diagrama
\begin{equation} \label{6.1.1}
    \mathbf{F}_{\mbox{\small marea}}\approx-GmM_L\frac{\mathbf{r}}{d_0^3} \propto -\mathbf{r}
\end{equation} \refstepcounter{subsection}
Entonces de (6.1.7) y (6.1.10) concluimos que horizontalmente las fuerzas son hacía fuera, y verticalmente son hacía dentro, por lo que de forma cualitativa nos podemos imaginar de forma cualitativa que el océano hace la curva achatada dibujada en el diagrama.

El hecho de que la Tierra este en rotación sobre su propio eje explica el periodo de 12 horas de las mareas, puesto que al transcurrir medio día, te encuentras en el punto antipodal al que te encontrabas, pero la altura de la marea es muy similar.

\subsubsection{Diferencia de altura}
Ahora queremos determinar la diferencia de altura del océano entre $A$ y $C$.

Como $\mathbf{F}_A$ es perpendicular a la superficie, eso implica que $m\mathbf{g}+\mathbf{F}_{\mbox{\small marea}}$ también lo es en el equilibrio por (6.1.6). Entonces los puntos de equilibrio deben de formar una superficie equipotencial de un potencial asociado a $m\mathbf{g}+\mathbf{F}_{\mbox{\small marea}}$, pues su gradiente, $m\mathbf{g}+\mathbf{F}_{\mbox{\small marea}}$, es siempre perpendicular a esa superficie.

El potencial de $m\mathbf{g}$ es de la forma $mgr$ con $r$ por ejemplo la distancia al centro de la Tierra, tal que $m\mathbf{g}=-\nabla(mgr)=-mg\mathbf{e}_r=m\mathbf{g}$.

La fuerza de marea la podemos descomponer en dos sumandos, tal que
\begin{equation} \label{6.1.1}
    \mathbf{F}_{\mbox{\small marea}}= -GmM_L\frac{\mathbf{e}_d}{d^2}+GmM_L\frac{\mathbf{e}_{d_0}}{d_0^2}=-\nabla\left(-\frac{GmM_L}{d}\right)-\nabla\left(-\frac{GmM_Lx}{d_0^2}\right)
\end{equation} \refstepcounter{subsection}
$d_0$ es aproximadamente constante en las escalas temporales en las que trabajamos. $x$ es la coordenada cartesiana horizontal y $\mathbf{e}_x=\mathbf{e}_d$. $d$ es una coordenada esférica no centrada en el origen, podemos expresar el gradiente así porque $\mathbf{d}=\mathbf{d}_0+\mathbf{r}$ y los gradientes de $d$ y $r$ son iguales ya que $\mathbf{d}_0$ es constante.

Entonces como tratamos con una superfice equipotencial, tenemos que $U(A)=U(C)$ y entonces tenemos que pasando los términos similares a cada lado
\begin{equation} \label{6.1.1}
    mg\Delta r = \Delta U_{\mbox{\small marea}}
\end{equation} \refstepcounter{subsection}
$\Delta r = h$ es la diferencia de altura que queremos hallar, así que tenemos que trabajar con el lado derecho para encontrar una expresión para esta.

A partir de (6.1.11) llegamos a que $U_{\mbox{\small marea}}$ es 
\begin{equation} \label{6.1.1}
    U_{\mbox{\small marea}} = -GmM_L\left(\frac{1}{d}+\frac{x}{d_0^2}\right)
\end{equation} \refstepcounter{subsection}
Entonces para $C$ tenemos que $d_c$ podemos aproximarla al igual que (6.1.8)
\begin{equation} \label{6.1.1}
    \frac{1}{d_c} = \frac{1}{d_0}\left[1+\left(\frac{r}{d_0}\right)^2\right]^{-1/2} = \frac{1}{d_0}\left[1-\frac{1}{2}\left(\frac{r}{d_0}\right)^2\right] +O\left(\left(\frac{r}{d_0}\right)^4\right)
\end{equation} \refstepcounter{subsection}
Entonces $U_{\mbox{\small marea}}(c)$ es, usando (6.1.13), (6.1.14), que $x_C=0$ y que $r = R_T+h \approx R_T$
\begin{equation} \label{6.1.1}
    U_{\mbox{\small marea}}(C) \approx -\frac{GmM_L}{d_0}\left(1-\frac{R_T^2}{2d_0^2}\right)
\end{equation} \refstepcounter{subsection}
Para $A$ tenemos que si $x_A = -r$, entonces $d_A = d_0+x_A =d_0-r$, esta última la aproximaremos para que también nos salga un orden cuadrático como en  (1.1.14)
\begin{equation} \label{6.1.1}
    \frac{1}{d_A} = \frac{1}{d_0}\left[1+\frac{r}{d_0}\right]^{-1} = \frac{1}{d_0}\left[1-\frac{r}{d_0} + \left(\frac{r}{d_0}\right)^2\right] +O\left(\left(\frac{r}{d_0}\right)^3\right)
\end{equation} \refstepcounter{subsection}
Entonces usando $r \approx R_T$, $U_{\mbox{\small marea}}(A)$ será
\begin{equation} \label{6.1.1}
    U_{\mbox{\small marea}}(A) \approx -\frac{GmM_L}{d_0}\left(1+\frac{R_T^2}{d_0^2}\right)
\end{equation} \refstepcounter{subsection}
Entonces finalmente desarrollamos (6.1.12) usando (6.1.15) y (6.1.17)
\begin{equation} \label{6.1.1}
mgh = U_{\mbox{\small marea}}(C)-U_{\mbox{\small marea}}(A) \approx \frac{GmM_L}{d_0} \frac{3}{2} \frac{R_T^2}{d_0^2}
\end{equation} \refstepcounter{subsection}
Entonces despejando $h$ de (6.1.18) y usando $g=GM_T/R_T^2$ llegamos a
\begin{equation} \label{6.1.1}
    h \approx \frac{GM_L}{gd_0} \frac{3}{2} \frac{R_T^2}{d_0^2} = \frac{3}{2} \frac{M_L}{M_T} \frac{R_T^4}{d_0^3} 
    \end{equation} \refstepcounter{subsection}
Para la Luna esto nos da del orden de medio metro, y para un sistema equivalente con el Sol nos da un cuarto de metro.

En función de la posición del Sol y la Luna, las mareas de ambos se sumarán o se restarán.
\section{Sistemas no inerciales en rotación}\refstepcounter{subsection}
Cualquier trayectoria no rectilínea puede intepresarse como una sucesión de arcos de círculo con radio y centro variables. Definimos la velocidad angular $\omega$ como $\dot{\theta}$, siendo $\theta$ el ángulo que recorre en ese arco circular.
\subsection{Rotación respecto a un punto fijo}
\begin{marginfigure}[0pt]
    \def\svgwidth{4 cm}
    \normalsize
	\input{images/rot.pdf_tex}
	\labfig{margin2}
    \vspace{-30pt}
\end{marginfigure}
\vspace{15pt}
Si tenemos un cuerpo describiendo un movimiento circular entorno a un eje y el origen de coordenadas del sistema inercial se encuentra en cualquier punto del eje.

La velocidad es, siendo $\rho$ el radio de rotación
\begin{equation} \label{6.1.1}
    v = \frac{ds}{dt} = \rho \frac{d\theta}{dt} = \rho \omega
    \end{equation} \refstepcounter{subsection}
Tenemos que $\rho = r \sin\alpha$, donde $r$ es la distancia al origen y $\alpha$ el ángulo que forma el vector posición $\mathbf{r}$ y el eje.
Entonces tenemos
\begin{equation} \label{6.1.1}
    v =  r \sin\alpha \omega
    \end{equation} \refstepcounter{subsection}
$\mathbf{v}$ es perpendicular al plano que forman $\mathbf{r}$ y el eje, pues es tangencial. Si definimos $\vec{\omega}=\omega \mathbf{e}_z$ donde $\mathbf{e}_z$ es paralelo al eje y va hacía arriba podemos deducir junto a (6.2.1)
\begin{equation} \label{6.1.1}
    \mathbf{v} =  \vec{\omega} \times \mathbf{r}
    \end{equation} \refstepcounter{subsection}
El orden hace que (6.2.3) verifique la regla de la mano derecha.

Entonces si tenemos $\mathbf{r}=\mathbf{e}_r$, podemos obtener la derivada de ese vector, que esta fijo en el sistema de referencia no inercial
\begin{equation} \label{6.1.1}
    \left(\frac{d \mathbf{e}_r}{dt}\right)_{\pazocal{S}_0} =  \vec{\omega} \times \mathbf{e}_r \implies \left(\frac{d \mathbf{e}}{dt}\right)_{\pazocal{S}_0} =  \vec{\omega} \times \mathbf{e}
    \end{equation} \refstepcounter{subsection}
Esto es por que en esféricas, $\mathbf{e}_r$ puede representar cualquier dirección.

Entonces si tenemos un vector cualquiera $\mathbf{Q}$ expresado en la base del sistema no inercial tenemos, usando la notación de Einstein para sumaciones
\begin{equation} \label{6.1.1}
    \mathbf{Q}=Q_i \mathbf{e}_i \ \ \ \ \ \ \dot{\mathbf{Q}}=\left(\frac{d \mathbf{Q}}{dt}\right)_\pazocal{S} = \frac{d Q_i}{dt} \mathbf{e}_i
    \end{equation} \refstepcounter{subsection}
La derivada de las componentes no depende del sistema de referencia por que son escalares. Como estamos en $\pazocal{S}$, su base es constante. Entonces si ahora hallamos la derivada con respecto a $\pazocal{S}_0$, usando (6.2.4)
\begin{equation} \label{6.1.1}
    \left(\frac{d \mathbf{Q}}{dt}\right)_{\pazocal{S}_0} = \frac{d Q_i}{dt} \mathbf{e}_i + Q_i \left(\frac{d \mathbf{e_i}}{dt}\right)_{\pazocal{S}_0} = \left(\frac{d \mathbf{Q}}{dt}\right)_\pazocal{S} + Q_i \vec{\omega} \times \mathbf{e}_i = \left(\frac{d \mathbf{Q}}{dt}\right)_\pazocal{S} + \vec{\omega} \times \mathbf{Q}
    \end{equation} \refstepcounter{subsection}
Esta expresión es muy importante y nos permite relacionar derivadas vectoriales de un vector en cada sistema de referencia.

Ahora vamos a reescribir la 2LN en el sistema de referencia $\pazocal{S}$, primero lo escribimos en $\pazocal{S}_0$
\begin{equation} \label{6.1.1}
    m\left(\frac{d^2 \mathbf{r}}{dt^2}\right)_{\pazocal{S}_0}= \mathbf{F}
    \end{equation} \refstepcounter{subsection}
Y entonces expresamos la derivada en términos de $\pazocal{S}$
\[
    \left(\frac{d^2 \mathbf{r}}{dt^2}\right)_{\pazocal{S}_0}= \left(\frac{d}{dt}\left(\frac{d \mathbf{r}}{dt}\right)_{\pazocal{S}_0}\right)_{\pazocal{S}_0} = \left(\frac{d}{dt}\left[\left(\frac{d \mathbf{r}}{dt}\right)_\pazocal{S} + \vec{\omega} \times \mathbf{r}\right]\right)_{\pazocal{S}_0}=
    \]
\begin{equation} \label{6.1.1}
    = \left(\frac{d}{dt}\left(\frac{d \mathbf{r}}{dt}\right)_\pazocal{S}\right)_{\pazocal{S}_0} + \vec{\omega} \times \left(\frac{d\mathbf{r}}{dt}\right)_{\pazocal{S}_0} = \left(\frac{d^2 \mathbf{r}}{dt^2}\right)_\pazocal{S} + 2\vec{\omega} \times \left(\frac{d \mathbf{r}}{dt}\right)_\pazocal{S} + \vec{\omega} \times \left(\vec{\omega} \times \mathbf{r}\right)
    \end{equation} \refstepcounter{subsection}
Usando (6.2.8) llegamos a que (6.2.7) nos queda en $\pazocal{S}$ como
\begin{equation} \label{6.1.1}
    m\left(\frac{d^2 \mathbf{r}}{dt^2}\right)_{\pazocal{S}}= m\ddot{\mathbf{r}}=  \mathbf{F} + 2 m \dot{\mathbf{r}} \times  \vec{\omega} + m\left(\vec{\omega} \times \mathbf{r}\right) \times \vec{\omega} = \mathbf{F} + \mathbf{F}_{\mbox{\small cor}} + \mathbf{F}_{\mbox{\small cf}}
    \end{equation} \refstepcounter{subsection}

Esta es la ecuación que describe el movimiento de un cuerpo visto desde un sistema no inercial giratorio, por ejemplo, las auténticas ecuaciones que rigen la caída libre vista desde un observador en movimiento junto a la Tierra.
\subsection{Rotación Terrestre}    
La $\omega$ de la Tierra con respecto a su eje es de alrededor $7.3 \cdot 10^{-5}$ s$^{-1}$, de (6.2.9) podemos saber que a una altura pequeña tal que $r\approx R_T$, $|\mathbf{F}_{\mbox{\small cf}}|\sim m \omega^2 R_T $ y $|\mathbf{F}_{\mbox{\small cor}}|\sim m v \omega$.

Entonces haciendo el siguiente cociente, obtenemos las velocidades para las que la fuerza de coriolis es despreciable en la superficie terrestre.
\begin{equation} \label{6.1.1}
    \frac{|\mathbf{F}_{\mbox{\small cor}}|}{|\mathbf{F}_{\mbox{\small cf}}|} \sim \frac{v}{\omega R_T} \approx \frac{v}{500 \mbox{ m s}^{-1}} \rightarrow v \ll 500 \mbox{ m s}^{-1} \implies \mathbf{F}_{\mbox{\small cor}} \mbox{ es despreciable}
\end{equation} \refstepcounter{subsection}
\vspace{-20pt}
\subsubsection{Fuerza centrífuga}  
\begin{marginfigure}[0pt]
    \def\svgwidth{3.5 cm}
    \normalsize
	\input{images/tierracf.pdf_tex}
	\labfig{margin2}
    \vspace{-30pt}
\end{marginfigure}
Vamos a estudiar la fuerza centrígufa, que es $m\left(\vec{\omega} \times \mathbf{r}\right) \times \vec{\omega}$, sin tener en cuenta la fuerza de Coriolis, que estudiaremos posteriormente.

Analizando con la regla de la mano derecha, podemos concluir que la dirección de la fuerza centrífuga es $\mathbf{e}_\rho$, en el plano de giro, normal a trayectoria y hacía fuera.

El producto vectorial primero es (6.2.2) y (6.2.3), y ese vector es tiene dirección $\mathbf{e}_\varphi$, que es perpendicular a $\vec{\omega}$ por que es resultado de un producto vectorial de $\vec{\omega}$, entonces tenemos que la fuerza centrífuga puede expresarse como, donde $\theta$ es la colatitud
\begin{equation} \label{6.1.1}
    \mathbf{F}_{\mbox{\small cf}}= m r \omega^2 \sin \theta \  \mathbf{e}_\rho = m \rho \omega^2 \  \mathbf{e}_\rho
\end{equation} \refstepcounter{subsection}
Vamos a estudiar entonces la caída libre en un cuerpo en rotación como puede ser la tierra, suponiendo primero que estamos a una altura relativamente baja de la tierra, es decir $r\approx R_T$, tenemos entonces que la fuerza gravitatoria es
\begin{equation} \label{6.1.1}
    \mathbf{F}_{g}= - m \frac{M_T}{R_T^2} \mathbf{e}_r = -m g_0 \mathbf{e}_r = m \mathbf{g}_0
\end{equation} \refstepcounter{subsection}
Haciendo el siguiente cociente, obtenemos el orden de magnitud de (6.2.11)
\begin{equation} \label{6.1.1}
    \frac{|\mathbf{F}_{\mbox{\small cf}}|}{|\mathbf{F}_{\mbox{g}}|} \sim \frac{\omega^2 R_T}{g_0} \approx 3 \cdot 10^{-3}}
\end{equation} \refstepcounter{subsection}
\begin{marginfigure}[0pt]
    \def\svgwidth{5 cm}
    \normalsize
	\input{images/earth2.pdf_tex}
	\labfig{margin2}
    \vspace{-10pt}
\end{marginfigure}
\vspace{-15}
Podemos descomponer $\mathbf{e}_\rho$ en términos de $\mathbf{e}_r$ y $\mathbf{e}_\theta$, tal que
\begin{equation} \label{6.1.1}
    \mathbf{e}_\rho = \sin\theta \mathbf{e}_r + \cos\theta \mathbf{e}_\theta
\end{equation} \refstepcounter{subsection}
De esta forma, la suma de (6.2.11) y (6.2.12) resulta en, donde $\mathbf{g}$ es la aceleración de caída libre en el SRNI
\begin{equation} \label{6.1.1}
    \mathbf{F}_{g} +  \mathbf{F}_{\mbox{\small cf}} = m\left[\left(r\omega^2\sin^2\theta-g_0\right) \mathbf{e}_r + r\omega^2 \sin \theta \cos \theta \mathbf{e}_\theta \right] = m \left[-g_r \mathbf{e}_r + g_\theta \mathbf{e}_\theta\right] =  m\mathbf{g}
\end{equation} \refstepcounter{subsection}
De esta forma $\mathbf{g}$ forma un ángulo con $\mathbf{g}_0$, que es lo que normalmente consideramos únicamente, pero teniendo en cuenta la fuerza centrífuga, observamos que una masa no cae hacía el centro de la tierra, sino que cae ligeramente inclinado hacía el ecuador, esta va a ser la dirección vertical o de plomada.

Esta inclinación es del orden de unos pocos minutos de arco como máximo (para $\theta= 45^\circ ,135^\circ $) y viene dada por
\begin{equation} \label{6.1.1}
    \alpha = \tan^{-1}\left(\left|\frac{g_\theta}{g_r}\right|\right) = \tan^{-1}\left(\left|\frac{r\omega^2 \sin \theta \cos \theta}{r\omega^2\sin^2\theta-g_0}\right|\right)
\end{equation} \refstepcounter{subsection}
\vspace{-20pt}
\subsubsection{Fuerza de Coriolis}
Ahora vamos a estudiar la fuerza de Coriolis, que es $2 m \dot{\mathbf{r}} \times  \vec{\omega}$, que vemos que es el producto de la velocidad por otro vector, lo que nos recuerda a la fuerza magnética $q \mathbf{v} \times \mathbf{B}$, como en este caso $\vec\omega$ es constante, el comportamiento será análogo al de una carga en presencia de un campo magnético uniforme, es decir, va a tender a girar.

Tenemos entonces la siguiente ecuación del movimiento para la caída libre
\begin{equation} \label{6.1.1}
    m\ddot{\mathbf{r}}=  m\mathbf{g} + 2 m \dot{\mathbf{r}} \times  \vec{\omega}
\end{equation} \refstepcounter{subsection}
Cuando $\mathbf{r} \approx \mathbf{R}_T$ tenemos $\mathbf{g}$ constante, entonces la ecuación no depende de $\mathbf{r}$, y podemos hacer una traslación del origen desde el centro de la tierra a un punto de la superficie, tal que ahora $\tilde{\mathbf{r}} = \mathbf{r} - \mathbf{R}_T$, y como $\mathbf{R}_T$ es constante $\dot{\tilde{\mathbf{r}}} = \dot{\mathbf{r}}$. Como $\mathbf{r} \approx \mathbf{R}_T$, $\tilde{\mathbf{r}}$ es muy pequeño en comparación de $\mathbf{R}_T$.

Un punto viene representado por las coordenadas $(x,y,z)$ relativas a un punto concreto de la superficie, que se asocian a los vectores $(\mathbf{e}_\varphi,-\mathbf{e}_\theta,-\mathbf{e}_\mathbf{g} \approx \mathbf{e}_r)$, tal que $x$ representa el este, $y$ representa el norte, y $z$ representa la altura en términos de la dirección de plomada que aproximaremos a la dirección radial tal que $\alpha << 1$.

Entonces tenemos que
\begin{equation} \label{6.1.1}
    m\ddot{\tilde{\mathbf{r}}}=  m\mathbf{g} + 2 m \dot{\tilde{\mathbf{r}}} \times  \vec{\omega} \ \ \ \ \mathbf{g} = \left(R_T\omega^2\sin^2\theta-g_0\right) \mathbf{e}_r + R_T\omega^2 \sin \theta \cos \theta \mathbf{e}_\theta 
\end{equation} \refstepcounter{subsection}
Vamos a calcular la fuerza de Coriolis en la base ortonormal y dextrógira que hemos escogido, expresando $\vec{\omega}$ en esa base
\begin{equation} \label{6.1.1}
    \left(\begin{matrix}
        \dot{x} \\ \dot{y} \\ \dot{z}
    \end{matrix}\right) \times \omega \left(\begin{matrix}
        0 \\  \sin\theta \\ \cos \theta
    \end{matrix}\right) = \omega \left(\begin{matrix}
        \dot{y}\cos\theta -\dot{z}\sin\theta \\  -\dot{x}\cos\theta \\ \dot{x}\sin\theta
    \end{matrix}\right)
\end{equation} \refstepcounter{subsection}
Llegamos entonces al siguiente sistema de ecuaciones diferenciales de segundo orden acopladas
\begin{equation} \label{6.1.1}
\left\{\begin{matrix}
  \ddot{x} = 2\dot{y}\omega\cos\theta -2\dot{z}\omega\sin\theta\\
  \ddot{y} = -2\dot{x}\omega \cos\theta \phantom{----,,}\\
  \ddot{z} = -g + 2\dot{x}\omega \sin\theta \phantom{---}
\end{matrix}\right.
\end{equation} \refstepcounter{subsection}
Las vamos a estudiar perturbativamente para $\omega<<1$, primero para orden $\omega = 0$ obtenemos
\begin{equation} \label{6.1.1}
    \left\{\begin{matrix}
      \ddot{x} = 0\\
      \ddot{y} = 0\\
      \ddot{z} = -g
    \end{matrix}\right. \implies
    \left\{\begin{matrix}
        x = x_0 + v_{x_0}t\phantom{--,,}\\
        y = y_0 + v_{y_0}t\phantom{--,,}\\
        z = z_0 + v_{z_0}t-\frac{g}{2}t^2
      \end{matrix}\right. , \ \ 
    \left\{\begin{matrix}
    \dot{x} = v_{x_0}\phantom{--,}\\
    \dot{y} = v_{y_0}\phantom{--,}\\
    \dot{z} = v_{z_0}-gt
    \end{matrix}\right. , \ \ g=g_0
\end{equation} \refstepcounter{subsection}
Este es el caso en el que la Tierra no gira, pero podemos tomar estas soluciones e introducirlas de nuevo en (6.2.2)
\begin{equation} \label{6.1.1}
    \left\{\begin{matrix}
      \ddot{x} = 2 v_{y_0} \omega\cos\theta -2 v_{z_0}\omega\sin\theta + 2 gt \omega\sin\theta\\
      \ddot{y} = -2 v_{x_0} \omega \cos\theta \phantom{----------,,}\\
      \ddot{z} = -g +2 v_{x_0} \omega \sin\theta \phantom{---------}
    \end{matrix}\right. 
\end{equation} \refstepcounter{subsection}
\begin{equation} \label{6.1.1}
    \left\{\begin{matrix}
      \dot{x} = \omega\left(2 v_{y_0} \cos\theta -2 v_{z_0}\sin\theta\right)t +  gt^2 \omega\sin\theta + v_{x_0}\\
      \dot{y} = -2 v_{x_0} \omega \cos\theta t + v_{y_0} \phantom{-----------,}\\
      \dot{z} = \left(-g +2 v_{x_0} \omega \sin\theta\right)t + v_{z_0} \phantom{--------,,}
    \end{matrix}\right. 
\end{equation} \refstepcounter{subsection}
\begin{equation} \label{6.1.1}
    \left\{\begin{matrix}
        x = \omega\left( v_{y_0} \cos\theta - v_{z_0}\sin\theta\right)t^2 +  \frac{g}{3}t^3 \omega\sin\theta + v_{x_0}t + x_0\\
        y = -v_{x_0} \omega \cos\theta t^2 + v_{y_0}t + y_0 \phantom{-----------}\\
        z = -\frac{g}{2}t^2 + v_{x_0} \omega \sin\theta t^2 + v_{z_0}t+z_0 \phantom{--------,,}
    \end{matrix}\right. 
\end{equation} \refstepcounter{subsection}
Esta es la solución a primer orden del sistema de ecuaciones, si volvemos a introducir (6.2.24) en (6.2.20) obtenemos términos $\omega^2$ y podríamos obtener una solución de segundo orden.
\begin{equation} \label{6.1.1}
    x =  x_0 + \frac{g}{3}t^3 \omega\sin\theta; \ \ y = y_0; \ \ z =z_0-\frac{g}{2}t^2 
\end{equation} \refstepcounter{subsection}
En el caso particular de que $\mathbf{v}_0 = 0$ tenemos un desplazamiento proporcional a $z_0^{3/2}$ en $z=0$ hacía el este en ambos hemisferios por que $\sin\theta >0$ siempre en la Tierra 
\begin{equation} \label{6.1.1}
    t = z_0^{1/2} \sqrt{\frac{2}{g}} \implies \Delta x = \frac{ \omega \sin\theta}{3} \sqrt{\frac{8}{g}} z_0^{3/2}
\end{equation} \refstepcounter{subsection}
\vspace{-15}
\subsubsection{Péndulo de Foucault}
\begin{marginfigure}[0pt]
    \def\svgwidth{3.5 cm}
    \normalsize
	\input{images/foucault.pdf_tex}
	\labfig{margin2}
    \vspace{-10pt}
\end{marginfigure}

\begin{marginfigure}[0pt]
    \def\svgwidth{3.5 cm}
    \normalsize
	\input{images/foucault2.pdf_tex}
	\labfig{margin2}
    \vspace{-10pt}
\end{marginfigure}

Tendremos una masa $m$ muy pequeña unida a una cuerda muy larga de longitud $L$, que puede oscilar durante un largo tiempo.

Además de las fuerzas no incerciales y la gravedad, hay que tener en cuenta la tensión de la cuerda, que usando el teorema de Tales y trigonometría llegamos a
\begin{equation} \label{6.1.1}
    \frac{-T_x}{T} = \frac{x}{L} \ \ \ \ \ \frac{-T_y}{T} = \frac{y}{L} \ \ \ \ \ T_z = T \cos \beta \ \ \ \ \ \sqrt{x^2 +y^2} =\rho = L \sin \beta \ \ \ \ \  \cos\beta = \frac{L-h}{L}
\end{equation} \refstepcounter{subsection}
Vamos a tomar oscilaciones tales que $\beta \ll 1$, tal que
\begin{equation} \label{6.1.1}
    \cos\beta = 1 -\frac{\beta^2}{2} + O(\beta^4) \ \ \ \ \ \ \ \sin \beta = \beta + O(\beta^3)
\end{equation} \refstepcounter{subsection}
Que hace que las expresiones de (6.2.27) nos queden como
\begin{equation} \label{6.1.1}
    \frac{\rho}{L} \approx \beta \implies \frac{x}{L} \approx \frac{y}{L} \approx \beta \ \ \ \ \ T_z \approx T \ \ \ \ \  \frac{h}{L} \approx \frac{\beta^2}{2}
\end{equation} \refstepcounter{subsection}
De esta forma, $h$ y sus derivadas son de orden cuadrado y podemos despreciarlos.

Escribimos entonces la 2 LN, y separamos las componentes, de la misma forma que en (6.2.20) añadiendo el término de la tensión
\begin{equation} \label{6.1.1}
    m\ddot{\tilde{\mathbf{r}}}= \mathbf{T} + m\mathbf{g} + 2 m \dot{\tilde{\mathbf{r}}} \times  \vec{\omega}
\end{equation} \refstepcounter{subsection}
\begin{equation} \label{6.1.1}
    \left\{\begin{matrix}
      \ddot{x} = T_x/m +2\dot{y}\omega\cos\theta -2\dot{z}\omega\sin\theta\\
      \ddot{y} = T_y/m -2\dot{x}\omega \sin\theta \phantom{-----,}\\
      \ddot{h} = T_z/m -g + 2\dot{x}\omega \sin\theta \phantom{---,,}
    \end{matrix}\right. \rightarrow
    \left\{\begin{matrix}
        \ddot{x} = T_x/m +2\dot{y}\omega\cos\theta \\
        \ddot{y} = T_y/m -2\dot{x}\omega \sin\theta \phantom{}\\
        0 = T_z/m -g \phantom{---,,}
      \end{matrix}\right.
\end{equation} \refstepcounter{subsection}
De tal forma que hemos despreciado $\ddot{h}$ por que es de orden $\beta^2$, el último término de la tercera ecuación porque ese término es de orden $\beta \omega$, siendo $\omega$ pequeño, y es mucho más pequeño en comparación con $T_z/m-g$, lo mismo ocurre con el último término de la primera ecuación, que es de orden $\beta^2 \omega$, mientras que los otros términos son de orden $\beta$ o $\beta \omega$. 

Con la tercera ecuación y (6.2.29) llegamos a que $T=mg$, y entonces usando (6.2.27) llegamos a 
\begin{equation} \label{6.1.1}
    \left\{\begin{matrix}
        \ddot{x} = -xg/L  +2\dot{y}\omega\cos\theta \\
        \ddot{y} = -yg/L -2\dot{x}\omega \sin\theta
      \end{matrix}\right. \rightarrow
    \left\{\begin{matrix}
        \ddot{x} = -\omega_0^2 x  +2\dot{y}\omega_z \\
        \ddot{y} = -\omega_0^2 y -2\dot{x}\omega_z
    \end{matrix}\right. \ \ \ \ \ \ \omega_0^2 = \frac{g}{L} \ \ \ \ \ \omega_z = \omega \cos\theta
\end{equation} \refstepcounter{subsection}
Para resolver este sistema de ecuaciones planteamos el número complejo $\eta = x+iy$, tal que $\ddot{\eta} = \ddot{x}+ i\ddot{y}$, entonces sustityendo desde (6.2.32) llegamos a 
\begin{equation} \label{6.1.1}
    \ddot{\eta} + 2 i \omega_z \dot{\eta} + \omega_0^2 \eta = 0 \ \ \ \ \ \ \eta = e^{-i\alpha t} \implies \alpha^2 -2\omega_z\alpha-\omega_0^2 = 0
\end{equation} \refstepcounter{subsection}
Las raices de ese polinomio son, aproximadamente, considerando que $\omega_z \ll \omega_0$
\begin{equation} \label{6.1.1}
    \alpha = \omega_z \pm \sqrt{\omega_z^2+\omega_0^2} = \omega_z \pm \omega_0 \sqrt{1+\frac{\omega_z^2}{\omega_0^2}} = \omega_z \pm \omega_0 \left(1 + O\left(\frac{\omega_z^2}{\omega_0^2}\right)\right) \approx \omega_z \pm \omega_0
\end{equation} \refstepcounter{subsection}
Tal que llegamos a 
\begin{equation} \label{6.1.1}
    \eta = e^{-i\omega_z t}\left(C_1 e^{i\omega_0t}+C_2e^{-i\omega_0t}\right)
\end{equation} \refstepcounter{subsection}
Imponiendo las condiciones iniciales $x_0 = A$, $y_0 = 0 \implies \eta(0) = A$ y $\dot{\eta}(0) = 0$, es decir parte desde un punto en reposo, usando $\omega_z \ll \omega_0$ llegamos a 
\begin{equation} \label{6.1.1}
    \eta = A \cos(\omega_0t)\left[\cos(\omega_zt) -i \sin(\omega_zt))\right]
\end{equation} \refstepcounter{subsection}
Y tomando la parte real para $x$ y la parte compleja para $y$ llegamos a 
\begin{equation} \label{6.1.1}
    \left\{\begin{matrix}
        x = \phantom{-}A \cos(\omega_0t)\cos(\omega_zt) \\
        y = -A \cos(\omega_0t)\sin(\omega_zt)
    \end{matrix}\right.
\end{equation} \refstepcounter{subsection}
La primera oscilación, rápida, ocurre en una dimensión y se corresponde al comportamiento oscilatorio habitual, mientras que la segunda oscilación corresponde a un moviento circular en sentido horario mucho más lento.
