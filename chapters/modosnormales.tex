\chapter{Modos normales}
\refstepcounter{subsection}
Vamos a suponer un caso más general formado por $N$ partículas que intectúan entre sí con un potencial $U$, que siguiendo las mísmas líneas que en (8.0.1) podemos expresar el potencial a pequeñas oscilaciones de la posición de equilibrio sin pérdida de generalidad estableciendo el mínimo y el origen de potencial en el origen de las coordenadas generalizadas $q_j$
\begin{equation} \label{6.1.1}
    U({q_j}) \approx \frac{1}{2}\mathbf{q}^{\mbox{\tiny T}} \mathbb{H}_{\mbox{\tiny U}}(0)\mathbf{q} = \frac{1}{2} \sum_{ij}^N \left.\frac{\partial^2 U}{\partial q_i q_j}\right|_0 q_i q_j \ \ \ \ \nabla_{\mathbf{q}} U \approx \mathbb{K} \mathbf{q} \ \ \ \ \mathbb{K} = \mathbb{H}_{\mbox{\tiny U}}(0)
\end{equation}\refstepcounter{subsection}
Por otro lado, usando el teorema de la energía cinética cuando las coordendas generalizadas no dependen explícitamente del tiempo, y aproximando la matriz a orden constante para tener orden cuadrático en T.
\begin{equation} \label{6.1.1}
     T \approx \sum_{j,k}^s{\left(\sum_{\alpha,i}^{N,d}{ \frac{1}{2} m_\alpha \left.\frac{\partial x_{\alpha i}}{\partial q_j}\right|_0\left.\frac{\partial x_{\alpha i}}{\partial q_k}\right|_0\right)\dot{q}_j\dot{q}_k}} = \frac{1}{2}\dot{\mathbf{q}}^{\mbox{\tiny T}} \mathbb{M}\dot{\mathbf{q}} \ \ \ \ \nabla_{\dot{\mathbf{q}}} T \approx \mathbb{M}\dot{\mathbf{q}}
\end{equation}\refstepcounter{subsection}
Entones aplicando (2.2.1) (E-L) llegamos a la siguiente ecuación que queremos resolver
\begin{equation} \label{6.1.1}
    \mathbb{M}\ddot{\mathbf{q}} = - \mathbb{K} \mathbf{q} \ \ \ \ M_{ij} = \sum_{\alpha,k}^{N,d}{ \frac{1}{2} m_\alpha \left.\frac{\partial x_{\alpha k}}{\partial q_j}\right|_0\left.\frac{\partial x_{\alpha k2}}{\partial q_k}\right|_0 \ \ \ \ K_{ij} = \left.\frac{\partial^2 U}{\partial q_i q_j}\right|_0
\end{equation}\refstepcounter{subsection}
En la práctica, lo que se hace es calcularte $T$ y $U$, aproximarlas a orden cuadrático de $\dot{\mathbf{q}}$ y $\mathbf{q}$ respectivamente, y sacar las matrices de las formas cuadráticas correspondientes, que serán $\mathbb{M}$ y $\mathbb{K}$, respectivamente. En ocasiones realizar las aproximaciones se vuelve complicado y es mejor sacar directamente $\mathbb{H}_{\mbox{\tiny U}}(0)$ de su definición.

En ocasiones también salen términos lineales en el potencial, lo cual significa que las coordenadas generalizadas que hemos escogido no tienen un mínimo en 0, para arreglarlo podemos utilizar completación de cuadrados en muchas circumstancias y cambiar a unas variables donde sí sea cuadrática, y si eso no funciona siempre se puede calcular el gradiente de $U$ y obtener los puntos donde se anula y centrar ahí las coordenadas.
\vspace{-15pt}
\subsection{Soluciones de la ecuación}
\refstepcounter{subsection}
Para resolver esa ecuación diferencial, hacemos la conjetura $\mathbf{q} = \vec{\xi} e^{i\omega t}$, tal que $\ddot{\mathbf{q}} = -\omega^2 \vec{\xi} e^{i\omega t}$, tal que nos queda la siguiente ecuación de autovalores
\begin{equation} \label{6.1.1}
    (\mathbb{K}-\omega^2 \mathbb{M})\vec{\xi} = 0
\end{equation}\refstepcounter{subsection}
Como se observa en (9.0.3), $\mathbb{M}$ y $\mathbb{K}$ son simétricas y definidas positivas, por lo tanto, por el teorema espectral, sabemos que exite una matriz $\mathbb{O} \mathbb{O}^{\mbox{\tiny T}} = \mathbb{I}$, de tal forma que $\mathbb{M} = \mathbb{O} \mathbb{D}_\mathbb{M} \mathbb{O}^{\mbox{\tiny T}}$, entonces si las entradas de $\mathbb{D}_\mathbb{M}$ son los autovalores positivos $\lambda_i$, podemos escribirla como el de una matriz diagonal cuyas entradas son $\sqrt{\lambda_i}$, $\mathbb{D}_\mathbb{M} = \mathbb{S} \mathbb{S}$, por lo tanto $\mathbb{M} = \mathbb{O} \mathbb{S} \mathbb{S} \mathbb{O}^{\mbox{\tiny T}} = \mathbb{R}^{\mbox{\tiny T}} \mathbb{R}$.

Entonces podemos hacer las siguientes manipulaciones, con $\vec{\zeta} = \mathbb{R}\vec{\xi}$ y $\mathbb{A} = (\mathbb{R}^{\mbox{\tiny T}})^{-1}\mathbb{K}\mathbb{R}^{-1}$, para llegar a 
\begin{equation} \label{6.1.1}
    (\mathbb{R}^{\mbox{\tiny T}})^{-1}(\mathbb{K}-\omega^2 \mathbb{M})\mathbb{R}^{-1} \mathbb{R}\vec{\xi} = (\mathbb{A}-\omega^2 \mathbb{I}) \vec{\zeta}= 0
\end{equation}\refstepcounter{subsection}
De esta forma, usando el hecho de que la operación inversa y transpueta conmutan, $\mathbb{A}^{\mbox{\tiny T}} = ((\mathbb{R}^{\mbox{\tiny T}})^{-1}\mathbb{K}\mathbb{R}^{-1})^{\mbox{\tiny T}}= (\mathbb{R}^{\mbox{\tiny T}})^{-1}\mathbb{K}^{\mbox{\tiny T}} \mathbb{R}^{-1}$ es simétrica, puesto que $\mathbb{K}$ lo es. Por lo tanto, por el teorema espectral existe una matriz $\mathbb{Z}\mathbb{Z}^{\mbox{\tiny T}} = \mathbb{I}$ tal que $\mathbb{Z}^{\mbox{\tiny T}} \mathbb{A} \mathbb{Z} = \mathbb{Z}^{\mbox{\tiny T}} (\mathbb{R}^{\mbox{\tiny T}})^{-1}\mathbb{K}\mathbb{R}^{-1} \mathbb{Z}=\mathbb{X}^{\mbox{\tiny T}} \mathbb{K} \mathbb{X} = \mathbb{D}$.

Se observa que $\mathbb{X} \mathbb{M} \mathbb{X}^{\mbox{\tiny T}} =\mathbb{Z} (\mathbb{R}^{\mbox{\tiny T}})^{-1}\mathbb{R}^{\mbox{\tiny T}} \mathbb{R}\mathbb{R}^{-1} \mathbb{Z}^{\mbox{\tiny T}} =\mathbb{I}$ y que $\mathbb{X}\mathbb{X}^{\mbox{\tiny T}} = \mathbb{Z} (\mathbb{R}^{\mbox{\tiny T}})^{-1} \mathbb{R}^{-1} \mathbb{Z}^{\mbox{\tiny T}} = \mathbb{Z} \mathbb{S}^{-1} \mathbb{O}^{-1} (\mathbb{O}^{\mbox{\tiny T}})^{-1} \mathbb{S}^{-1} \mathbb{Z}^{\mbox{\tiny T}} =  \mathbb{Z} (\mathbb{S}^{-1})^2 \mathbb{Z}^{\mbox{\tiny T}} = \mathbb{Z} \mathbb{D}_\mathbb{M}^{-1} \mathbb{Z}^{\mbox{\tiny T}}\neq \mathbb{I}$, por lo tanto, como se verá a continuación, los vectores $\vec{\xi}$ en general no van a ser ortogonales respecto a la identidad, sino respecto al la forma bilineal simétrica definida por $\mathbb{M}$.

Tenemos que $\vec{\xi} = \mathbb{R}^{-1} \vec{\zeta}$ es equivalente a $\mathbb{X} = \mathbb{R}^{-1} \mathbb{Z}$, puesto que $\mathbb{X}$ y $\mathbb{Z}$ son las matrices que contienen a los vectores $\vec{\xi}$ y $\vec{\zeta}$, respectivamente, como columnas. De esta forma, aplicando varias veces el torema espectral, hemos demostrado que la ecuación siempre tiene soluciones y hemos encontrado una expresión explícita para ellas.

Otra forma de ver todo esto que hemos hecho es aplicar directamente el Teorema espectral para $(\mathbb{M}^{-1}\mathbb{K}-\omega^2 \mathbb{I})\vec{\xi} = 0$. Un operador, en este caso $\mathbb{P} = \mathbb{M}^{-1}\mathbb{K}$, es autoadjunto con respecto a una forma bilineal expresada en una misma base, en este caso $\mathbb{M}$, si $\mathbb{P} = \mathbb{M}^{-1} \mathbb{P}^T \mathbb{M} = \mathbb{M}^{-1} (\mathbb{M}^{-1}\mathbb{K})^T \mathbb{M}= \mathbb{M}^{-1} \mathbb{K}^T (\mathbb{M}^{-1})^T\mathbb{M} = \mathbb{M}^{-1} \mathbb{K} \mathbb{M}^{-1}\mathbb{M} = \mathbb{P}$, por lo tanto el Teorema espectral se aplica y existe una matriz $\mathbb{X}$ que cumple que $\mathbb{X}^T \mathbb{M}\mathbb{X} = \mathbb{I}$ tal que $\mathbb{X}^{-1} \mathbb{P} \mathbb{X} = \mathbb{D}$, de tal forma que es diagonalizable y $\mathbb{X}$ es un cambio de base ortogonal de la forma bilineal desde $\mathbb{M}$ a $\mathbb{I}$.

$\mathbb{X}$ es la matriz de cambio de base de la base que diagonaliza a $\mathbb{K}$ hasta la base canónica de $\mathbf{q}$, cuyas columnas estan formadas por los autovectores de la ecuación (9.0.4), de tal forma que si expresamos $\mathbf{\mathbf{q}} = \mathbb{X} \vec{\varsigma}$ entonces la ecuación original (9.0.3) nos queda 
\begin{equation} \label{6.1.1}
    \mathbb{M}\mathbb{X} \ddot{\vec{\varsigma}} + \mathbb{K} \mathbb{X} \vec{\varsigma} = \mathbb{X}^{\mbox{\tiny T}}\mathbb{M}\mathbb{X} \ddot{\vec{\varsigma}} + \mathbb{X}^{\mbox{\tiny T}}\mathbb{K} \mathbb{X} \vec{\varsigma} = \mathbb{I}\ddot{\vec{\varsigma}} + \mathbb{D}\vec{\varsigma} = 0 \iff \ddot{\varsigma_i} + \omega_i^2 \varsigma_i = 0
\end{equation}\refstepcounter{subsection}
Las soluciones de cada una de esas ecuaciones, ya en forma real, son $\varsigma_i = A_i \cos{(\omega_i t -\delta_i)} = A_i \cos{(\omega_i t)}+B_i \sin{(\omega_i t)}$, dónde $A_i$ y $\delta_i$ dependen de las condiciones iniciales, de tal forma que la solución final es $\mathbf{q} = \sum_i^s A_i \vec{\xi}_i \cos{(\omega_i t -\delta_i)}$, que coincide con tomar la parte real de una combinación lineal arbitraria de las conjeturas $\mathbf{q} = \vec{\xi} e^{i\omega t}$.

Hay que tener en cuenta que si obtenemos $\omega_i = 0$ en alguno de los autovalores, eso significa que los movimientos asociados a ese autovalor mantienen al potencial $U$ en un mínimo, además la solución de $\ddot{\varsigma_i}= 0$ es distinta a las otras, es $\varsigma_i = A+Bt$.


\vspace{-15pt}
\subsection{Energía}
\refstepcounter{subsection}
Utilizando las expresiones de (9.0.1) y (9.0.2) tenemos que 
\begin{equation} \label{6.1.1}
    T = \frac{1}{2} \dot{\mathbf{q}}^{\mbox{\tiny T}}\mathbb{M} \dot{\mathbf{q}} = \frac{1}{2} \dot{\vec{\varsigma}}^{\mbox{\tiny T}} \mathbb{X}^{\mbox{\tiny T}} \mathbb{M} \mathbb{X}\dot{\vec{\varsigma}} =\frac{1}{2} \dot{\vec{\varsigma}}^{\mbox{\tiny T}} \mathbb{I}\dot{\vec{\varsigma}} = \frac{1}{2} \sum_i^s \dot{\varsigma}_i^2
\end{equation}\refstepcounter{subsection}
\begin{equation} \label{6.1.1}
    U = \frac{1}{2} \mathbf{q}^{\mbox{\tiny T}}\mathbb{K} \mathbf{q} = \frac{1}{2} \vec{\varsigma}^{\mbox{\tiny T}} \mathbb{X}^{\mbox{\tiny T}} \mathbb{K} \mathbb{X}\vec{\varsigma} = \frac{1}{2} \vec{\varsigma}^{\mbox{\tiny T}} \mathbb{D}\vec{\varsigma} =\frac{1}{2} \sum_i^s \omega_i^2\varsigma_i^2
\end{equation}\refstepcounter{subsection}
\begin{equation} \label{6.1.1}
    E = T+U = \frac{1}{2}\sum_i^2 \left(\dot{\varsigma}_i^2 + \omega_i^2\varsigma_i^2\right)
\end{equation}\refstepcounter{subsection}

\subsection{Péndulo Doble}
\refstepcounter{subsection}
Se va a hacer como ejemplo el péndulo doble para pequeñas oscilaciones en torno a la posición de equilibrio para dos masas iguales y longitudes de los péndulos iguales, por simplificar.
\[
    \begin{matrix}
        x_1 = l \sin\theta_1 && x_2 = l (\sin\theta_1 + \sin\theta_2) \\
        y_1 = l \cos\theta_1 && y_2 = l (\cos\theta_1 + \cos\theta_2)
    \end{matrix}
\]
\[
    T = \frac{1}{2} m \left(\dot{x}_1^2+\dot{y}_1^2+\dot{x}_2^2+\dot{y}_2^2\right) = \frac{1}{2} m l^2 \left(2\cos^2\theta_1 \dot{\theta}_1^2 +\cos^2\theta_2 \dot{\theta}_2^2 + \right.
\]\[
    \left.+2\cos\theta_1\cos\theta_2 \dot{\theta}_1 \dot{\theta}_2 + 2\sin^2\theta_1 \dot{\theta}_1^2 +\cos^2\theta_2 \dot{\theta}_2 - 2\sin\theta_1\cos\theta_2 \dot{\theta}_1 \dot{\theta}_2\right)  \approx  
\]\[
    \approx \frac{1}{2} m l^2 \left(2 \dot{\theta}_1^2 + \dot{\theta}_2^2 + 2 \dot{\theta}_1 \dot{\theta}_2\right) \implies \mathbb{M} = ml^2 \left[\begin{matrix}
        2 && 1 \\ 1 && 1
    \end{matrix}\right]
\]
\[U = -mg (y_1+y_2) = -mgl (2 \cos\theta_1+\cos\theta_2) \approx \frac{1}{2}mgl\left(2 \theta_1^2 + \theta_2^2\right)+ U_0 \implies \mathbb{K} = mgl \left[\begin{matrix}
    2 && 0 \\ 0 && 1
\end{matrix}\right]\]
Ahora resolvemos $\det(\mathbb{K}-\omega^2 \mathbb{M}) = 0$
\[\det\left[\begin{matrix}
    2g/l-2\omega^2 && -\omega^2 \\ -\omega^2 && g/l-\omega^2
\end{matrix}\right] = 0 \implies (\omega^2)^2=(2g/l-2\omega^2)(g/l-\omega^2)\]
Esta ecuación es bicuadrada y las soluciones positivas que obtenemos son
\[\omega_1 = \sqrt{\frac{g}{l}(2-\sqrt{2})} \ \ \ \ \ \omega_2 = \sqrt{\frac{g}{l}(2+\sqrt{2})}\]
Ahora obtenemos los autovectores
\[\left[\begin{matrix}
    2g/l-2g/l(2-\sqrt{2}) && -2g/l(2-\sqrt{2}) \\ -2g/l(2-\sqrt{2}) && g/l-2g/l(2-\sqrt{2})
\end{matrix}\right] \left(\begin{matrix}
    \xi_{11} \\ \xi_{21}
\end{matrix}\right) = 0 \implies \xi_{21} = \sqrt{2}\xi_{11}\]
\[\left[\begin{matrix}
    2g/l-2g/l(2+\sqrt{2}) && -2g/l(2-\sqrt{2}) \\ -2g/l(2+\sqrt{2}) && g/l-2g/l(2-\sqrt{2})
\end{matrix}\right] \left(\begin{matrix}
    \xi_{21} \\ \xi_{22}
\end{matrix}\right) = 0 \implies \xi_{22} = -\sqrt{2}\xi_{21}\]
Entonces, la solución será
\[\left(\begin{matrix}
    \theta_1 \\ \theta_2
\end{matrix}\right) = A_1 \left(\begin{matrix}
    1 \\ \sqrt{2}
\end{matrix}\right) \cos{(\omega_1 t -\delta_1)}+A_2 \left(\begin{matrix}
    1 \\ -\sqrt{2}
\end{matrix}\right) \cos{(\omega_2 t -\delta_2)}\]
