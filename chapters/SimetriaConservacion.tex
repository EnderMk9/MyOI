\chapter{Simetrias y cantidades conservadas}
\labch{intro}

\section{Ejemplos de invariancias} 
\subsection{Invariancia temporal y energía generalizada}\refstepcounter{subsection}
Si tenemos un desplazamiento arbitratio en el tiempo, $t\mapsto t + \delta t$, y se verifica que $\pazocal{L}(\{q_j,\dot{q}_j\};t)=\pazocal{L}(\{q_j,\dot{q}_j\};t+\delta t)$, esto implica que la parcial de $\pazocal{L}$ con respecto a $t$ es 0. Si ahora desarrollamos la derivadada total de de $\pazocal{L}$ con respecto a $t$, tenemos
\vspace{-13pt}
\[\frac{d \pazocal{L}}{dt}=\sum^s\left(\frac{\partial \pazocal{L}}{\partial q_j}\dot{q}_j+\frac{\partial \pazocal{L}}{\partial \dot{q}_j}\ddot{q}_j\right)+\cancelto{0}{\frac{\partial \pazocal{L}}{\partial t}}\]
El primer término del primer sumando dentro del sumario lo podemos expresar en función de (2.2.1)  (\textit{E-L}), tal que
\[\sum^s\left[\frac{d}{dt}\left(\frac{\partial \pazocal{L}}{\partial \dot{q}_j}\right)\dot{q}_j+\frac{\partial \pazocal{L}}{\partial \dot{q}_j}\ddot{q}_j\right]-\frac{d \pazocal{L}}{dt} = -\frac{\partial \pazocal{L}}{\partial t} =0\]
Ahora lo de dentro del paréntesis es la derivada de un producto, y usando la linearidad de la derivada
\begin{equation} \label{3.1.1}
    \frac{d}{dt}\left(\sum^s \frac{\partial \pazocal{L}}{\partial \dot{q}_j}\dot{q}_j-\pazocal{L}\right) = -\frac{\partial \pazocal{L}}{\partial t} =0
\end{equation} \refstepcounter{subsection}
Definimos entonces la función energía generalizada $H$, que se conservará cuando $\pazocal{L}$ no dependa explícitamente del tiempo.
\begin{equation} \label{3.1.2}
    H \equiv \sum^s \frac{\partial \pazocal{L}}{\partial \dot{q}_j}\dot{q}_j-\pazocal{L} \ \ \ \ \ \frac{d H}{dt}=-\frac{\partial \pazocal{L}}{\partial t}
\end{equation} \refstepcounter{subsection} 
Podemos además observar que si se verifican los supuestos del teorema de la energía cinética (el cambio de coordenadas no depende del tiempo) podemos aplicar (2.4.10), y la energía potencial es conservativa, llegamos a $H=E$, es decir, la energía generalizada es igual a la energía clásica.
\vspace{-10pt}
\begin{equation} \label{3.1.3}
    H = \cancelto{2T}{\sum^s\frac{\partial \pazocal{T}}{\partial \dot{q}_j}\dot{q}_j}-\cancelto{0}{\sum^s\frac{\partial \pazocal{U}}{\partial \dot{q}_j}\dot{q}_j}-(T-U)= 2T -T + U = T+U =E 
\end{equation} \refstepcounter{subsection}

\vspace{-25pt}
\subsection{Invariancia espacial}\refstepcounter{subsection}
Si tenemos un desplazamiento arbitrario en una de las coordenadas generalizadas, $q_k\mapsto q_k + \delta q_k$, y se verifica que $\pazocal{L}(q_k,\{q_j,\dot{q}_j\};t)=\pazocal{L}(q_k+\delta q_k,\{q_j,\dot{q}_j\};t)$, esto implica que la parcial de $\pazocal{L}$ con respecto a $q_k$ es 0. Cuando esto ocurre se dice que $q_k$ es una \textbf{variable ignorable}, y de (2.2.1)  (\textit{E-L}) deducimos que su momento generalizado asociado se conserva.
\begin{equation} \label{3.2.1}
    \frac{d}{dt}\left(\frac{\partial \pazocal{L}}{\partial \dot{q}_k}\right) = \dot{p}_k = 0 \implies p_k = C
\end{equation} \refstepcounter{subsection}
\section{Teorema de Noether} \refstepcounter{subsection}
Consideremos unas transformaciones genéricas $h_j$ de las coordendas $q_j$, parametrizadas por un parámetro $\epsilon$ independiente del tiempo tal que 
\begin{equation} \label{3.3.1}
    q_j \mapsto q'_j=h_j(\{q_i\},\epsilon) \ \ \ \ h_j(\{q_i\},0)=q_j
\end{equation} \refstepcounter{subsection}
Si el conjunto de las transformaciones $h_j$ verifican
\begin{equation} \label{3.3.2}
    \tilde{\pazocal{L}}(\(\{q_j,\dot{q}_j\};t;\epsilon) = \pazocal{L}(\(\{q_j,\dot{q}_j\};t)+\epsilon \frac{dB}{dt}+ O(\epsilon^2)=\pazocal{L}(\{h_j(\{q_i\},\epsilon),\dot{h}_j(\{q_i,\dot{q}_i\},\epsilon)\};t)
\end{equation} \refstepcounter{subsection}
Donde $B = B(\{q_i\};t)$, entonces se conseva la siguiente cantidad para soluciones de (2.2.1)(E-L)
\vspace{-20pt}
\begin{equation} \label{3.3.3}
    I (\{q_j,\dot{q}_j\};t) = \sum_j^s{\frac{\partial \pazocal{L}}{\partial \dot{q}_j}\left.\frac{\partial h_j}{\partial \epsilon}\right|_{\epsilon=0}} -B\ \ \ \ \ \frac{d I}{dt} = 0
\end{equation} \refstepcounter{subsection}
\vspace{-35pt}
\subsection{Demostración}
Primero hay que considerar por qué usamos la condición (3.2.2) para decir que $h_j$ forman una simetría, para ello, vamos considerar que una vez hecha la transformación $h_j$, variamos con respecto a otro parámetro llamado $a$ (con sus funciones $\eta_i$ asociadas)
\begin{equation} \label{3.3.3}
    \tilde{S} = \int_{t_A}^{t_B}  \tilde{\pazocal{L}}(\(\{q'_j,\dot{q}'_j\};t) = S + \epsilon B|_{t_A}^{t_B} \implies \delta_a \tilde{S} = \delta_a S
\end{equation} \refstepcounter{subsection}
Aplicando (1.2.5) vemos que la acción se mantiene invariante, es decir, que si consideramos $\epsilon \ll 1$, entonces si $q_j$ extremizan a $S$, entonces $q_j' = h_j(\{q_i\};\epsilon)$ extremizan a $\tilde{S}$, por este motivo se llaman simetrías. Hay que tener cuidado, muchas simetrías que cumplen $\tilde{\pazocal{L}}=\pazocal{L}$, pero a la hora de calcular (3.2.6) no sale 0, sino que luego se anula con los términos de orden superior.

Vamos a expandir en Taylor $h_j$ en términos de $\epsilon$ y usamos la definición (1.1.1) (despreciamos los órdenes superiores porque $\delta q_j$ es la variación primera, $\Delta q_j$, la variación total, sí que contiene a todos los órdenes)
\begin{equation} \label{3.3.3}
    q_j' = h_j({q_j},\epsilon) = q_j + \epsilon\left.\frac{\partial h_j}{\partial \epsilon}\right|_{\epsilon=0} + O(\epsilon^2) \implies \delta_\epsilon q_j = q_j'-q_j = \epsilon\left.\frac{\partial h_j}{\partial \epsilon}\right|_{\epsilon=0}
\end{equation} \refstepcounter{subsection}
Ahora tenemos la primera variación del lagrangiano siguiendo las simetrias definidas en (3.2.5), variando $\epsilon$, para unas funciones de refenrencia $q_j$ arbitrarias
\begin{equation} \label{3.3.3}
    \delta_\epsilon \pazocal{L}(\{q_j\}) = \epsilon \frac{dB}{dt} = \epsilon \left.\frac{d\tilde{\pazocal{L}}}{d\epsilon}\right|_{\epsilon=0} = \epsilon\sum_j^s \frac{\partial \pazocal{L}}{\partial q_j}\left.\frac{d h_j}{d\epsilon}\right|_{\epsilon=0} + \frac{\partial \pazocal{L}}{\partial \dot{q}_j}\left.\frac{d \dot{h}_j}{d\epsilon}\right|_{\epsilon=0}
\end{equation} \refstepcounter{subsection}

Ahora, calculamos $\delta_a \pazocal{L}$, la variación respecto a un parámetro a (junto con sus funciones  $\eta_i$ asociadas) arbitrario, usando la regla de la cadena, y además, en este caso los caminos de referencia $q_j$ extremizan el funcional,  por tanto, verifican las ecuaciones de \texit{Euler-Lagrange} (2.2.1), además usaremos que la derivada temporal y la variación conmutan como se demostró en el prmer capítulo
\begin{equation} \label{3.3.3}
    \delta_a \pazocal{L}(\{\delta q_j\}) =\sum_j^s \left[\frac{\partial \pazocal{L}}{\partial q_j} \delta_a q_j + \frac{\partial \pazocal{L}}{\partial \dot{q}_j} \delta_a \dot{q}_j\right] =\sum_j^s \left[\frac{d}{dt}\left(\frac{\partial \pazocal{L}}{\partial \dot{q}_j}\right) \delta_a q_j + \frac{\partial \pazocal{L}}{\partial \dot{q}_j} \delta_a \dot{q}_j\right] = \frac{d}{dt} \left(\sum_j^s \frac{\partial \pazocal{L}}{\partial \dot{q}_j} \delta_a q_j\right)
\end{equation} \refstepcounter{subsection}
Ahora, esta expresión (3.2.8) es válida para variaciones $\delta_a q_j$ arbitrarias, y los caminos $q_j$ verifican (2.2.1) (E-L). Por otro lado (3.2.7) se aplica solo para las funciones  $\delta_\epsilon q_j$ definidas en (3.2.6), puesto que son las que vienen dadas por la simetría, pero es válida para cualquier camino $q_j$ genérico. De esta forma evaluando (3.2.7) con las $q_j$ que verifican (2.2.1) (E-L) y evaluando (3.2.8) con (3.2.6), ambas variaciones serán iguales, tal que
\begin{equation} \label{3.3.3}
    \epsilon \frac{d}{dt} \left(\sum_j^s \frac{\partial \pazocal{L}}{\partial \dot{q}_j}\left.\frac{\partial h_j}{\partial \epsilon}\right|_{\epsilon=0}\right) = \epsilon \frac{dB}{dt} \implies \epsilon \frac{d}{dt}\left[\sum_j^s \frac{\partial \pazocal{L}}{\partial \dot{q}_j} \left.\frac{\partial h_j}{\partial \epsilon}\right|_{\epsilon=0} - B \right] = 0
\end{equation} \refstepcounter{subsection}
Ahora, $\epsilon$ es una variación arbitraria de la transformación de la simetría, por lo tanto, la derivada debe anularse, obteniendo (3.2.3).
\subsection{Transformaciones del tiempo}
Si además hacemos la transformación 
\begin{equation} \label{3.3.1}
    t \mapsto \tau=\tau(t,\epsilon) \ \ \ \ \tau(t,0)=t \ \ \ \ \ \ \ \delta t = \epsilon \frac{d \tau}{d\epsilon}
\end{equation} \refstepcounter{subsection}
Y tenemos que
\begin{equation} \label{3.3.2}
    \pazocal{L}(\(\{q_j,\dot{q}_j\};t)+\epsilon \frac{dB}{dt}+ O(\epsilon^2)=\pazocal{L}(\{h_j(\{q_i\},\epsilon),\dot{h}_j(\{q_i,\dot{q}_i\},\epsilon)\};\tau(t,\epsilon))
\end{equation} \refstepcounter{subsection}
Basta con añadir el término de la variación de $t$ en la regla de la cadena en (3.2.8), y aplicando los mismos razonamientos que antes, llegamos a una expresión más general del teorema (aunque hay una versión aún más general, véase \textit{Johns, O. D.; Analytical mechanics for relativity and quantum mechanics.}).
\begin{equation} \label{3.3.3}
    \frac{d}{dt}\left[\sum_j^s \frac{\partial \pazocal{L}}{\partial \dot{q}_j} \frac{\partial h_j}{\partial \epsilon} - B \right] + \frac{\partial \pazocal{L}}{\partial t}\frac{d \tau}{d\epsilon} = 0
\end{equation} \refstepcounter{subsection}
\subsection{Ejemplo}
Si tenemos una masa en en plano bajo la acción de una fuerza central, tal que $\pazocal{L}=1/2m(\dot{x}^2+\dot{y}^2)-U(\sqrt{x^2+y^2})$, si tomamos las transformaciones $x \mapsto x +\epsilon y$ y $y \mapsto y - \epsilon x$, vemos que el lagrangiano se mantiene invariante a orden $\epsilon^2$.
\[\pazocal{L}'=1/2m((\dot{x} +\epsilon \dot{y})^2+(\dot{y} -\epsilon \dot{x})^2)-U\left(\sqrt{(x +\epsilon y)^2+(y - \epsilon x)^2}\right)\]
\[\pazocal{L}'=1/2m(\dot{x}^2+\dot{y}^2+\cancel{\epsilon^2(\dot{x}^2+\dot{y}^2)}) - U\left(\sqrt{x^2 +y^2 +\cancel{\epsilon^2(x^2+y^2)}}\right)\]
Entonces la cantidad conservada es el momento angular
\[I=\frac{\partial \pazocal{L}}{\partial \dot{x}}\frac{d}{d\epsilon}(x+\epsilon y)+\frac{\partial \pazocal{L}}{\partial \dot{y}}\frac{d}{d\epsilon}(y-\epsilon x)=m(\dot{x}y-\dot{y}x)=-m \mathbf{r} \times \mathbf{v}= -\mathbf{J}_z\]
Si expresamos la transformación en términos de senos y cosenos de épsilon, en vez de sus series de taylor a primer orden, se puede observar que $\pazocal{L}$ es invariante a todos los órdenes.