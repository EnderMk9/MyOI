\chapter{Simetrias y cantidades conservadas}
\labch{intro}

\section{Ejemplos de invariancias} 
\subsection{Invariancia temporal y Hamiltoniano}\refstepcounter{subsection}
Si tenemos un desplazamiento arbitratio en el tiempo, $t\mapsto t + \delta t$, y se verifica que $\pazocal{L}(\{q_j,\dot{q}_j\};t)=\pazocal{L}(\{q_j,\dot{q}_j\};t+\delta t)$, esto implica que la parcial de $\pazocal{L}$ con respecto a $t$ es 0. Si ahora desarrollamos la derivadada total de de $\pazocal{L}$ con respecto a $t$, tenemos
\vspace{-13pt}
\[\frac{d \pazocal{L}}{dt}=\sum^s\left(\frac{\partial \pazocal{L}}{\partial q_j}\dot{q}_j+\frac{\partial \pazocal{L}}{\partial \dot{q}_j}\ddot{q}_j\right)+\cancelto{0}{\frac{\partial \pazocal{L}}{\partial t}}\]
El primer término del primer sumando dentro del sumario lo podemos expresar en función de (2.2.1)  (\textit{E-L}), tal que
\[\sum^s\left[\frac{d}{dt}\left(\frac{\partial \pazocal{L}}{\partial \dot{q}_j}\right)\dot{q}_j+\frac{\partial \pazocal{L}}{\partial \dot{q}_j}\ddot{q}_j\right]-\frac{d \pazocal{L}}{dt} = -\frac{\partial \pazocal{L}}{\partial t} =0\]
Ahora lo de dentro del paréntesis es la derivada de un producto, y usando la linearidad de la derivada
\begin{equation} \label{3.1.1}
    \frac{d}{dt}\left(\sum^s \frac{\partial \pazocal{L}}{\partial \dot{q}_j}\dot{q}_j-\pazocal{L}\right) = -\frac{\partial \pazocal{L}}{\partial t} =0
\end{equation} \refstepcounter{subsection}
Definimos entonces el \textit{Hamiltoniano} $\pazocal{H}$, que se conservará cuando $\pazocal{L}$ no dependa explícitamente del tiempo.
\begin{equation} \label{3.1.2}
    \pazocal{H} \equiv \sum^s \frac{\partial \pazocal{L}}{\partial \dot{q}_j}\dot{q}_j-\pazocal{L} \ \ \ \ \ \frac{d \pazocal{H}}{dt}=-\frac{\partial \pazocal{L}}{\partial t}
\end{equation} \refstepcounter{subsection} 
Podemos además observar que si se verifican los supuestos del teorema de la energía cinética (el cambio de coordenadas no depende del tiempo) podemos aplicar (2.4.10), y la energía potencial es conservativa, llegamos a $\pazocal{H}=E$
\vspace{-10pt}
\begin{equation} \label{3.1.3}
    \pazocal{H} = \cancelto{2T}{\sum^s\frac{\partial \pazocal{T}}{\partial \dot{q}_j}\dot{q}_j}-\cancelto{0}{\sum^s\frac{\partial \pazocal{U}}{\partial \dot{q}_j}\dot{q}_j}-(T-U)= 2T -T + U = T+U =E 
\end{equation} \refstepcounter{subsection}

\vspace{-25pt}
\subsection{Invariancia espacial}\refstepcounter{subsection}
Si tenemos un desplazamiento arbitrario en una de las coordenadas generalizadas, $q_k\mapsto q_k + \delta q_k$, y se verifica que $\pazocal{L}(q_k,\{q_j,\dot{q}_j\};t)=\pazocal{L}(q_k+\delta q_k,\{q_j,\dot{q}_j\};t)$, esto implica que la parcial de $\pazocal{L}$ con respecto a $q_k$ es 0. Cuando esto ocurre se dice que $q_k$ es una \textbf{variable ignorable}, y de (2.2.1)  (\textit{E-L}) deducimos que su momento generalizado asociado se conserva.
\begin{equation} \label{3.2.1}
    \frac{d}{dt}\left(\frac{\partial \pazocal{L}}{\partial \dot{q}_k}\right) = \dot{p}_k = 0 \implies p_k = C
\end{equation} \refstepcounter{subsection}
\section{Teorema de Noether} \refstepcounter{subsection}
Consideremos unas transformaciones genéricas $h_j$ de las coordendas $q_j$, parametrizadas por un parámetro $\epsilon$ independiente del tiempo tal que 
\begin{equation} \label{3.3.1}
    q_j \mapsto q'_j=h_j(\{q_i\},\epsilon) \ \ \ \ h_j(\{q_i\},0)=q_j
\end{equation} \refstepcounter{subsection}
\vspace{-40pt}
\subsection{Enunciado} 
Si el conjunto de las transformaciones $h_j$ deja invariante a $\pazocal{L}$ a orden $\epsilon$ (orden uno)
\begin{equation} \label{3.3.2}
    \pazocal{L}(\(\{q_j,\dot{q}_j\};t)+ O(\epsilon^2)=\pazocal{L}(\(h_j(\{q_i\},\epsilon),\dot{h}_j(\{q_i,\dot{q}_i\},\epsilon)\);t)
\end{equation} \refstepcounter{subsection}
Entonces se conseva la siguiente cantidad
\begin{equation} \label{3.3.3}
    I (\{q_j,\dot{q}_j\};t) = \sum_j^s{\frac{\partial \pazocal{L}}{\partial \dot{q}_j}\frac{d h_j}{d\epsilon}} \ \ \ \ \ \frac{d I}{dt} = 0
\end{equation} \refstepcounter{subsection}
Nuestra misión va a ser encontrar las transformaciones (simetrías) que no alteren $\pazocal{L}$ para hallar cantidades conservadas asociadas.

%\subsection{Demostración}

\subsection{Ejemplo} \refstepcounter{subsection}
Si tenemos una masa en en plano bajo la acción de una fuerza central, tal que $\pazocal{L}=1/2m(\dot{x}^2+\dot{y}^2)-U(\sqrt{x^2+y^2})$, si tomamos las transformaciones $x \mapsto x +\epsilon y$ y $y \mapsto y - \epsilon x$, vemos que el lagrangiano se mantiene invariante a orden $\epsilon$.
\[\pazocal{L}'=1/2m((\dot{x} +\epsilon \dot{y})^2+(\dot{y} -\epsilon \dot{x})^2)-U\left(\sqrt{(x +\epsilon y)^2+(y - \epsilon x)^2}\right)\]
\[\pazocal{L}'=1/2m(\dot{x}^2+\dot{y}^2+\cancel{\epsilon^2(\dot{x}^2+\dot{y}^2)}) - U\left(\sqrt{x^2 +y^2 +\cancel{\epsilon^2(x^2+y^2)}}\right)\]
Entonces la cantidad conservada es el momento angular
\[I=\frac{\partial \pazocal{L}}{\partial \dot{x}}\frac{d}{d\epsilon}(x+\epsilon y)+\frac{\partial \pazocal{L}}{\partial \dot{y}}\frac{d}{d\epsilon}(y-\epsilon x)=m(\dot{x}y-\dot{y}x)=-m \mathbf{r} \times \mathbf{v}= -\mathbf{J}_z\]
