\afterpage{\blankpage}

\chapter{Simetrias y cantidades conservadas}
\labch{intro}
\refstepcounter{section}
\section{Ejemplos de invariancias} 
\subsection{Invariancia temporal y energía generalizada}\refstepcounter{subsection}
Si tenemos un desplazamiento arbitratio en el tiempo, $t\mapsto t + \delta t$, y se verifica que $\pazocal{L}(\{q_j,\dot{q}_j\};t)=\pazocal{L}(\{q_j,\dot{q}_j\};t+\delta t)$, esto implica que la parcial de $\pazocal{L}$ con respecto a $t$ es 0. Si ahora desarrollamos la derivadada total de de $\pazocal{L}$ con respecto a $t$, tenemos
\vspace{-13pt}
\[\frac{d \pazocal{L}}{dt}=\sum^s\left(\frac{\partial \pazocal{L}}{\partial q_j}\dot{q}_j+\frac{\partial \pazocal{L}}{\partial \dot{q}_j}\ddot{q}_j\right)+\cancelto{0}{\frac{\partial \pazocal{L}}{\partial t}}\]
El primer término del primer sumando dentro del sumario lo podemos expresar en función de \eqref{2.2.1}, tal que
\[\sum^s\left[\frac{d}{dt}\left(\frac{\partial \pazocal{L}}{\partial \dot{q}_j}\right)\dot{q}_j+\frac{\partial \pazocal{L}}{\partial \dot{q}_j}\ddot{q}_j\right]-\frac{d \pazocal{L}}{dt} = -\frac{\partial \pazocal{L}}{\partial t} =0\]
Ahora lo de dentro del paréntesis es la derivada de un producto, y usando la linearidad de la derivada
\begin{equation} \label{3.1.1}
    \frac{d}{dt}\left(\sum^s \frac{\partial \pazocal{L}}{\partial \dot{q}_j}\dot{q}_j-\pazocal{L}\right) = -\frac{\partial \pazocal{L}}{\partial t} =0
\end{equation} \refstepcounter{subsection}
Definimos entonces la función energía generalizada $H$, que se conservará cuando $\pazocal{L}$ no dependa explícitamente del tiempo.
\begin{equation} \label{3.1.2}
    H \equiv \sum^s \frac{\partial \pazocal{L}}{\partial \dot{q}_j}\dot{q}_j-\pazocal{L} \ \ \ \ \ \frac{d H}{dt}=-\frac{\partial \pazocal{L}}{\partial t}
\end{equation} \refstepcounter{subsection} 
Podemos además observar que si se verifican los supuestos del teorema de la energía cinética (el cambio de coordenadas no depende del tiempo) podemos aplicar \eqref{2.4.10}, y la energía potencial es conservativa, llegamos a $H=E$, es decir, la energía generalizada es igual a la energía clásica.
\vspace{-10pt}
\begin{equation} \label{3.1.3}
    H = \cancelto{2T}{\sum^s\frac{\partial \pazocal{T}}{\partial \dot{q}_j}\dot{q}_j}-\cancelto{0}{\sum^s\frac{\partial \pazocal{U}}{\partial \dot{q}_j}\dot{q}_j}-(T-U)= 2T -T + U = T+U =E 
\end{equation} \refstepcounter{subsection}

\vspace{-25pt}
\subsection{Invariancia espacial}\refstepcounter{subsection}
Si tenemos un desplazamiento arbitrario en una de las coordenadas generalizadas, $q_k\mapsto q_k + \delta q_k$, y se verifica que $\pazocal{L}(q_k,\{q_j,\dot{q}_j\};t)=\pazocal{L}(q_k+\delta q_k,\{q_j,\dot{q}_j\};t)$, esto implica que la parcial de $\pazocal{L}$ con respecto a $q_k$ es 0. Cuando esto ocurre se dice que $q_k$ es una \textbf{variable ignorable}, y de \eqref{2.2.1} deducimos que su momento generalizado asociado se conserva.
\begin{equation} \label{3.1.4}
    \frac{d}{dt}\left(\frac{\partial \pazocal{L}}{\partial \dot{q}_k}\right) = \dot{p}_k = 0 \implies p_k = C
\end{equation} \refstepcounter{subsection}
\refstepcounter{section}
\section{Teorema de Noether} \refstepcounter{subsection}
Consideremos unas transformaciones genéricas $h_j$ de las coordendas $q_j$, además de la transformación $\tau$ de t, parametrizadas por un parámetro $\epsilon$ de cualquier orden
\begin{equation} \label{3.2.1}
    \begin{matrix}
        q_j \mapsto q'_j=h_j(\{q_i\},t,\epsilon) \ \ \ \ h_j(\{q_i\},t,0)=q_j \\
        t \mapsto t'=\tau(\{q_i\},t,\epsilon) \ \ \ \ \tau(\{q_i\},t,0)=t \\
    \end{matrix}
\end{equation} \refstepcounter{subsection}
Si estas transformaciones verifican para una cierta función $B(\{q_j\};t)$
\begin{equation} \label{3.2.2}
    \int_{\tau_A}^{\tau_B} \pazocal{L}(h_j,h'_j;\tau)d \tau=\int_{t_A}^{t_B}\left(\pazocal{L}(\{q_j,\dot{q}_j\};t) + \epsilon\frac{dB}{dt}\right)dt \ \ \ \ \ h'_j = \frac{dh_j}{d\tau}
\end{equation} \refstepcounter{subsection}
Se dice que esas transformaciones son una simetría del sistema. Vamos a expandir las transformaciones en Taylor en función de $\epsilon$
\begin{equation} \label{3.2.3}
    \begin{split}
        q_j' = h_j(\{q_i\},t,\epsilon) &= q_j + \epsilon\left.\frac{\partial h_j}{\partial \epsilon}\right|_{\epsilon=0} \hspace{-10pt}+ O(\epsilon^2) = q_j + \epsilon\psi_j + O(\epsilon^2) \\ 
        t' = \tau(\{q_i\},t,\epsilon) &= t + \epsilon\left.\frac{\partial \tau}{\partial \epsilon}\right|_{\epsilon=0} \hspace{-10pt}+ O(\epsilon^2) = t + \epsilon\phi + O(\epsilon^2)
    \end{split} 
\end{equation} \refstepcounter{subsection}
Ahora vamos a calcular $h'_j$, para ello usaremos la regla de la cadena, el teorema de la función inversa, y expandiremos en serie
\begin{equation} \label{3.2.4}
    \begin{split}
        h'_j& =\frac{d h_j}{d\tau} = \frac{\partial h_j}{\partial t} \frac{\partial t}{\partial \tau} = \frac{\partial h_j}{\partial t} \frac{\partial \tau}{\partial t}^{-1} = \left(\dot{q}_j + \epsilon \dot{\psi}_j + O(\epsilon^2)\right)  \left(1+\epsilon \dot{\phi} + O(\epsilon^2)\right)^{-1} =  \\ 
        & =\left(\dot{q}_j + \epsilon \dot{\psi}_j\right) \left(1-\epsilon\dot{\phi} + O(\epsilon^2)\right) = \dot{q}_j + \epsilon \left[\dot{\psi}_j-\dot{q}_j \dot{\phi}\right] + O(\epsilon^2)
    \end{split} 
\end{equation} \refstepcounter{subsection}
Expandimos en Taylor $\pazocal{L}(h_j,h'_j;\tau)$ en términos de $\epsilon$ usando \eqref{3.2.3} y \eqref{3.2.4}, $\pazocal{L}$ a secas es el lagrangiano origianal sin variar
\begin{equation} \label{3.2.5}
    \begin{split}
        \tilde{\pazocal{L}}=\pazocal{L}(h_j,h'_j;\tau) = \pazocal{L} + \epsilon \left.\frac{d \tilde{\pazocal{L}}}{d\epsilon}\right|_{\epsilon=0} \hspace{-10pt} +O(\epsilon^2) &=  \pazocal{L} + \epsilon\frac{\partial \pazocal{L}}{\partial t} \left.\frac{\partial \tau}{\partial \epsilon}\right|_{\epsilon=0} \hspace{-10pt} + \epsilon\sum_j \left(\frac{\partial \pazocal{L}}{\partial q_j} \left.\frac{\partial h_j}{\partial \epsilon}\right|_{\epsilon=0} \hspace{-10pt} + \frac{\partial \pazocal{L}}{\partial \dot{q}_j} \left.\frac{\partial h'_j}{\partial \epsilon}\right|_{\epsilon=0} \right) +O(\epsilon^2) = \\ 
        &= \pazocal{L} + \epsilon\frac{\partial \pazocal{L}}{\partial t} \phi + \epsilon\sum_j \left[\frac{\partial \pazocal{L}}{\partial q_j} \psi_j + \frac{\partial \pazocal{L}}{\partial \dot{q}_j} \left(\dot{\psi}_j-\dot{q}_j \dot{\phi}\right)\right] +O(\epsilon^2)
    \end{split}
\end{equation} \refstepcounter{subsection}
Tenemos además el siguiente cambio de variable, y nótese que los límites de la integral vuelven a $t_A$ y $t_B$ al hacerlo
\begin{equation} \label{3.2.6}
    d\tau = \frac{d \tau}{d t} dt = (1+\epsilon \dot{\phi} + O(\epsilon^2))dt
\end{equation} \refstepcounter{subsection}
Ahora sustituimos en la parte izquierda de \eqref{3.2.2}
\begin{equation} \label{3.2.7}
    \begin{split}
        \int_{\tau_A}^{\tau_B} \tilde{\pazocal{L}}d \tau &= \int_{t_A}^{t_B}  dt\left(1+\epsilon \dot{\phi} + O(\epsilon^2)\right) \left(\pazocal{L} + \epsilon\frac{\partial \pazocal{L}}{\partial t} \phi + \epsilon\sum_j \left[\frac{\partial \pazocal{L}}{\partial q_j} \psi_j + \frac{\partial \pazocal{L}}{\partial \dot{q}_j} \left(\dot{\psi}_j-\dot{q}_j \dot{\phi}\right)\right] +O(\epsilon^2)\right)= \\ 
        &= \int_{t_A}^{t_B} dt\left[\pazocal{L} +\epsilon \dot{\phi}\pazocal{L} +\epsilon \frac{\partial \pazocal{L}}{\partial t} +\epsilon\sum_j \left(\frac{\partial \pazocal{L}}{\partial q_j} \psi_j + \frac{\partial \pazocal{L}}{\partial \dot{q}_j} \left(\dot{\psi}_j-\dot{q}_j \dot{\phi}\right)\right)\right]  + O(\epsilon^2)
    \end{split}
\end{equation} \refstepcounter{subsection}
Igualamos al término derecho de \eqref{3.2.2}, que nos permite cancelar el primer término de la integral, pasamo el termino de $K$, y sacando $\epsilon$ y dividiendo nos queda
\begin{equation} \label{3.2.8}
    \int_{t_A}^{t_B} dt\left[\dot{\phi}\pazocal{L} +\frac{\partial \pazocal{L}}{\partial t}\phi +\sum_j \left(\frac{\partial \pazocal{L}}{\partial q_j} \psi_j + \frac{\partial \pazocal{L}}{\partial \dot{q}_j} \left(\dot{\psi}_j-\dot{q}_j \dot{\phi}\right)\right)- \frac{dB}{dt}\right] = O(\epsilon)
\end{equation} \refstepcounter{subsection}
Ahora, podemos hacer el límite cuando $\epsilon \rightarrow 0$ a ambos miembros, obteniendo que la integral debe anularse, pues esta no depende de $\epsilon$. Vamos a integrar por partes los términos que van multiplicando a $\dot{\phi}$ y $\dot{\psi}_j$
\begin{equation} \label{3.2.9}
    \begin{split}
        \int_{t_A}^{t_B} dt \dot{\phi}\left(\pazocal{L}-\sum_j \dot{q}_j\frac{\partial \pazocal{L}}{\partial \dot{q}_j}\right) & = \left[\phi \left(\pazocal{L}-\sum_j \dot{q}_j\frac{\partial \pazocal{L}}{\partial \dot{q}_j}\right) \right]_{t_A}^{t_B} - \int_{t_A}^{t_B} dt \phi \frac{d}{dt}\left(\pazocal{L}-\sum_j \dot{q}_j\frac{\partial \pazocal{L}}{\partial \dot{q}_j}\right) \\ 
        \int_{t_A}^{t_B} dt \sum_j \frac{\partial \pazocal{L}}{\partial \dot{q}_j} \dot{\psi}_j & = \left[\sum_j \frac{\partial \pazocal{L}}{\partial \dot{q}_j} \psi_j\right]_{t_A}^{t_B} - \int_{t_A}^{t_B} dt \sum_j \frac{d}{dt}\left(\frac{\partial \pazocal{L}}{\partial \dot{q}_j}\right) \psi_j
    \end{split}  
\end{equation} \refstepcounter{subsection}
Sustituyendo de nuevo en \eqref{3.2.8} tenemos
\begin{equation} \label{3.2.10}
    \begin{split}
        0 =  \left[ \phi \left(\pazocal{L}-\sum_j \dot{q}_j\frac{\partial \pazocal{L}}{\partial \dot{q}_j}\right) +\sum_j \frac{\partial \pazocal{L}}{\partial \dot{q}_j} \psi_j - B\right]_{t_A}^{t_B} & +  \int_{t_A}^{t_B} dt \sum_j \psi_k \left(\frac{\partial \pazocal{L}}{\partial q_j}-\frac{d}{dt}\left(\frac{\partial \pazocal{L}}{\partial \dot{q}_j}\right)\right) \\ 
        &+\int_{t_A}^{t_B} dt \phi\left[\frac{\partial \pazocal{L}}{\partial t} -\frac{d}{dt}\left(\pazocal{L}-\sum_j \dot{q}_j\frac{\partial \pazocal{L}}{\partial \dot{q}_j}\right)\right]
    \end{split} 
\end{equation} \refstepcounter{subsection}
Vemos entonces que el segundo sumando se anula cuando $q_j$ verifican \eqref{2.2.1}, y el tercero es la expresión \eqref{3.1.1}, que se anula también cuando $q_j$ verifican \eqref{2.2.1}. En el primer término sustituimos por \eqref{3.2.3} y \eqref{3.2.2} y, como los límites de integración son arbitrarios, lo contenido en los corchetes debe ser igual $\forall t$, por lo tanto es constante
\begin{equation} \label{3.2.11}
    I = \sum_j \frac{\partial \pazocal{L}}{\partial \dot{q}_j} \psi_j - H\phi -B \ \ \ \ \ \ \psi_j = \left.\frac{\partial h_j}{\partial \epsilon}\right|_{\epsilon=0} \ \ \ \ \ \ \phi = \left.\frac{\partial \tau}{\partial \epsilon}\right|_{\epsilon=0}  \ \ \ \ \ \ \ \frac{dI}{dt} = 0 
\end{equation} \refstepcounter{subsection}
\vspace{-25pt}
\subsection{Resumen}
Sean las siguientes transformaciones (nótese que solo necesitamos conocer $\phi$ y $\psi_j$ para aplicar el teorema)
\vspace{-15pt}
\[
    \begin{split}
        q_j \mapsto \tilde{q}_j(\{q_i\},t,\epsilon) \ \ \ \ \tilde{q}_j(\{q_i\},t,0)=q_j  \ \ \ \ \ \  \psi_j & = \left.\frac{\partial \tilde{q}_j}{\partial \epsilon}\right|_{\epsilon=0} \ \ \ \ \ s_j = q_j + \psi_j \epsilon\\ 
        t \mapsto \tilde{t}(\{q_i\},t,\epsilon) \ \ \ \ \tilde{t}(\{q_i\},t,0)=t \ \ \ \ \ \ \ \ \ \  \phi & = \left.\frac{\partial \tilde{t}}{\partial \epsilon}\right|_{\epsilon=0} \ \ \ \ \ \ \  \tau = t + \phi \epsilon
    \end{split}     
\]

\vspace{-20pt}
Si verifican para una cierta función $B(\{q_j\};t)$
\vspace{-5pt}
\[
    \int_{\tau_A}^{\tau_B} \pazocal{L}(s_j,s'_j;\tau)d \tau=\int_{t_A}^{t_B}\left(\pazocal{L}(\{q_j,\dot{q}_j\};t) + \epsilon\frac{dB}{dt}\right)dt + O(\epsilon^2)\ \ \ \ \ s'_j = \frac{d s_j}{d \tau}
    \]
Entonces se verifica la siguiente expresión para los $q_j$ solución de \eqref{2.2.1}
\[
    I = \sum_j \frac{\partial \pazocal{L}}{\partial \dot{q}_j} \psi_j - H\phi -B \ \ \ \ \ \ \ \ \ \frac{dI}{dt} = 0 
    \]
\subsection{Ejemplo}
Tenemos la acción
\[S = \int_{t_A}^{t_B} -mc^2 \sqrt{1-\frac{\dot{x}^2}{c^2}}dt = \int_{t_A}^{t_B} \frac{-mc^2}{\gamma_0} dt\]
tenemos las transformaciones $\tilde{x} = \gamma(x-\beta c t)$ y $\tilde{t} = \gamma(t-\beta x/c)$, con $\gamma = \sqrt{1-\beta^2}^{-1}$, podemos verificar que dejan invariante la acción, para ello primero calculamos
\[\tilde{\dot{x}}=\frac{d\tilde{x}}{d\tilde{t}} = \frac{\gamma (dx - \beta c dt)}{\gamma(dt-\beta dx/c)} = \frac{\dot{x}-\beta c}{1-\beta \dot{x}/c} = c \frac{\beta_0-\beta}{1-\beta \beta_0} \ \ \ \ \ \beta_0 = \frac{\dot{x}}{c}\]
\[\tilde{S} = \int_{t_A}^{t_B} -mc^2 \sqrt{1-\frac{\tilde{\dot{x}}^2}{c^2}}d\tilde{t} = \int_{t_A}^{t_B} -mc^2 \gamma (1-\beta \beta_0)\sqrt{1-\left(\frac{\beta_0-\beta}{1-\beta \beta_0}\right)^2}  dt = \]
\[= \int_{t_A}^{t_B} -mc^2 \gamma \sqrt{1 + \beta^2 \beta_0^2 -\beta^2-\beta_0^2} dt= \int_{t_A}^{t_B} -mc^2 \gamma \sqrt{1-\beta^2}\sqrt{1-\beta_0^2}dt = \int_{t_A}^{t_B} -mc^2 \sqrt{1-\beta_0^2}dt = S\]
Por lo tanto, podemos aplicar el Teorema de Noether, calculamos 
\[\psi = \left.\frac{\partial \tilde{x}}{\partial \beta}\right|_{\beta = 0} = \left[\frac{\beta}{(1-\beta^2)^{-3/2}}(x-\beta c t) - \frac{ct}{\sqrt{1-\beta^2}} \right]_{\beta = 0} = -ct\]
\[\phi = \left.\frac{\partial \tilde{t}}{\partial \beta}\right|_{\beta = 0} = \left[\frac{\beta}{(1-\beta^2)^{-3/2}}(t-\beta x/c) - \frac{x/c}{\sqrt{1-\beta^2}} \right]_{\beta = 0} = -\frac{x}{c}\]
\[H = \frac{\partial \pazocal{L}}{\partial \dot{x}} \dot{x} - \pazocal{L} = \gamma_0 m \dot{x}^2+\frac{mc^2}{\gamma_0} = \gamma_0 m \left(\dot{x}^2+c^2\left(1-\frac{\dot{x}^2}{c^2}\right)\right) = +\gamma_0 m c^2 \ \ \ \ \ \ \frac{\partial \pazocal{L}}{\partial \dot{x}} = \gamma_0 m \dot{x}\]
Por lo tanto, la cantidad conservada será
\[I =\gamma_0 m c x- \gamma_0 m \dot{x} ct = H \frac{x}{c} - pct\]
Se pueden encontrar las ecuaciones del movimiento aplicando (E-L), que salen $x = x_0 + vt$, entonces $I = \gamma_0 m c x_0$, que efectivamente se conserva, puesto que $x_0$ y $v$ son constantes.

\newpage\null\thispagestyle{empty}\newpage
