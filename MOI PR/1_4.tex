
\section{Ejercicio 1.4}

\fbox{\begin{minipage}{35em}
Escribir las ecuaciones de Euler-Lagrange y los momentos generalizados en coordenadas esféricas para una partícula en movimiento en el espacio a tres dimensiones sometida a una fuerza conservativa Repetir el ejercicio para una fuerza central.        
\end{minipage}}

Si tenemos una fuerza conservativa $\mathbf{F}(\mathbf{r})$ significa que podemos expresarla como $\mathbf{F}(\mathbf{r})=-\nabla V$, donde V es el potencial de la fuerza.

Entonces tomando coordenadas polares en un plano donde esta confinada la partícula, $V=V(r,\theta,\varphi)$. Ahora podemos hallar el \textit{Lagrangiano} del sistema, $L = T-V=\frac{1}{2}mv^2-V$.

Necesitamos encontrar la expresión de $v$ en coordenadas polares, si $\hat{u}_r=\sin{\theta}\cos{\varphi} \hat{u}_x + \sin{\theta}\sin{\varphi} \hat{u}_y + \cos{\theta} \hat{u}_z$+ , entonces si $\mathbf{r}=r \hat{u}_r$, 
\[\dot{\mathbf{r}}=\dot{r} \hat{u}_r + r \frac{d \hat{u}_r}{dt}=\dot{r} \hat{u}_r + \frac{\partial \hat{u}_r}{\partial r} r\dot{r} + \frac{\partial \hat{u}_r}{\partial \theta} r\dot{\theta} + \frac{\partial \hat{u}_r}{\partial \varphi} r\dot{\varphi}=\dot{r} \hat{u}_r+0 + r\dot{\theta} \hat{u}_{\theta} + r\dot{\varphi}\sin{\theta} \hat{u}_{\varphi}\]
\[v^2=\dot{r}\dot{r}={\dot{r}}^2+r^2{\dot{\theta}}^2 + r^2{\dot{\varphi}^2\sin^2{\theta}}; \ \ \ \ L = \frac{1}{2}m({\dot{r}}^2+r^2{\dot{\theta}}^2 + r^2{\dot{\varphi}^2\sin^2{\theta}})-V(r,\theta,\varphi)\]
Ahora podemos aplicar las ecuaciones de \textit{Euler-Lagrange} para hallar la ecuación del movimiento del sistema.
\[\frac{\partial L}{\partial q_i}-\frac{d}{dt}\left(\frac{\partial L}{\partial \dot{q}_i}\right)=0 \ \ \ \ \begin{matrix}
    q_1= r \\ q_2 = \theta \\ q_3 = \varphi
\end{matrix}\]
\[mr(\dot{\theta}^2+\dot{\varphi}^2\sin^2{\theta})-\frac{\partial V}{\partial r}-m\ddot{r}=0 \rightarrow m\ddot{r}- mr(\dot{\theta}^2+\dot{\varphi}^2\sin^2{\theta})=-\frac{\partial V}{\partial r}\]
\[m r^2 \dot{\varphi} \sin{\theta} \cos{\theta} - \frac{\partial V}{\partial \theta }- mr^2\ddot{\theta}-2mr\dot{r}\dot{\theta}=0 \rightarrow\ mr^2\ddot{\theta}+2mr\dot{r}\dot{\theta}-m r^2 \dot{\varphi} \sin{\theta} \cos{\theta} = - \frac{\partial V}{\partial \theta }\]
\[- \frac{\partial V}{\partial \varphi } = 2 m r \dot{r} \dot{\varphi} \sin^2{\theta}+ m r^2 \ddot{\varphi} \sin^2{\theta}+2mr^2 \dot{\varphi} \sin{\theta} \cos{\theta}\]
Si la fuerza es central, implica que $V=V(r)$ y que $\partial_r V \hat{u}_r= \mathbf{F}(r)$, y que $\partial_{\theta} V = \partial_{\varphi} V = 0$

También podemos calcular los momentos generalizados.
\[p_i = \frac{\partial L}{\partial \dot{q}_i}; \ \ \ \ p_r = \frac{\partial L}{\partial \dot{r}}=m\dot{r}; \ \ \ \ p_{\theta} = \frac{\partial L}{\partial \dot{\theta}}=mr^2\dot{\theta}; \ \ \ \ p_{\varphi} = \frac{\partial L}{\partial \dot{\varphi}}=mr^2\dot{\varphi} \sin^2{\theta}\]