% file: main.tex

\documentclass[14pt,a4paper,oneside]{extarticle}
\usepackage{setspace}
\usepackage[spanish]{babel}
\usepackage[utf8]{inputenc}
\usepackage{fancyhdr}
\usepackage{tabularx}
\usepackage{xfrac}
%\renewcommand{\rmdefault}{phv}
%\renewcommand{\sfdefault}{phv}
\usepackage[a4paper,left=2cm,right=2cm,top=2cm,bottom=2cm]{geometry}
\onehalfspacing

\setcounter{tocdepth}{2} % to get subsubsections in toc 
% cf. http://www.latex-community.org/forum/viewtopic.php?f=47&p=44760

\usepackage{amssymb,latexsym}
\usepackage{amsmath, amsthm}
\usepackage{bm}

%for bibliography; installation using 'sudo tlmgr install amsrefs'
\usepackage{amsrefs}

\usepackage{graphics}

\usepackage{hyperref}
\hypersetup{colorlinks=true, urlcolor=blue}

\usepackage{cancel} % http://jansoehlke.com/2010/06/strikethrough-in-latex/

\usepackage{listings} % http://en.wikibooks.org/wiki/LaTeX/Source_Code_Listings
% http://olmjo.com/files/teaching/PSC505/LaTeXandR.pdf

% package for flower symbol (\ding(96))
\usepackage{pifont}
% required installation: sudo apt-get install texlive-fonts-recommended (30MB)
% http://tug.ctan.org/info/symbols/comprehensive/symbols-a4.pdf

\usepackage{tikz} % for diagrams
\usetikzlibrary{matrix,positioning,arrows,calc,decorations.pathmorphing,shapes}
% for snaky lines (http://tex.stackexchange.com/questions/209942/curved-arrows-in-tikz) 
\tikzset{snake it/.style={-stealth,
decoration={snake, 
    amplitude = .4mm,
    segment length = 2mm,
    post length=0.9mm},decorate}}

\usepackage[parfill]{parskip}

\usepackage{framed} %for putting some text in boxes using \begin{framed}

\usepackage{enumerate}

%for displaying tensor indices properly. requires installation of tensor package using 'sudo tlmgr install tensor'
\usepackage{tensor}

%for placing captions of figures on the side instead of above/below the figure
\usepackage{sidecap}
\linespread{1.2}

%plain makes sure that we have page numbers
\pagestyle{plain}

\theoremstyle{plain}
\newtheorem{axiom}{Axioma}
\newtheorem{theorem}{Teorema}
\newtheorem{corollary}{Corolario}
\newtheorem*{main}{Teorema Principal}
\newtheorem{lemma}{Lema}
\newtheorem{proposition}{Proposición}

\theoremstyle{definition}
\newtheorem{definition}{Definición}

\theoremstyle{remark}
\newtheorem*{notation}{Notación}

\numberwithin{equation}{section}
\numberwithin{figure}{section}
\numberwithin{theorem}{section}


\newcommand{\lrg}[1]{\hspace{-10pt} \begin{large} #1 \end{large} \hspace{-10pt}}

%symbol for maps
\renewcommand{\to}{\longrightarrow}
\newcommand{\injmapto}{\hookrightarrow}
\newcommand{\surjmapto}{\twoheadrightarrow}
\newcommand{\linearmapto}{\stackrel{\sim}{\longrightarrow}}
\newcommand{\projmapto}{\stackrel{\pi}{\longrightarrow}}

%for real numbers
\newcommand{\R}{\mathbb{R}}

% manifold, atlas and topology
\newcommand{\A}{\mathcal{A}}
%\newcommand{\O}{\mathcal{O}}
\newcommand{\mfd}{(M, \mathcal{O}, \mathcal{A})}

\newcommand{\after}{\circ}
\newcommand{\stdtop}{\mathcal{O}_{std}}
\newcommand{\cibasis}[2][]{\frac{\partial #1}{\partial #2}}

%connection coefficient functions or gammas
\newcommand{\ccf}[2]{\Gamma\indices{^{#1}_{#2}}}
\newcommand{\ccfx}[3]{\left(\Gamma_{#3}\right)\indices{^{#1}_{#2}}} % with chart index

%set theory symbols
%\renewcommand{\exists}{\exists\,}
%\renewcommand{\forall}{\forall\,}

%This defines a new command \questionhead which takes one argument and prints out Question #. with some space.
\newcommand{\questionhead}[1]
  {
   \noindent{\small\bf Question #1.}
  }

\newcommand{\problemhead}[1]
  {
   \noindent{\small\bf Problem #1.}
  }

\newcommand{\exercisehead}[1]
  { \smallskip
   \noindent{\small\bf Exercise #1.}
  }

\newcommand{\solutionhead}[1]
  {
   \noindent{\small\bf Solution #1.}
  }

\newcommand{\bubblethis}[2]{
  \tikz[remember picture,baseline]{\node[anchor=base,inner sep=0,outer sep=0](#1) {#1};\node[overlay,cloud callout,callout relative pointer={(0.2cm,-0.7cm)}, aspect=2.5,fill=white!90] at ($(#1.north)+(-0.5cm,1.6cm)$) {#2};}
}

%-----------------------------------
\begin{document}

%-----------------------------------

\linespread{1}

\title{Problemas resueltos MO I}
\author{Abel Rosado}
\date{\today}

\maketitle

\tableofcontents

\section{Ejercicio 1.1}

\fbox{\begin{minipage}{35em}
Considerar un medio inhomogéneo con índice de refracción n$(x)$. Implementar el principio de Fermat que establece que entre dos puntos del medio, la luz sigue el camino que extremiza su tiempo de propagación. Usar las ecuaciones de Euler para obtener el camino de la luz para el caso de un medio homogéneo.
\end{minipage}}

Si $ds=\sqrt{1+(y')^2} \ dx$ es la longitud infintesimal, entonces $\frac{ds}{v}$ será el tiempo que tardará la luz en recorrela, $v$ está relacionada con n$(x)$, tal que $v=\frac{c}{\mbox{n}(x)}$, así el tiempo que tarda en recorrer la luz un intervalo $dx$ es
\[ \frac{\mbox{n}(x) \sqrt{1+(y')^2}}{c} \ dx\]
De esta forma, si tenemos el siguiente funcional
\[ F[y] = \int_{x_A}^{x_B}{\frac{\mbox{n}(x) \sqrt{1+(y')^2}}{c} \ dx}\]
podemos hallar su extremo aplicando la ecuación de \textit{Euler-Lagrange} para la función del integrando
\[\frac{\partial g}{\partial y} - \frac{d}{dx}\left(\frac{\partial g}{\partial y'}\right)=0\]
La función no depende explícitamente de $y$, entonces el primer término es 0, y luego el segundo tenemos
\[\frac{\partial g}{\partial y'} = \frac{ y' \mbox{n}(x)}{c \sqrt{1+(y')^2}}\]
Como la derivada de esta expresión debe ser igual a 0, significa que esta expresión es igual a una constante $k$, y manipulando algebraicamente llegamos a la siguiente expresión
\[y' = \frac{kc}{\sqrt{n(x)^2-k^2 c^2}}\]
Esta EDO de variable separable se puede resolver integrando a cada lado
\[y = \int{\frac{kc}{\sqrt{n(x)^2-k^2 c^2}} dx} + \beta\]
Si el medio es homogéneo, n$(x)$=n constante, y entonces el resultado es 
\[y = \alpha x + \beta \ ; \ \ \alpha = \frac{kc}{\sqrt{n^2-k^2 c^2}}\]
\section{Identidad de Beltrami}

Si $\frac{\partial g}{\partial x} = 0$, entonces podemos obtener una expresión equivalente a la ecuación de \textit{Euler-Lagrange} mucho más sencilla que nos facilitará la resolución de problemas. Comenzamos derivando $g$ con respecto a $x$ usando la regla de la cadena
\[\frac{dg}{dx} = \frac{\partial g}{\partial y} y' + \frac{\partial g}{\partial y'}y'' + \frac{\partial g}{\partial x}\]
El tercer término es 0, y podemos fijarnos que el primer término aparece en la ecuación de \textit{Euler-Lagrange} sin multiplicar $y'$. Si multiplicamos toda la ecuación por $y'$ y luego despejamos y sustituimos por lo obtenido es la expresión tenemos
\[\frac{\partial g}{\partial y} y' = \frac{dg}{dx} - \frac{\partial g}{\partial y'}y'' \ ; \ \ \ \ \ \ y' \frac{\partial g}{\partial y} - y' \frac{d}{dx}\left(\frac{\partial g}{\partial y'}\right)=0\]
\[\frac{dg}{dx} - \left[\frac{\partial g}{\partial y'}y'' + y' \frac{d}{dx}\left(\frac{\partial g}{\partial y'}\right)\right]=0\]
Se observa que lo que hay entre paréntesis es la derivada de un producto
\[\frac{\partial g}{\partial y'}y'' + y' \frac{d}{dx}\left(\frac{\partial g}{\partial y'}\right) = \frac{d}{dx}\left(y' \frac{\partial g}{\partial y'}\right)\]
Sustituyendo y aprovechando la linearidad de la derivada llegamos a 
\[\frac{d}{dx}\left(g - y' \frac{\partial g}{\partial y'}\right)=0 \implies \boxed{g - y' \frac{\partial g}{\partial y'}= C}\]
\section{Ejercicio 1.2}

\fbox{\begin{minipage}{35em}
Problema de la braquistócrona. Una partícula puede rodar (sin rozamiento) por efecto de la gravedad entre
los puntos A y B del plano vertical. Encontrar la ecuación de la curva que minimiza el tiempo del camino entre A y B.    
\end{minipage}}

Si $ds=\sqrt{1+(y')^2} \ dx$ es la longitud infintesimal, entonces $\frac{ds}{v}$ será el tiempo que tardará en recorrerse, podemos obtener $v$ mediante la conservación de la energía, si colocamos el sistema de referencia en el punto de inicio y suponemos que se parte del resposo, la energía inicial del sistema es 0, y como esta se conserva, $K=U$, si orientamos el eje $y$ del sistema en el mismo sentido de la gravedad, tenemos que $\frac{1}{2}m v^2 = mgy$, si despejamos v llegamos a $v = \sqrt{2gy}$.

De esta forma, si tenemos el siguiente funcional
\[ F[y] = \int_{x_A}^{x_B}{\sqrt{\frac{1+(y')^2}{2gy}} \ dx}\]
podemos ver que el integrando no depende explícitamente de $x$ y aplicar la Identidad de Beltrami
\[g - y' \frac{\partial g}{\partial y'}= C\]
La derivada del segundo término nos da
\[\frac{\partial g}{\partial y'} = \frac{y'}{\sqrt{2gy(1+(y')^2)}}\]
Sustituyendo y simplificando llegamos a 
\[\sqrt{\frac{1+(y')^2}{2gy}} - \frac{(y')^2}{\sqrt{2gy(1+(y')^2)}}= C \rightarrow \frac{1}{\sqrt{2gy(1+(y')^2)}}=C\]
Podemos despejar $y'$ llegando a la siguiente EDO cuya solución es la curva deseada
\[y' = \sqrt{\frac{1}{2gC^2 y}-1} = \sqrt{\frac{1-ky}{k y}}\]
Donde $2gC^2=k$ y ahora resolvemos separando variables
\[x+\alpha = \int{\sqrt{\frac{ky}{1-k y}}dy}\]
Ahora hacemos el cambio de variable $ky=\sin^2{\theta}$, $kdy=2\sin{\theta}\cos{\theta}d\theta$
\[x+\alpha = \int{\sqrt{\frac{\sin^2{\theta}}{1-\sin^2{\theta}}}\frac{2}{k}\sin{\theta}\cos{\theta}d\theta}=\frac{2}{k}\int{\sqrt{\frac{\sin^2{\theta}}{\cos^2{\theta}}}\sin{\theta}\cos{\theta}d\theta}=\]
\[=\frac{2}{k}\int{\frac{\sin{\theta}}{\cos{\theta}}\sin{\theta}\cos{\theta}d\theta}=\frac{2}{k}\int{\sin^2{\theta} \ d\theta}=\frac{1}{k}\int{1-\cos{2\theta} \ d\theta}=\frac{\theta}{k}-\frac{\sin{2\theta}}{2k}\]
Tenemos entonces las siguientes ecuaciones paramétricas de la curva, que definen un cicloide
\[\begin{matrix}
    x= \frac{\theta}{k}-\frac{\sin{2\theta}}{2k} + \alpha = \frac{1}{2k}(2\theta - \sin{2\theta})+\alpha \\
    y = \frac{1}{k} \sin^2{\theta}=\frac{1}{2k}(1-\cos{2\theta})
\end{matrix}\]
\section{Ejercicio 1.3}

\fbox{\begin{minipage}{35em}
Escribir las ecuaciones de Euler-Lagrange y los momentos generalizados en coordenadas polares para una partícula en movimiento sobre un plano sometida a una fuerza conservativa.    
\end{minipage}}

Si tenemos una fuerza conservativa $\mathbf{F}(\mathbf{r})$ significa que podemos expresarla como $\mathbf{F}(\mathbf{r})=-\nabla V$, donde V es el potencial de la fuerza.

Entonces tomando coordenadas polares en un plano donde esta confinada la partícula, $V=V(r,\theta)$

Ahora podemos hallar el \textit{Lagrangiano} del sistema, $L = T-V=\frac{1}{2}mv^2-V$.

Necesitamos encontrar la expresión de $v$ en coordenadas polares, si $\hat{u}_r=\cos{\theta} \hat{u}_x + \sin{\theta} \hat{u}_y$, entonces si $\mathbf{r}=r \hat{u}_r$, 
\[\dot{\mathbf{r}}=\dot{r} \hat{u}_r + r \frac{d \hat{u}_r}{dt}=\dot{r} \hat{u}_r + \frac{\partial \hat{u}_r}{\partial r} r\dot{r} + \frac{\partial \hat{u}_r}{\partial \theta} r\dot{\theta}=\dot{r} \hat{u}_r+0 + \hat{u}_{\theta} r\dot{\theta}\]
\[v^2=\dot{r}\dot{r}={\dot{r}}^2+r^2{\dot{\theta}}^2; \ \ \ \ L = \frac{1}{2}m({\dot{r}}^2+r^2{\dot{\theta}}^2)-V(r,\theta)\]


Ahora podemos aplicar las ecuaciones de \textit{Euler-Lagrange} para hallar la ecuación del movimiento del sistema.
\[\frac{\partial L}{\partial q_i}-\frac{d}{dt}\left(\frac{\partial L}{\partial \dot{q}_i}\right)=0 \ \ \ \ \begin{matrix}
    q_1= r \\ q_2 = \theta 
\end{matrix}\]
\[mr\dot{\theta}^2-\frac{\partial V}{\partial r}-m\ddot{r}=0; \ \ \ \ \frac{\partial V}{\partial \theta} + mr^2\ddot{\theta}+2mr\dot{r}\dot{\theta}=0\]

También podemos calcular los momentos generalizados.
\[p_i = \frac{\partial L}{\partial \dot{q}_i}; \ \ \ \ p_r = \frac{\partial L}{\partial \dot{r}}=m\dot{r}; \ \ \ \ p_{\theta} = \frac{\partial L}{\partial \dot{\theta}}=mr^2\dot{\theta}\]

\section{Ejercicio 1.4}

\fbox{\begin{minipage}{35em}
Escribir las ecuaciones de Euler-Lagrange y los momentos generalizados en coordenadas esféricas para una partícula en movimiento en el espacio a tres dimensiones sometida a una fuerza conservativa Repetir el ejercicio para una fuerza central.        
\end{minipage}}

Si tenemos una fuerza conservativa $\mathbf{F}(\mathbf{r})$ significa que podemos expresarla como $\mathbf{F}(\mathbf{r})=-\nabla V$, donde V es el potencial de la fuerza.

Entonces tomando coordenadas polares en un plano donde esta confinada la partícula, $V=V(r,\theta,\varphi)$. Ahora podemos hallar el \textit{Lagrangiano} del sistema, $L = T-V=\frac{1}{2}mv^2-V$.

Necesitamos encontrar la expresión de $v$ en coordenadas polares, si $\hat{u}_r=\sin{\theta}\cos{\varphi} \hat{u}_x + \sin{\theta}\sin{\varphi} \hat{u}_y + \cos{\theta} \hat{u}_z$+ , entonces si $\mathbf{r}=r \hat{u}_r$, 
\[\dot{\mathbf{r}}=\dot{r} \hat{u}_r + r \frac{d \hat{u}_r}{dt}=\dot{r} \hat{u}_r + \frac{\partial \hat{u}_r}{\partial r} r\dot{r} + \frac{\partial \hat{u}_r}{\partial \theta} r\dot{\theta} + \frac{\partial \hat{u}_r}{\partial \varphi} r\dot{\varphi}=\dot{r} \hat{u}_r+0 + r\dot{\theta} \hat{u}_{\theta} + r\dot{\varphi}\sin{\theta} \hat{u}_{\varphi}\]
\[v^2=\dot{r}\dot{r}={\dot{r}}^2+r^2{\dot{\theta}}^2 + r^2{\dot{\varphi}^2\sin^2{\theta}}; \ \ \ \ L = \frac{1}{2}m({\dot{r}}^2+r^2{\dot{\theta}}^2 + r^2{\dot{\varphi}^2\sin^2{\theta}})-V(r,\theta,\varphi)\]
Ahora podemos aplicar las ecuaciones de \textit{Euler-Lagrange} para hallar la ecuación del movimiento del sistema.
\[\frac{\partial L}{\partial q_i}-\frac{d}{dt}\left(\frac{\partial L}{\partial \dot{q}_i}\right)=0 \ \ \ \ \begin{matrix}
    q_1= r \\ q_2 = \theta \\ q_3 = \varphi
\end{matrix}\]
\[mr(\dot{\theta}^2+\dot{\varphi}^2\sin^2{\theta})-\frac{\partial V}{\partial r}-m\ddot{r}=0 \rightarrow m\ddot{r}- mr(\dot{\theta}^2+\dot{\varphi}^2\sin^2{\theta})=-\frac{\partial V}{\partial r}\]
\[m r^2 \dot{\varphi} \sin{\theta} \cos{\theta} - \frac{\partial V}{\partial \theta }- mr^2\ddot{\theta}-2mr\dot{r}\dot{\theta}=0 \rightarrow\ mr^2\ddot{\theta}+2mr\dot{r}\dot{\theta}-m r^2 \dot{\varphi} \sin{\theta} \cos{\theta} = - \frac{\partial V}{\partial \theta }\]
\[- \frac{\partial V}{\partial \varphi } = 2 m r \dot{r} \dot{\varphi} \sin^2{\theta}+ m r^2 \ddot{\varphi} \sin^2{\theta}+2mr^2 \dot{\varphi} \sin{\theta} \cos{\theta}\]
Si la fuerza es central, implica que $V=V(r)$ y que $\partial_r V \hat{u}_r= \mathbf{F}(r)$, y que $\partial_{\theta} V = \partial_{\varphi} V = 0$

También podemos calcular los momentos generalizados.
\[p_i = \frac{\partial L}{\partial \dot{q}_i}; \ \ \ \ p_r = \frac{\partial L}{\partial \dot{r}}=m\dot{r}; \ \ \ \ p_{\theta} = \frac{\partial L}{\partial \dot{\theta}}=mr^2\dot{\theta}; \ \ \ \ p_{\varphi} = \frac{\partial L}{\partial \dot{\varphi}}=mr^2\dot{\varphi} \sin^2{\theta}\]
\section{Ejercicio 1.11}

\fbox{\begin{minipage}{35em}
Demostrar que la geodésica entre dos puntos A y B sobre una superficie esférica de centro O, se encuentra sobre la curva formada por la intersección de esta superficie con el plano que pasa por los puntos A, B y O.    
\end{minipage}}

En coordenadas esféricas, fijando $r=R$ constante, tenemos una esfera de centro O y radio $R$, tal que $d\mathbf{l}=Rd\theta \hat{u}_{\theta}+R\sin{\theta} d\varphi \hat{u}_{\varphi}$, así pues tenemos que el diferencial de distancia es $ds^2=R^2(d\theta^2+\sin^2{\theta}d\varphi^2)$.

Si suponemos entonces que $\varphi = \varphi(\theta)$, entonces $d\varphi = \varphi' d\theta$ y podemos expresar $ds$ tal que $ds^2 = R^2(1+\sin^2{\theta}\varphi'^2)d\theta^2$.
Entonces podemos extremizar el siguiente funcional
\[F[\varphi]=\int_A^B{ds}=R\int_{t_A}^{t_B}{\sqrt{1+\sin^2{\theta}\varphi'^2}d\theta}\]
Usando la ecuación de \textit{Euler-Lagrange}
\[\frac{\partial G}{\partial \varphi}-\frac{d}{d\theta}\left(\frac{\partial G}{\partial \varphi'}\right)=0\]
Vemos que $G$ no depende explícitamente de $\varphi$, entonces lo que encontramos dentro de la derivada del segundo término debe ser constante
\[\frac{\partial G}{\partial \varphi'}=\frac{\sin^2{\theta}\varphi'}{\sqrt{1+\sin^2{\theta}\varphi'^2}}=c \rightarrow \varphi'^2(\sin^4{\theta}-c_1\sin^2{\theta})=c_1\]
\[\frac{d\varphi}{d \theta} = \frac{k}{\sin{\theta}\sqrt{\sin^2{\theta}-k^2}} \rightarrow \varphi -\varphi_0= \int{\frac{k}{\sin{\theta}\sqrt{\sin^2{\theta}-k^2}}d\theta}\]
Podemos hacer el cambio de variable $u=k \cot{\theta}$, $du=k\csc^2{\theta}$, si $k/\sin{\theta}=h$ y $h\cos{\theta}=u$, esto implica que $h^2=u^2+k^2$ y entonces $\sin{\theta}=k/\sqrt{u^2+k^2}$.
\[\varphi -\varphi_0= \int{\frac{k}{\sin{\theta}\sqrt{\sin^2{\theta}-k^2}}d\theta}=\int{\frac{k\sin{\theta}}{\sin^2{\theta}\sqrt{\sin^2{\theta}-k^2}}d\theta}=-\int{\frac{\sin{\theta}}{\sqrt{\sin^2{\theta}-k^2}}du}=\]
\[=-\int{\frac{du}{\sqrt{1-\frac{k^2}{\sin^2{\theta}}}}}=-\int{\frac{du}{\sqrt{1-k^2-u^2}}}=-\int{\frac{du}{\sqrt{\alpha^2-u^2}}}=-\frac{1}{\alpha}\int{\frac{du}{\sqrt{1-\frac{u}{\alpha}^2}}}=\]
\[=\arccos{\frac{u}{\alpha}}=\arccos{\beta \frac{\cos\theta}{\sin\theta}}\implies \cos({\varphi -\varphi_0})=\beta \frac{\cos\theta}{\sin\theta}\]
Ahora queremos intersecar un plano que pasa por O $ax+by+cz=0$ con la esfera $x^2+y^2+z^2=R^2$ y comprobar si obtenemos una expresión equivalente.

Tenemos que $x=R\sin{\theta}\cos{\varphi}$, $y=R\sin{\theta}\sin{\varphi}$ y $z=R\cos{\theta}$, sustituyendo en la ecuación del plano tenemos
\[a\sin{\theta}\cos{\varphi}+b\sin{\theta}\sin{\varphi}+c\cos{\theta}=0 \rightarrow a\cos{\varphi}+b\sin{\varphi}=-c\frac{\cos{\theta}}{\sin{\theta}}\]
Usando la identidad de la adición armónica
\[a\cos{\varphi}+b\sin{\varphi}=\sqrt{a^2+b^2}\cos\left({x+\arctan\left({-\frac{b}{a}}\right)}\right)\]
Tal que si $\varphi_0=-\arctan{-a/b}$ y $\beta = -c/\sqrt{a^2+b^2}$
\[\cos({\varphi -\varphi_0})=\beta \frac{\cos\theta}{\sin\theta}\]
Vemos por lo tanto que la intersección entre el plano y la esfera se corresponde a la geodésica de la esfera, lo que se denomina círculo mayor.

\end{document}
