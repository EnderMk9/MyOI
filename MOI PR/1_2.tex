\section{Ejercicio 1.2}

\fbox{\begin{minipage}{35em}
Problema de la braquistócrona. Una partícula puede rodar (sin rozamiento) por efecto de la gravedad entre
los puntos A y B del plano vertical. Encontrar la ecuación de la curva que minimiza el tiempo del camino entre A y B.    
\end{minipage}}

Si $ds=\sqrt{1+(y')^2} \ dx$ es la longitud infintesimal, entonces $\frac{ds}{v}$ será el tiempo que tardará en recorrerse, podemos obtener $v$ mediante la conservación de la energía, si colocamos el sistema de referencia en el punto de inicio y suponemos que se parte del resposo, la energía inicial del sistema es 0, y como esta se conserva, $K=U$, si orientamos el eje $y$ del sistema en el mismo sentido de la gravedad, tenemos que $\frac{1}{2}m v^2 = mgy$, si despejamos v llegamos a $v = \sqrt{2gy}$.

De esta forma, si tenemos el siguiente funcional
\[ F[y] = \int_{x_A}^{x_B}{\sqrt{\frac{1+(y')^2}{2gy}} \ dx}\]
podemos ver que el integrando no depende explícitamente de $x$ y aplicar la Identidad de Beltrami
\[g - y' \frac{\partial g}{\partial y'}= C\]
La derivada del segundo término nos da
\[\frac{\partial g}{\partial y'} = \frac{y'}{\sqrt{2gy(1+(y')^2)}}\]
Sustituyendo y simplificando llegamos a 
\[\sqrt{\frac{1+(y')^2}{2gy}} - \frac{(y')^2}{\sqrt{2gy(1+(y')^2)}}= C \rightarrow \frac{1}{\sqrt{2gy(1+(y')^2)}}=C\]
Podemos despejar $y'$ llegando a la siguiente EDO cuya solución es la curva deseada
\[y' = \sqrt{\frac{1}{2gC^2 y}-1} = \sqrt{\frac{1-ky}{k y}}\]
Donde $2gC^2=k$ y ahora resolvemos separando variables
\[x+\alpha = \int{\sqrt{\frac{ky}{1-k y}}dy}\]
Ahora hacemos el cambio de variable $ky=\sin^2{\theta}$, $kdy=2\sin{\theta}\cos{\theta}d\theta$
\[x+\alpha = \int{\sqrt{\frac{\sin^2{\theta}}{1-\sin^2{\theta}}}\frac{2}{k}\sin{\theta}\cos{\theta}d\theta}=\frac{2}{k}\int{\sqrt{\frac{\sin^2{\theta}}{\cos^2{\theta}}}\sin{\theta}\cos{\theta}d\theta}=\]
\[=\frac{2}{k}\int{\frac{\sin{\theta}}{\cos{\theta}}\sin{\theta}\cos{\theta}d\theta}=\frac{2}{k}\int{\sin^2{\theta} \ d\theta}=\frac{1}{k}\int{1-\cos{2\theta} \ d\theta}=\frac{\theta}{k}-\frac{\sin{2\theta}}{2k}\]
Tenemos entonces las siguientes ecuaciones paramétricas de la curva, que definen un cicloide
\[\begin{matrix}
    x= \frac{\theta}{k}-\frac{\sin{2\theta}}{2k} + \alpha = \frac{1}{2k}(2\theta - \sin{2\theta})+\alpha \\
    y = \frac{1}{k} \sin^2{\theta}=\frac{1}{2k}(1-\cos{2\theta})
\end{matrix}\]