\section{Identidad de Beltrami}

Si $\frac{\partial g}{\partial x} = 0$, entonces podemos obtener una expresión equivalente a la ecuación de \textit{Euler-Lagrange} mucho más sencilla que nos facilitará la resolución de problemas. Comenzamos derivando $g$ con respecto a $x$ usando la regla de la cadena
\[\frac{dg}{dx} = \frac{\partial g}{\partial y} y' + \frac{\partial g}{\partial y'}y'' + \frac{\partial g}{\partial x}\]
El tercer término es 0, y podemos fijarnos que el primer término aparece en la ecuación de \textit{Euler-Lagrange} sin multiplicar $y'$. Si multiplicamos toda la ecuación por $y'$ y luego despejamos y sustituimos por lo obtenido es la expresión tenemos
\[\frac{\partial g}{\partial y} y' = \frac{dg}{dx} - \frac{\partial g}{\partial y'}y'' \ ; \ \ \ \ \ \ y' \frac{\partial g}{\partial y} - y' \frac{d}{dx}\left(\frac{\partial g}{\partial y'}\right)=0\]
\[\frac{dg}{dx} - \left[\frac{\partial g}{\partial y'}y'' + y' \frac{d}{dx}\left(\frac{\partial g}{\partial y'}\right)\right]=0\]
Se observa que lo que hay entre paréntesis es la derivada de un producto
\[\frac{\partial g}{\partial y'}y'' + y' \frac{d}{dx}\left(\frac{\partial g}{\partial y'}\right) = \frac{d}{dx}\left(y' \frac{\partial g}{\partial y'}\right)\]
Sustituyendo y aprovechando la linearidad de la derivada llegamos a 
\[\frac{d}{dx}\left(g - y' \frac{\partial g}{\partial y'}\right)=0 \implies \boxed{g - y' \frac{\partial g}{\partial y'}= C}\]